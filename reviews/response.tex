\documentclass[a4paper,10pt]{article}
%\documentclass[draftcls, onecolumn, 11pt]{IEEEtran}
%\documentclass[journal]{IEEEtran}
 
\usepackage{../mathbf-abbrevs}
%\newcommand {\tbf}[1] {\textbf{#1}}
%\newcommand {\tit}[1] {\textit{#1}}
%\newcommand {\tmd}[1] {\textmd{#1}}
%\newcommand {\trm}[1] {\textrm{#1}}
%\newcommand {\tsc}[1] {\textsc{#1}}
%\newcommand {\tsf}[1] {\textsf{#1}}
%\newcommand {\tsl}[1] {\textsl{#1}}
%\newcommand {\ttt}[1] {\texttt{#1}}
%\newcommand {\tup}[1] {\textup{#1}}
%
%\newcommand {\mbf}[1] {\mathbf{#1}}
%\newcommand {\mmd}[1] {\mathmd{#1}}
%\newcommand {\mrm}[1] {\mathrm{#1}}
%\newcommand {\msc}[1] {\mathsc{#1}}
%\newcommand {\msf}[1] {\mathsf{#1}}
%\newcommand {\msl}[1] {\mathsl{#1}}
%\newcommand {\mtt}[1] {\mathtt{#1}}
%\newcommand {\mup}[1] {\mathup{#1}}

%some math functions and symbols
\newcommand{\reals}{{\mathbb R}}
\newcommand{\ints}{{\mathbb Z}}
\newcommand{\complex}{{\mathbb C}}
\newcommand{\integers}{{\mathbb Z}}
\newcommand{\sign}[1]{\mathtt{sign}\left( #1 \right)}
\newcommand{\NP}{\operatorname{NPt}}
\newcommand{\NS}{\operatorname{NearestSet}}
\newcommand{\bres}{\operatorname{Bres}}
\newcommand{\vol}{\operatorname{vol}}
\newcommand{\vor}{\operatorname{Vor}}
\newcommand{\relevant}{\operatorname{Rel}}

\newcommand{\term}{\emph}
\newcommand{\var}{\operatorname{var}}
\newcommand{\prob}{\operatorname{P}}

%distribution fucntions
\newcommand{\projnorm}{\operatorname{ProjectedNormal}}
\newcommand{\vonmises}{\operatorname{VonMises}}
\newcommand{\wrapnorm}{\operatorname{WrappedNormal}}
\newcommand{\wrapunif}{\operatorname{WrappedUniform}}

\newcommand{\selectindicies}{\operatorname{selectindices}}
\newcommand{\sortindicies}{\operatorname{sortindices}}
\newcommand{\largest}{\operatorname{largest}}
\newcommand{\quickpartition}{\operatorname{quickpartition}}
\newcommand{\quickpartitiontwo}{\operatorname{quickpartition2}}

%some commonly used underlined and
%hated symbols
\newcommand{\uY}{\ushort{\mbf{Y}}}
\newcommand{\ueY}{\ushort{Y}}
\newcommand{\uy}{\ushort{\mbf{y}}}
\newcommand{\uey}{\ushort{y}}
\newcommand{\ux}{\ushort{\mbf{x}}}
\newcommand{\uex}{\ushort{x}}
\newcommand{\uhx}{\ushort{\mbf{\hat{x}}}}
\newcommand{\uehx}{\ushort{\hat{x}}}

% Brackets
\newcommand{\br}[1]{{\left( #1 \right)}}
\newcommand{\sqbr}[1]{{\left[ #1 \right]}}
\newcommand{\cubr}[1]{{\left\{ #1 \right\}}}
\newcommand{\abr}[1]{\left< #1 \right>}
\newcommand{\abs}[1]{{\left| #1 \right|}}
\newcommand{\floor}[1]{{\left\lfloor #1 \right\rfloor}}
\newcommand{\ceiling}[1]{{\left\lceil #1 \right\rceil}}
\newcommand{\ceil}[1]{\lceil #1 \rceil}
\newcommand{\round}[1]{{\left\lfloor #1 \right\rceil}}
\newcommand{\magn}[1]{\left\| #1 \right\|}
\newcommand{\fracpart}[1]{\left< #1 \right>}

% Referencing
\newcommand{\refeqn}[1]{\eqref{#1}}
\newcommand{\reffig}[1]{Figure~\ref{#1}}
\newcommand{\reftable}[1]{Table~\ref{#1}}
\newcommand{\refsec}[1]{Section~\ref{#1}}
\newcommand{\refappendix}[1]{Appendix~\ref{#1}}
\newcommand{\refchapter}[1]{Chapter~\ref{#1}}

\newcommand {\figwidth} {100mm}
\newcommand {\Ref}[1] {Reference~\cite{#1}}
\newcommand {\Sec}[1] {Section~\ref{#1}}
\newcommand {\App}[1] {Appendix~\ref{#1}}
\newcommand {\Chap}[1] {Chapter~\ref{#1}}
\newcommand {\Lem}[1] {Lemma~\ref{#1}}
\newcommand {\Thm}[1] {Theorem~\ref{#1}}
\newcommand {\Cor}[1] {Corollary~\ref{#1}}
\newcommand {\Alg}[1] {Algorithm~\ref{#1}}
\newcommand {\etal} {\emph{~et~al.}}
\newcommand {\bul} {$\bullet$ }   % bullet
\newcommand {\fig}[1] {Figure~\ref{#1}}   % references Figure x
\newcommand {\imp} {$\Rightarrow$}   % implication symbol (default)
\newcommand {\impt} {$\Rightarrow$}   % implication symbol (text mode)
\newcommand {\impm} {\Rightarrow}   % implication symbol (math mode)
\newcommand {\vect}[1] {\mathbf{#1}} 
\newcommand {\hvect}[1] {\hat{\mathbf{#1}}}
\newcommand {\del} {\partial}
\newcommand {\eqn}[1] {Equation~(\ref{#1})} 
\newcommand {\tab}[1] {Table~\ref{#1}} % references Table x
\newcommand {\half} {\frac{1}{2}} 
\newcommand {\ten}[1] {\times10^{#1}}
\newcommand {\bra}[2] {\mbox{}_{#2}\langle #1 |}
\newcommand {\ket}[2] {| #1 \rangle_{#2}}
\newcommand {\Bra}[2] {\mbox{}_{#2}\left.\left\langle #1 \right.\right|}
\newcommand {\Ket}[2] {\left.\left| #1 \right.\right\rangle_{#2}}
\newcommand {\im} {\mathrm{Im}}
\newcommand {\re} {\mathrm{Re}}
\newcommand {\braket}[4] {\mbox{}_{#3}\langle #1 | #2 \rangle_{#4}} 
%\newcommand{\dotprod}[2]{ #1^\prime #2}
\newcommand{\dotprod}[2]{ #1 \cdot #2}
\newcommand {\trace}[1] {\text{tr}\left(#1\right)}

% spell things correctly
\newenvironment{centre}{\begin{center}}{\end{center}}
\newenvironment{itemise}{\begin{itemize}}{\end{itemize}}

%%%%% set up the bibliography style
\bibliographystyle{siam}
%\bibliographystyle{uqthesis}  % uqthesis bibliography style file, made
			      % with makebst

%%%%% optional packages
\usepackage[square,comma,numbers,sort]{natbib}
		% this is the natural sciences bibliography citation
		% style package.  The options here give citations in
		% the text as numbers in square brackets, separated by
		% commas, citations sorted and consecutive citations
		% compressed 
		% output example: [1,4,12-15]

%\usepackage{cite}		
			
\usepackage{units}
	%nice looking units
		
\usepackage{booktabs}
		%creates nice looking tables
		
\usepackage{ifpdf}
\ifpdf
  \usepackage[pdftex]{graphicx}
  %\usepackage{thumbpdf}
  \usepackage[naturalnames]{hyperref}
\else
	\usepackage{graphicx}% standard graphics package for inclusion of
		      % images and eps files into LaTeX document
\fi

\usepackage{amsmath,amsfonts,amssymb} % this is handy for mathematicians and physicists
% see http://www.ams.org/tex/amslatex.html
%\let\proof\relax
%\let\endproof\relax
%\usepackage{amsmath,amsfonts,amssymb, amsthm, bm} % this is handy for mathematicians and physicists
			      % see http://www.ams.org/tex/amslatex.html

		 
%\usepackage[vlined, linesnumbered]{algorithm2e}
	%algorithm writing package
	
\usepackage{mathrsfs}
%fancy math script

%\usepackage{ushort}
%enable good underlining in math mode

%------------------------------------------------------------
% Theorem like environments
%
 \newtheorem{theorem}{Theorem}
% %\theoremstyle{plain}
% \newtheorem{acknowledgement}{Acknowledgement}
% %\newtheorem{algorithm}{Algorithm}
% \newtheorem{axiom}{Axiom}
% \newtheorem{case}{Case}
% \newtheorem{claim}{Claim}
% \newtheorem{conclusion}{Conclusion}
% \newtheorem{observation}{Observation}
% \newtheorem{condition}{Condition}
% \newtheorem{conjecture}{Conjecture}
 \newtheorem{corollary}{Corollary}
% \newtheorem{criterion}{Criterion}
% \newtheorem{definition}{Definition}
% \newtheorem{example}{Example}
% \newtheorem{exercise}{Exercise}
% \newtheorem{lemma}{Lemma}
% \newtheorem{notation}{Notation}
% \newtheorem{problem}{Problem}
% \newtheorem{proposition}{Proposition}
% \newtheorem{remark}{Remark}
% \newtheorem{solution}{Solution}
% \newtheorem{summary}{Summary}
%\numberwithin{equation}{section}
%--------------------------------------------------------


\usepackage{zref-xr,zref-user}
\zexternaldocument*{../paper}

\usepackage[left=2cm,right=2cm,top=2cm,bottom=2cm]{geometry}

\usepackage{amsmath,amsfonts,amssymb, amsthm, bm}

%\usepackage[square,comma,numbers,sort&compress]{natbib}

\usepackage{color}
\newcommand{\comment}[1]{\textcolor{red}{#1}}

\newcommand{\sgn}{\operatorname{sgn}}
\newcommand{\sinc}{\operatorname{sinc}}
\newcommand{\rect}[1]{\operatorname{rect}\left(#1\right)}

%opening
\title{Finding a closest point in a lattice of Voronoi's first kind}
\author{Robby~G.~McKilliam, Alex Grant and I. Vaughan L. Clarkson
%\thanks{}
}

\renewcommand{\theenumii}{(\alph{enumii})}
\renewcommand{\labelenumii}{\theenumii}

\begin{document}

\maketitle

We thank both reviewers for their thoughtful comments.  Below, the citations within the reviewer comments are left as given by the reviewer.  Citations in our responses refer to the references at the end of this document.

%The authors feel the paper has improved as a result.

\section*{Reviewer 1}\label{sec:reviewer-1}

\begin{itemize}

\item\textbf{Comment:} 
From the perspective of quadratic forms, the quadratic forms induced by obtuse superbases are in one to one correspondence with graph Laplacians (this is not
explicitly stated in the paper for some reason). Indeed, the quadratic form $\|B\xbf\|^2$ is exactly $\xbf^T L_G \xbf$, where $L_G$ is the Laplacian of the weighted graph $G$ on vertices $\{ 1, . . . , n + 1 \}$ , with edges weights $w_{ij} = -\bbf_i \cdot \bbf_j \geq 0, i = j$ (which they call the Selling parameters). More precisely:
\[
\|B\xbf\|_2^2 = \xbf^T L_G \xbf = \sum_{ \{i,j\} \in E[G] } w_{ij} (\xbf_{i} - \xbf_{j} )^2.
\]
\\ \textbf{Response:} We thank the reviewer for pointing out the connection with the Laplacian matrix of a graph.  The following text has been included in Section~\zref{sec:comp-clos-relev} to highlight this link.

``At the core of this technique is the fact that a one-to-one correspondence exists between the obtuse superbasis of a lattice of Voronoi's first kind and the~\emph{Laplacian matrix}~\cite{Chung_spectral_graph_theory_1997,Cvetković_spectra_graphs_1998} of a simple weighted graph with $n+1$ vertices and positive edge weights equal to the negated Selling parameters $-q_{ij}$.''

The formula for $\|B\xbf\|_2^2$ stated by the reviewer is known as~\emph{Selling's formula} in the literature on quadratic forms and the geometry of numbers~\cite[Proposition 2.3.1]{Valentin2003_coverings_tilings_low_dimension}~\cite{Selling1874}.  We do indirectly use this formula in the form of the function $\Phi(S,p)$ and our Lemma~\zref{lem:decSellings}.  A very early draft of our paper attempted to use Selling's formula more directly, but we ultimately found that it was $\Phi(S,p)$ that was used repeatedly and not Selling's formula.

\item\textbf{Comment:} 
To present the result from the perspective of Computational Complexity (also not given in the
paper), this paper can be viewed as giving a polynomial time algorithm for the Closest Vector
Problem with Preprocessing (CVPP) over a special class of lattices. In the preprocessing model,
we are allowed to compute any polynomial amount of advice about the lattice (i.e. a good basis, or many short vectors in the lattice, etc.) before answering any CVP queries, where the resources needed to compute this advice are not counted in the runtime of the algorithm. In its
gap decisional version, where we need only decide whether a target is at distance $\leq d$ or $> \alpha d$
(for some approximation factor $\alpha \geq 1$), it is was shown [1, 4] to be NP-Hard for any constant $\alpha = 2^{\log^{1-\epsilon}n}$, and to not be polynomial time solvable for $\alpha = 2 \log n$ , for any fixed $\epsilon > 0$, under the assumption  that $NP \subsetneq \text{RTIME}(2^{\log^{O(1/\epsilon)}n})$ (quasi-polynomial time). For its approximate search version (i.e. find a lattice vector whose distance to the target is some bounded factor away from optimal), an $O(n^{1.5})$ approximation algorithm was (implicitly) given in [5], which was only recently improved to $O( \sqrt{ n / \log n} )$ in [3].
\\\textbf{Response:}

The following paragraph has been included in the introduction to put our results into context with the closest vector problem with preprocessing:

``Our results can be placed in the context of a modification of the closest lattice point problem called the \emph{closest vector problem with preprocessing}~\cite{micciancio_hardness_2001,feige_inapproximability_2004,Regev_2004_inappox_lattice_with_preprocessing,Aleknovish_hardness_with_preprocessing_2011,knot_hardness_cvvp_2014,Dadush_cvp_with_distance_guarantee_2014}.  In this problem some ``advice'' of polynomial size about the lattice is assumed to be given.  The advice might come in the form of a particular basis for the lattice or might take other forms.  The advice may be used to compute a closest lattice point, hopefully with reduced complexity.  %Even this modification of the closest lattice point problem is known to be NP-hard~\cite{micciancio_hardness_2001,feige_inapproximability_2004}.  
Our algorithm can be viewed as an efficient solution for the closest vector problem with preprocessing for the lattices of Voronoi's first kind.  The advice given is the obtuse superbasis.''


\item\textbf{Comment:} 
Hence, from the standpoint of CVPP, the algorithm in this paper gives essentially the first
exact polynomial time algorithm over a reasonably large and natural class of lattices, i.e. those
of Voronoi's first kind. Lastly, removing the need for preprocessing for these lattices, at least for
solving SVP without being given an obtuse superbasis, would require a substantial breakthrough
in the theory of lattice problems (not stated in paper). Indeed, by a simple reduction this would
allow us to decide whether a lattice (given any of its bases) is a rotation of $\ints^n$ (which is clearly
of Voronoi's first kind). This is a long standing open problem in the theory of lattices, which has
only recently been solved in certain very special cases~[6].
\\\textbf{Response:}
The final paragraph of Section~\zref{sec:discussion} has been modified to make the connection with the existing problem of determining whether a rectangular basis exists in a given lattice.

``A final remark is that our algorithm assumes that the obtuse superbasis is known in advance.  It is known that all lattices of dimension less than or equal to 3 are of Voronoi's first kind and an algorithm exists to recover the obtuse superbasis in this case~\cite{SPLAG}.  Lattices of dimension larger than 3 need not be of Voronoi's first kind.  Given a lattice, is it possible to efficiently decide whether it is of Voronoi's first kind?  Is it possible to efficiently find an obtuse superbasis if it exists?  It is suspected that the answer to this second question is no because an efficient solution would yield a solution to a known problem, that of determining whether a lattice is rectangular (has a basis consisting of pairwise orthogonal vectors) given an arbitrary basis~\cite{Lenstra_Silverberg_revisting_gentra_szydlo_2014}.''  %It is suspected that this simpler problem is difficult.''


\item\textbf{Comment:} 
This paper makes a solid contribution to the theory of lattice problems. I think the results here are
well-motivated (I encourage the authors to include some of the motivation mentioned in the summary), and point to the possibility that there might be large and useful classes of lattices where
solving CVP is “easy”.
\\\textbf{Response:}
The following paragraph has been added to Section~\zref{sec:discussion} to suggest that similar fast algorithms for other classes of lattices might exist:

``Another interesting question is: are there subfamilies of Voronoi's first kind that admit even faster algorithms?  Both $A_n$ and $A_n^*$ are examples of this, but there might exist other subfamilies with algorithms faster than $O(n^4)$.  A related question is: can the techniques developed in this paper be applied to other families of lattices, i.e., beyond just those of Voronoi's first kind?''


\item\textbf{Comment:} 
On a technical level, I found the main convergence proof to be interesting, as I've rarely seen
combinatorial arguments successfully applied to lattice problems. On the other hand, while I've
convinced myself the authors have the ``right'' convergence proof, I found its presentation to be
rather awkward and unintuitive. I highly recommend that the authors rephrase the proof using
the language of Laplacians, which would make things more clear (see the additional comments
below for some technical suggestions).
\\\textbf{Response:}
We thank the reviewer for their comments.  We response to the authors other concerns below.

\end{itemize}

We now respond to the reviewers numbered comments.

\begin{enumerate}

\item\textbf{Comment:} 
I believe Proposition 2.1 and 2.2 appear essentially verbatim in [8]. Same for Propositions
3.1 (which is essentially by definition) and 3.2, they appear in [9]. I don’t think it adds any
readability/anything of value to reprove them here.
\\\textbf{Response:}

We agree that these propositions are well known/straightforward, but would prefer to keep them so that the paper is self contained. We do not know of references to specific proofs that we can give for these (unlike Theorem~4.1 for which a proof is directly given in \cite[Theorem~3]{ConwaySloane1992_voronoi_lattice_3d_obtuse_superbases}).  The proof of Proposition 2.1 has been shortened in response to comment~\ref{com:rev2halfopen} of reviewer 2 and the proof of Proposition 2.2 has also been shortened.  A result like Proposition 2.2 does appear informally in~\cite[Sec~5]{Shalvi_iterativeslicer_2009}.  As far as we are aware, the remarkable algorithm in~\cite{MicciancioVoulgaris_deterministic_jv_2013} does not rely upon bounding the number of lattice points inside a sphere (or any other volume).  Instead, the procedure operates recursively on scaled versions of the lattice.  The scaling is chosen to ensure that the target point always lies inside twice the Voronoi cell. 

Motivating Propositions 3.1 and 3.2 is that the stopping criterion in our procedure is slightly different from those in~\cite{Shalvi_iterativeslicer_2009} and~\cite{MicciancioVoulgaris_deterministic_jv_2013}.  The stopping criteria in~\cite{Shalvi_iterativeslicer_2009} and~\cite{MicciancioVoulgaris_deterministic_jv_2013} essentially involve a direct check that a given point, say $x$, is contained inside the Voronoi cell.  That is, one directly asserts that $2 x \cdot v \leq \|v\|^2$ for all relevant vectors $v$.  We cannot use this approach since it would amount to checking this inequality over as many as $2^{n+1}-2$ relevant vectors.  Instead we simply check that $\|y - x_K\| \leq \|y - x_{K+1}\|$.  That is, we check that the procedure converges in the sense that 
\[
\|y - x_K\| = \|y - x_{K+1}\| = \|y - x_{K+2}\| = \dots.
\]
This is a straightforward, but important, difference and it motivates Propositions 3.1 and 3.2.

\newcommand{\rng}{\operatorname{rng}}
\newcommand{\subrng}{\operatorname{subr}}
\newcommand{\decrng}{\operatorname{decrng}}

\item\textbf{Comment:} \label{com:uselaplaicesubr1}
Almost all the inequalities in the paper can be derived from the following simple observation.  For an $n$ vertex weighted graph $G$ with non-negative edge weights, and vectors, $\xbf,\ybf \in \reals^n$, if there exists a permutation $\pi$ of $[n]$ such that $\xbf_{\pi[1]} \leq \dots \leq \xbf_{\pi[n]}$ and $\ybf_{\pi[1]} \leq \dots \leq \ybf_{\pi[n]}$ then
\[
\xbf^T L_G \ybf = \sum_{ \{i,j\} \in E[G] } \wbf_{ij} (\xbf_{i} - \xbf_{j}) (\ybf_{i} - \ybf_{j}) \geq 0.
\]
Furthermore, if $\xbf_{\pi[1]} \leq \dots \leq \xbf_{\pi[n]}$ and $\ybf_{\pi[1]} \geq \dots \geq \ybf_{\pi[n]}$ then the above inequality if reversed, i.e $\xbf^T L_G \ybf \leq 0$. 

I believe using the above will make many of your proofs more transparent.  For example, if you redefine $\subrng(\xbf)$ to return the largest set $S$ such that $\min_{i\in S} \xbf_i - \max_{j \notin S} \xbf_j \geq 1$, then notice that if $S \neq \emptyset$ then there is a permutation that puts both $\xbf - 1_S$ and $1_S$ in non-decreasing order. In particular this implies that
\[
\xbf L_G \xbf = (\xbf - 1_S)^T L_G (\xbf - 1_S) + 2(\xbf - 1_S)^T L_G 1_S + 1_S^T L_G 1_S \geq (\xbf - 1_S)^T L_G (\xbf - 1_S),
\]
i.e. $\xbf$ can be made ``shorter'' by squeezing its components closer together.
\\\textbf{Response:}

The property described by the reviewer is indeed needed to give a proof of Theorem 4.1 (see our response to the next comment number~\ref{com:provethm4.1}).  However, this is not the property that is needed to prove that our algorithm converges in at most $n$ steps.  To see this, let $\subrng_k(p)$ denote the largest set $S$ such that $\min_{i\in S} \xbf_i - \max_{j \notin S} \xbf_j \geq k$. Thus $\subrng_1$ is that suggested by the reviewer and $\subrng_2$ is that which is used in the paper.  Our proof works by considering a $v \in \ints^n$ that can be added to $\floor{z}$ so that $B(\floor{z} + v)$ is a closest lattice point to $y = Bz$.  Our Lemma~\zref{lem:roundzclose} shows that there exists such a $v$ with $\subrng_2(v) = \emptyset$ and so $\rng(v) \leq n$.  It is not true that there always exists such a $v$ with  $\subrng_1(v) = \emptyset$.

To consider this in more detail it is worth looking at the proof of Lemma~\zref{lem:decrngpreservesclosestpoints}.  The lemma shows that $B(\floor{z} + v)$ is a closest point to $y = Bz$ only if $B\big(\floor{z} + \decrng(v)\big)$ is.  A key part of the proof is the line
\[
1 + 2(\zeta_i-\zeta_j) - 2(v_i - v_j) < 1 + 2 - 4 = -1 < 0 \qquad \text{for $i \in S$ and $j \notin S$}
\]
where $\zeta_i = z_i - \floor{z_i} \in [0,1)$.  Since $\zeta_i-\zeta_j \in (-1,1)$ we require that the gap between $v_i - v_j$ to be greater than $\tfrac{3}{2}$ for this inequality to hold.  Because $v \in \ints^{n+1}$ this means $v_i - v_j \geq 2$.  We cannot replace the current $\subrng_2$ with the suggested $\subrng_1$ and have this proof hold because then we would only have $v_i - v_j \geq 1$.

% Very early on (more than 2 years ago now!) when first looking at lattices of Voronoi's first kind, the first author believed that a single iteration of the proceedure would surfice.  That is, compute
% \[
% v = \arg\min_{t \in \{0,1\}^{n+1}}\| B(z - \floor{z} - t) \|^2
% \]
% and then $B(\floor{z} + v)$ would be a closest lattice point.  Numerical counterexamples dispelled this possibility.

\item\textbf{Comment:} \label{com:provethm4.1}
Furthermore, one can use this to give a two line proof of Theorem 4.1.  This theorem I actually do think make sense to reprove here, because modifications of this proof are what drives the convergence bound. 
\\\textbf{Response:}

The property described by the reviewer is indeed needed to give a proof of Theorem 4.1.  To our knowledge, one also needs the property that the relevant vectors of a lattice are given by the shortest coset representatives of $\Lambda / 2\Lambda$.  This result is attributed to Voronoi (see for example \cite[Theorem~2]{ConwaySloane1992_voronoi_lattice_3d_obtuse_superbases} and \cite[Theorem~3.2]{MicciancioVoulgaris_deterministic_jv_2013}).  This is used by Conway and Sloane in~\cite[Theorem~3]{ConwaySloane1992_voronoi_lattice_3d_obtuse_superbases}.

We believe that the proof, whilst not difficult, is ultimately a distraction from our main results.  This would be different if properties needed for the proof, such as the function $\subrng_1$ and Selling's formula were also needed for proving the convergence of our algorithm.  However, as we have stated in comment~\ref{com:uselaplaicesubr1} above, Selling's formula is not directly needed, and $\subrng_2$, not $\subrng_1$ is the function required. 

The order of some paragraphs and lemmas has been changed in Section~\zref{sec:seri-relev-vect} with the intent of improving the readability of the proofs and presenting the intuition behind the proofs before formalities begin.

\item\textbf{Comment:} 
Lemma 5.1: there is a typo here.  The last identity should read
\[
\|B\pbf\|^2 - \|B(\pbf + 1_S - 1_T)\|^2 = \Phi(S,\pbf) + \Phi(\bar{T}, \pbf) + 2\sum_{i \in S}\sum_{j\in S} q_{ij}.
\]
Notice that factor 2 in front of the last term.  There are also two related typos on the top of page 12 where $\sum_{i \in S}\sum_{j\in S} q_{ij}$ (this appears twice on different lines) should be $2\sum_{i \in S}\sum_{j\in T} q_{ij}$.
\\\textbf{Response:}
This error has been fixed.


\item\textbf{Comment:} 
Lemma 5.7: $h$ is ill-defined if $\rng(\vbf) = 0$.  You should explicitly state that $\rng(\vbf) \geq 1$.
\\\textbf{Response:}
Fixed.


\end{enumerate}


\section*{Reviewer 2}\label{sec:reviewer-2}

\begin{enumerate}
\item\textbf{Comment:}  \label{com:cvvp}
page 1, last line ``The closest lattice point problem is known to be NP-hard under certain condition ...'': I found this paragraph very confusing, and somehow inaccurate. What ``certain conditions''? I think a better way to describe the situation regarding the NP-hardness of CVP in relation to this paper is the following. CVP is known to be NP-hard even to approximate within almost polynomial factor (see 35) when the lattice is arbitrary and not known in advance. Interestingly (see 35 and 17) there are families of lattices when CVP remain NP-hard even when given the best possible basis (or any hard-to-compute side information about the lattice). This version of the problem is called CVP with preprocessing (CVPP), and it is particularly relevant to this work, where it is shown that CVPP can be solved exactly and in polynomial time for the class of lattices of Voronoi first kind, when given as auxiliary information a special basis (namely, an obtuse superbasis.) 
\\\textbf{Response:}

The following paragraph has been included in the introduction to put our results into context with the closest vector problem with preprocessing:

``Our results can be placed in the context of a modification of the closest lattice point problem called the \emph{closest vector problem with preprocessing}~\cite{micciancio_hardness_2001,feige_inapproximability_2004,Regev_2004_inappox_lattice_with_preprocessing,Aleknovish_hardness_with_preprocessing_2011,knot_hardness_cvvp_2014,Dadush_cvp_with_distance_guarantee_2014}.  In this problem some ``advice'' of polynomial size about the lattice is assumed to be given.  The advice might come in the form of a particular basis for the lattice or might take other forms.  The advice may be used to compute a closest lattice point, hopefully with reduced complexity.  %Even this modification of the closest lattice point problem is known to be NP-hard~\cite{micciancio_hardness_2001,feige_inapproximability_2004}.  
Our algorithm can be viewed as an efficient solution for the closest vector problem with preprocessing for the lattices of Voronoi's first kind.  The advice given is the obtuse superbasis.''

\item\textbf{Comment:}  
Additional comments about the above paragraph: one author is missing in the reference for 12. Of [1,23,40] only 23 achieves $n^{O(n)}$ running time. (I think the algorithms in 40 take $2^{O(n^2)}$.) There is a journal version of 37 in SIAM J. Computing, and the paper is by two authors (not just Micciancio). 
\\\textbf{Response:}
We have fixed these errors.  This paragraph now reads:

``The closest lattice point problem is known to be NP-hard~\cite{micciancio_hardness_2001, Dinur2003_approximating_CVP_NP_hard,Regev_2004_inappox_lattice_with_preprocessing,feige_inapproximability_2004,Jalden2005_sphere_decoding_complexity}. Nevertheless, algorithms exist that can compute a closest lattice point in reasonable time if the dimension is small~\cite{Pohst_sphere_decoder_1981,Kannan1987_fast_general_np,Agrell2002}.  These algorithms require a number of operations that grows as $O(n^{O(n)})$ or $O(n^{O(n^2)})$ where $n$ is the dimension of the lattice.  Recently, Micciancio and Voulgaris~\cite{MicciancioVoulgaris_deterministic_jv_2013} described a solution for the closest lattice point problem that requires a number of operations that grows as $O(2^{2n})$.  This single exponential growth in complexity is the best known.''

\item\textbf{Comment:}  
page 2, second paragraph: Aren't the results of 3,29 and 30 subsumed by 34? (The authors sure know better than me. A a short clarification may direct the interested reader to the best paper to read.) Also, CVP is always solvable in polynomial time when the dimension is fixed. So, it is unclear the significance of CVP algorithms for Leech lattice (in fixed dimension 24) in the context of this discussion. 
\\\textbf{Response:}
This sentence has been reworded and the reference list has been reduced.  It now reads:

``Although the problem is NP-hard in general, fast algorithms are known for specific highly regular lattices, such the integer lattice $\ints^{n}$, the root lattices $A_n$ and $D_n$, their dual lattices $A_n^*$ and $D_n^*$, and the related Coxeter lattices~\cite[Chap.~4]{SPLAG}\cite{Conway1982FastQuantDec, McKilliam2008, McKilliam2009CoxeterLattices}.''

The interested reader could read these references in the order given. %, i.e., first Chapter 4 of~\cite{SPLAG} and~\cite{Conway1982FastQuantDec} for $D_n$ and $D_n^*$ and older algorithms for $A_n$ and $A_n^*$.  Then~\cite{McKilliam2008} for a fast algorithm for $A_n^*$, then~\cite{McKilliam2009CoxeterLattices} for fast algorithms for all of the Coxeter lattices that includes $A_n$ and $A_n^*$, $E_8$, etc.  Technically, the algorithm from~\cite{McKilliam2009CoxeterLattices} does include that from~\cite{McKilliam2008}.  However~\cite{McKilliam2008} is simpler and shorter to read first.

\item\textbf{Comment:}  
When saying ``if the obtuse superbasis is known'', you may want to cast this requirement in the context of CVPP. (The obtuse superbasis is the advice provided by the preprocessing.) 
\\\textbf{Response:}
Please see our response to comment~\ref{com:cvvp}.

\item\textbf{Comment:}  
page 2,bottom, displayed equation ``$v \cdot x \leq x \cdot x$''. Maybe you mean strict inequality.
\\\textbf{Response:}
%In our original definition we intended to include what Conway and Sloane call the `lax' relevant vectors~\cite{ConwaySloane1992_voronoi_lattice_3d_obtuse_superbases}.  This we not necessary
Yes, this has been fixed.  The text now reads reads:

The \emph{relevant vectors} are those lattice points $v \in \Lambda \backslash  \{\zerobf\}$ for which  
\[
\dotprod{v}{x} < \dotprod{x}{x} \qquad \text{for all $x \in \Lambda \backslash  \{\zerobf\}$}.
\]

% Our definition includes what Conway and Sloane call the `lax' relevant vectors.  To make this clear the following sentences have been added below the definition of the relevant vectors in Section~\zref{sec:voron-cells-relev}:

% ``This definition includes what Conway and Sloane call the `lax' relevant vectors~\cite{ConwaySloane1992_voronoi_lattice_3d_obtuse_superbases}.  If the definition above is replaced by one with strict inequality then only the `strict' relevant vectors would be considered.'' %It will be of benefit to include all of the strict and lax relevant vectors here.''

\item\textbf{Comment:}  
page 3, and maybe elsewhere ``short vector in a lattice'' should probably be ``shortest nonzero vector in a lattice''. 
\\\textbf{Response:}
We have instead introduced the term `short vector' in the introduction.  The following sentence is included in the first paragraph of the introduction: ``This is called the \emph{shortest vector problem} and a solution is called a~\emph{short vector}.''

\item\textbf{Comment:}  
page 3, definition of ceiling(r) as smallest integer strictly larger than 3 seems unconventional. The standard definition is not strict. Write floor(r)+1 instead, which is consistent with the definition of floor(r) given later in the paper. 
\\\textbf{Response:}
Done.

\item\textbf{Comment:}  \label{com:rev2halfopen}
page 3, proof of Prop. 2.1. Your definition of ``half-open Voronoi cell'' is not a proper definition, and also seems incorrect. E.g., take the integer lattice $\ints^n$, then only one of the corners (rather than half) of the corners of the Voronoi cell should be included if you want a tiling. 
\\\textbf{Response:}
The reviewer is correct.  We only require that some modification of the boundary of the closed Voronoi cell results in a tiling.  The proof of Proposition~\zref{eq:latticepointsinvorcvell} has been modified accordingly.

\item\textbf{Comment:}  
page 3, prop 2.2, again definition of ceiling(r) is non-standard. Use floor(r)+1 instead. 
\\\textbf{Response:}
Done. 

\item\textbf{Comment:}  
page 3, section 3. 
The process described in (3.1) is almost identical to those in [45,37], so those papers should probably be referenced at the very beginning, and a comparison provided. As far as I remember, the difference is that 37 picks the v that minimizes the Voronoi norm (i.e., the norm defined by the Voronoi cell of the origin) instead of the Euclidean norm. On the other hand, 45 uses the Euclidean norm, but selects an arbitrary relevant vector that reduces the Euclidean norm, rather than the one that reduces it the most. In this respect, your process is a special case of the one described in 45, and termination (your propositions 3.1 and 3.2) is already proved in 45. 
\\\textbf{Response:}
We have included the following text in the first paragraph of Section~\zref{sec:iterative-slicer} when introducing the iterative algorithm.

``Motivated by Shalvi~et.~al.~\cite{Shalvi_iterativeslicer_2009} and  Micciancio and Voulgaris~\cite{MicciancioVoulgaris_deterministic_jv_2013}, we consider the following iteration,''

Among the iterative procedures in our paper, in~\cite{Shalvi_iterativeslicer_2009}, and in~\cite{MicciancioVoulgaris_deterministic_jv_2013}, ours is the simplest.  The choice of our iteration is not motivated by what might be fastest in general lattices.  It is motivated by what we can prove to be polynomial time for those lattices of Voronoi's first kind.

The~\emph{iterative slicer}~\cite{Shalvi_iterativeslicer_2009} is similar but uses the (presumably accelerated?) iteration
\begin{align*}
x_{k+1} &= x_k + \alpha_k v_k  \\
(v_k, \alpha_k) &= \arg\min_{ (v, \alpha) \in ( \relevant(\Lambda) \cup \{\zerobf\}, \ints) } \|y - x_k - \alpha v \|.
\end{align*} 
So it allows adding integer multiples of a relevant vector.  We do not believe using this will improve the complexity of our algorithm beyond $O(n^4)$.  It would greatly complicate our proofs.

The iterative algorithm of Micciancio and Voulgaris~\cite{MicciancioVoulgaris_deterministic_jv_2013} is different again.  To our (admittedly rough) knowledge, the procedure computes an $\alpha_k$ (different, but related to that above) and then proceeds to compute a closest point in the scaled lattice $\alpha_k \Lambda$.  The scaling $\alpha_k$ is chosen so that the target point lies within $2\alpha_k \vor(\Lambda)$.  Once a closest lattice point in $\alpha_k\Lambda$ is found, the next $\alpha_{k+1}$ is computed and the procedure repeats.

We believe that detailing the differences between these iterative procedures would only detract from the core purpose of our paper to describe an $O(n^4)$ time algorithm to compute a closest point in lattice of Voronoi's first kind.

\item\textbf{Comment:}  
page 5, as before, you should probably mention also Voulgaris when citing 37. 
\\\textbf{Response:}
Done.

\item\textbf{Comment:}  
page 5, last line, ``dimension less than 4'', does it include 4 or not? 
\\\textbf{Response:}
It does not include 4.  We have modified the text here to read ``less than or equal to $3$''. We have also modified similar text in the last paragraph of Section~\zref{sec:discussion} to read ``It is known that all lattices of dimension less than or equal to 3 are of Voronoi's first kind and an algorithm exists to recover the obtuse superbasis in this case~\cite{SPLAG}.  Lattices of dimension larger than 3 need not be of Voronoi's first kind.''

\item\textbf{Comment:}  
page 5, as before, you should probably mention also Voulgaris when citing 37. 
\\\textbf{Response:}
Done.

\item\textbf{Comment:}  
page 6, Theorem 4.1 (this follows from [ 9, Theorem 2+3], not just theorem 3. 
Also, make the theorem statement self contained by adding ``Let $L$ be a lattice of Voronoi first kind with obtuse superbasis $b_1,\dots,b_{n+1}$''. 
\\\textbf{Response:}
We have included the text ``Let $\Lambda$ be a lattice of Voronoi's first kind with obtuse superbasis $b_1,\dots,b_{n+1}$.'' into the statement of Theorem 4.1.

We believe that it is only Theorem 3 from~\cite{ConwaySloane1992_voronoi_lattice_3d_obtuse_superbases} that needs to cited.  Theorem~2 in \cite{ConwaySloane1992_voronoi_lattice_3d_obtuse_superbases} asserts that the relevant vectors in any lattice $\Lambda$ are the shortest coset representatives of the quotient $\Lambda/2\Lambda$.  Theorem~3 uses this, Selling's formula, and the fact that the Selling parameters are nonpositive to prove that the relevant vectors in a lattice of Voronoi's first kind take the form
\[
\sum_{i \in I} b_i
\]
where $I$ is a strict subset of $\{1, 2, \dots, n+1\}$ that is not empty.

%BLERG: Check that Theorems 2 and 3 in the published version of\cite{ConwaySloane1992_voronoi_lattice_3d_obtuse_superbases} are what I think they are!

\item\textbf{Comment:}  
page 8, lemma 5.1. I think you forgot a factor 2 preceding the double summations over S and T (at end of lemma statement, and also in the proof.) 
\\\textbf{Response:}
This error has been fixed.

\item\textbf{Comment:}  
page 11, ``with be'' $\to$ ``will be'' Displayed equation before ``Putting p=...'', A | should be || (5.7) again, missing factor 2 before double summation, and second set should be T, not S. 
\\\textbf{Response:}
These errors have been fixed.

\item\textbf{Comment:}  
page 12, first displayed equation, usual problem: missing factor 2, and summation over S\&S instead of S\&T both on line 2 and 3. 
\\\textbf{Response:}
These errors have been fixed.

\item\textbf{Comment:}  
page 12, last line, definition of $s_i$, missing (-2) factor. 
\\\textbf{Response:}
This error has been fixed.

\item\textbf{Comment:}  
page 13, after (6.1), ``be computing'' $\to$ ``by computing'' 
\\\textbf{Response:}
Fixed.

\item\textbf{Comment:}  
page 14, ``have no effect'' $\to$ ``has no effect'' 
\\\textbf{Response:}
Fixed.

\item\textbf{Comment:}  
page 15, definition of $Q$, should the summation defining $q_{ii}$ be restricted to $j\neq i$? 
\\\textbf{Response:}
Yes. This has been fixed.

\item\textbf{Comment:}  
page 15, ``there many'' $\to$ ``there may'' 
\\\textbf{Response:}
Fixed.

\item\textbf{Comment:}  
References: 
- 12: missing author (R. Raz) 
- 18: ``A V'' $\to$ ``A. V.'' etc, 
- 37: journal version in SIAM J. computing, 42(3):1364-1391 (2013) 
\\\textbf{Response:}
Fixed.

\item\textbf{Comment:}  
Some additional references on CVPP you may find interesting 
(cite at your own discretion): 

- O. Regev (2004) Improved Inapproximability of Lattice and Coding Problems With Preprocessing. IEEE Transactions on Information Theory 50(9): 2031-2037 
- ``Hardness of Approximating the Closest Vector Problem with Pre-Processing'', M. Alekhnovich, S. Khot, G. Kindler, N. Vishnoi, Computational Complexity 20(4):741-753, 2011. 

(Improve NP-hardness result of [35,17] to any constant approximation factor) 

- D. Aharonov and Oded Regev (2005). Lattice problems in NP ∩ co-NP. J. ACM 52, 749-765. 

(Gives polynomial time solution to approximate the distance estimation version of CVPP within $\gamma=\sqrt(n \log n)$ approximation factor, i.e., for any lattice there is an appropriate advice (more complex than just a good basis) that allows to approximate in polynomial time the distance of any target point to the lattice within gamma, but without actually finding a nearby lattice point.) 

- On the Closest Vector Problem with a Distance Guarantee 
D. Dadush, O. Regev, N. Stephens-Davidowitz, Computational Complexity Conference 2014, to appear (next week!) 

I am not sure, as the paper has not appeared yet, but I suspect it is the same as the paper on Dadush webpage with title ``On BDD and CVPP''. 
See link 
http://cs.nyu.edu/~dadush/papers/cvpp.pdf 
and read abstract for more information.
\\\textbf{Response:}
We thank the reviewer for these interesting references highlighting the difficulty of computing close approximations of a closest lattice point.  These have been referenced in Section~\zref{sec:iterative-slicer} with the following text:

``Unfortunately, computing close approximations of a closest lattice point is known to be computationally difficult~\cite{feige_inapproximability_2004,Regev_2004_inappox_lattice_with_preprocessing,Aharonov_Regev_2005,Aleknovish_hardness_with_preprocessing_2011,Dadush_cvp_with_distance_guarantee_2014}.''

\end{enumerate}




\bibstyle{../siam}
%\bibliography{../bib}
\documentclass[final,leqno]{siamltex}
%\documentclass[draftcls, onecolumn, 11pt]{../bib/IEEEtran}
%\documentclass[journal]{../bib/IEEEtran}
%\documentclass[conference]{../bib/IEEEtran}
   
\usepackage{mathbf-abbrevs}

%\newcommand {\tbf}[1] {\textbf{#1}}
%\newcommand {\tit}[1] {\textit{#1}}
%\newcommand {\tmd}[1] {\textmd{#1}}
%\newcommand {\trm}[1] {\textrm{#1}}
%\newcommand {\tsc}[1] {\textsc{#1}}
%\newcommand {\tsf}[1] {\textsf{#1}}
%\newcommand {\tsl}[1] {\textsl{#1}}
%\newcommand {\ttt}[1] {\texttt{#1}}
%\newcommand {\tup}[1] {\textup{#1}}
%
%\newcommand {\mbf}[1] {\mathbf{#1}}
%\newcommand {\mmd}[1] {\mathmd{#1}}
%\newcommand {\mrm}[1] {\mathrm{#1}}
%\newcommand {\msc}[1] {\mathsc{#1}}
%\newcommand {\msf}[1] {\mathsf{#1}}
%\newcommand {\msl}[1] {\mathsl{#1}}
%\newcommand {\mtt}[1] {\mathtt{#1}}
%\newcommand {\mup}[1] {\mathup{#1}}

%some math functions and symbols
\newcommand{\reals}{{\mathbb R}}
\newcommand{\ints}{{\mathbb Z}}
\newcommand{\complex}{{\mathbb C}}
\newcommand{\integers}{{\mathbb Z}}
\newcommand{\sign}[1]{\mathtt{sign}\left( #1 \right)}
\newcommand{\NP}{\operatorname{NPt}}
\newcommand{\NS}{\operatorname{NearestSet}}
\newcommand{\bres}{\operatorname{Bres}}
\newcommand{\vol}{\operatorname{vol}}
\newcommand{\vor}{\operatorname{Vor}}
\newcommand{\relevant}{\operatorname{Rel}}

\newcommand{\term}{\emph}
\newcommand{\var}{\operatorname{var}}
\newcommand{\prob}{\operatorname{P}}

%distribution fucntions
\newcommand{\projnorm}{\operatorname{ProjectedNormal}}
\newcommand{\vonmises}{\operatorname{VonMises}}
\newcommand{\wrapnorm}{\operatorname{WrappedNormal}}
\newcommand{\wrapunif}{\operatorname{WrappedUniform}}

\newcommand{\selectindicies}{\operatorname{selectindices}}
\newcommand{\sortindicies}{\operatorname{sortindices}}
\newcommand{\largest}{\operatorname{largest}}
\newcommand{\quickpartition}{\operatorname{quickpartition}}
\newcommand{\quickpartitiontwo}{\operatorname{quickpartition2}}

%some commonly used underlined and
%hated symbols
\newcommand{\uY}{\ushort{\mbf{Y}}}
\newcommand{\ueY}{\ushort{Y}}
\newcommand{\uy}{\ushort{\mbf{y}}}
\newcommand{\uey}{\ushort{y}}
\newcommand{\ux}{\ushort{\mbf{x}}}
\newcommand{\uex}{\ushort{x}}
\newcommand{\uhx}{\ushort{\mbf{\hat{x}}}}
\newcommand{\uehx}{\ushort{\hat{x}}}

% Brackets
\newcommand{\br}[1]{{\left( #1 \right)}}
\newcommand{\sqbr}[1]{{\left[ #1 \right]}}
\newcommand{\cubr}[1]{{\left\{ #1 \right\}}}
\newcommand{\abr}[1]{\left< #1 \right>}
\newcommand{\abs}[1]{{\left| #1 \right|}}
\newcommand{\floor}[1]{{\left\lfloor #1 \right\rfloor}}
\newcommand{\ceiling}[1]{{\left\lceil #1 \right\rceil}}
\newcommand{\ceil}[1]{\lceil #1 \rceil}
\newcommand{\round}[1]{{\left\lfloor #1 \right\rceil}}
\newcommand{\magn}[1]{\left\| #1 \right\|}
\newcommand{\fracpart}[1]{\left< #1 \right>}

% Referencing
\newcommand{\refeqn}[1]{\eqref{#1}}
\newcommand{\reffig}[1]{Figure~\ref{#1}}
\newcommand{\reftable}[1]{Table~\ref{#1}}
\newcommand{\refsec}[1]{Section~\ref{#1}}
\newcommand{\refappendix}[1]{Appendix~\ref{#1}}
\newcommand{\refchapter}[1]{Chapter~\ref{#1}}

\newcommand {\figwidth} {100mm}
\newcommand {\Ref}[1] {Reference~\cite{#1}}
\newcommand {\Sec}[1] {Section~\ref{#1}}
\newcommand {\App}[1] {Appendix~\ref{#1}}
\newcommand {\Chap}[1] {Chapter~\ref{#1}}
\newcommand {\Lem}[1] {Lemma~\ref{#1}}
\newcommand {\Thm}[1] {Theorem~\ref{#1}}
\newcommand {\Cor}[1] {Corollary~\ref{#1}}
\newcommand {\Alg}[1] {Algorithm~\ref{#1}}
\newcommand {\etal} {\emph{~et~al.}}
\newcommand {\bul} {$\bullet$ }   % bullet
\newcommand {\fig}[1] {Figure~\ref{#1}}   % references Figure x
\newcommand {\imp} {$\Rightarrow$}   % implication symbol (default)
\newcommand {\impt} {$\Rightarrow$}   % implication symbol (text mode)
\newcommand {\impm} {\Rightarrow}   % implication symbol (math mode)
\newcommand {\vect}[1] {\mathbf{#1}} 
\newcommand {\hvect}[1] {\hat{\mathbf{#1}}}
\newcommand {\del} {\partial}
\newcommand {\eqn}[1] {Equation~(\ref{#1})} 
\newcommand {\tab}[1] {Table~\ref{#1}} % references Table x
\newcommand {\half} {\frac{1}{2}} 
\newcommand {\ten}[1] {\times10^{#1}}
\newcommand {\bra}[2] {\mbox{}_{#2}\langle #1 |}
\newcommand {\ket}[2] {| #1 \rangle_{#2}}
\newcommand {\Bra}[2] {\mbox{}_{#2}\left.\left\langle #1 \right.\right|}
\newcommand {\Ket}[2] {\left.\left| #1 \right.\right\rangle_{#2}}
\newcommand {\im} {\mathrm{Im}}
\newcommand {\re} {\mathrm{Re}}
\newcommand {\braket}[4] {\mbox{}_{#3}\langle #1 | #2 \rangle_{#4}} 
%\newcommand{\dotprod}[2]{ #1^\prime #2}
\newcommand{\dotprod}[2]{ #1 \cdot #2}
\newcommand {\trace}[1] {\text{tr}\left(#1\right)}

% spell things correctly
\newenvironment{centre}{\begin{center}}{\end{center}}
\newenvironment{itemise}{\begin{itemize}}{\end{itemize}}

%%%%% set up the bibliography style
\bibliographystyle{siam}
%\bibliographystyle{uqthesis}  % uqthesis bibliography style file, made
			      % with makebst

%%%%% optional packages
\usepackage[square,comma,numbers,sort]{natbib}
		% this is the natural sciences bibliography citation
		% style package.  The options here give citations in
		% the text as numbers in square brackets, separated by
		% commas, citations sorted and consecutive citations
		% compressed 
		% output example: [1,4,12-15]

%\usepackage{cite}		
			
\usepackage{units}
	%nice looking units
		
\usepackage{booktabs}
		%creates nice looking tables
		
\usepackage{ifpdf}
\ifpdf
  \usepackage[pdftex]{graphicx}
  %\usepackage{thumbpdf}
  \usepackage[naturalnames]{hyperref}
\else
	\usepackage{graphicx}% standard graphics package for inclusion of
		      % images and eps files into LaTeX document
\fi

\usepackage{amsmath,amsfonts,amssymb} % this is handy for mathematicians and physicists
% see http://www.ams.org/tex/amslatex.html
%\let\proof\relax
%\let\endproof\relax
%\usepackage{amsmath,amsfonts,amssymb, amsthm, bm} % this is handy for mathematicians and physicists
			      % see http://www.ams.org/tex/amslatex.html

		 
%\usepackage[vlined, linesnumbered]{algorithm2e}
	%algorithm writing package
	
\usepackage{mathrsfs}
%fancy math script

%\usepackage{ushort}
%enable good underlining in math mode

%------------------------------------------------------------
% Theorem like environments
%
 \newtheorem{theorem}{Theorem}
% %\theoremstyle{plain}
% \newtheorem{acknowledgement}{Acknowledgement}
% %\newtheorem{algorithm}{Algorithm}
% \newtheorem{axiom}{Axiom}
% \newtheorem{case}{Case}
% \newtheorem{claim}{Claim}
% \newtheorem{conclusion}{Conclusion}
% \newtheorem{observation}{Observation}
% \newtheorem{condition}{Condition}
% \newtheorem{conjecture}{Conjecture}
 \newtheorem{corollary}{Corollary}
% \newtheorem{criterion}{Criterion}
% \newtheorem{definition}{Definition}
% \newtheorem{example}{Example}
% \newtheorem{exercise}{Exercise}
% \newtheorem{lemma}{Lemma}
% \newtheorem{notation}{Notation}
% \newtheorem{problem}{Problem}
% \newtheorem{proposition}{Proposition}
% \newtheorem{remark}{Remark}
% \newtheorem{solution}{Solution}
% \newtheorem{summary}{Summary}
%\numberwithin{equation}{section}
%--------------------------------------------------------


\title{Finding a closest point in a lattice of Voronoi's first kind}

\author{Robby~G.~McKilliam, Alex~Grant and I.~Vaughan~L.~Clarkson}

\begin{document}

% make the title area 
\maketitle

 \begin{abstract} 
We show that for those lattices of Voronoi's first kind with known obtuse superbasis, a closest lattice point can be computed in $O(n^4)$ operations where $n$ is the dimension of the lattice.  To achieve this a series of relevant lattice vectors that converges to a closest lattice point is found.  We show that the series converges after at most $n$ terms.  Each vector in the series can be efficiently computed in $O(n^3)$ operations using an algorithm to compute a minimum cut in an undirected flow network.  %We discuss potential applications of this algorithm as suggest some future research questions.
\end{abstract}

%\begin{IEEEkeywords}
\begin{keywords}
Lattices, closest point algorithm, closest vector problem.
\end{keywords}
%\end{IEEEkeywords}

% The paper headers
\pagestyle{myheadings}
\thispagestyle{plain}
\markboth{Finding a closest point in a lattice of Voronoi's first kind}{DRAFT \today} 


\section{Introduction}\label{sec:introduction}

An $n$-dimensional \term{lattice} $\Lambda$ is a discrete set of vectors from $\reals^m$, $m \geq n$, formed by the integer linear combinations of a set of linearly independent basis vectors $b_1, \dots, b_n$ from $\reals^m$~\cite{SPLAG}.  That is, $\Lambda$ consists of all those vectors, or \emph{lattice points}, $x \in \reals^m$ satisfying
\[
  x = b_1 u_1 + b_2u_2 + \dots + b_n u_n \qquad u_1, \dots , u_n \in \ints. 
\] 
%In this paper vectors are assumed to be column vectors unless otherwise stated.
Given a lattice $\Lambda$ in $\reals^m$ and a vector $y \in \reals^m$, a problem of interest is to find a lattice point $x \in \Lambda$ such that the squared Euclidean norm
\[
\| y - x \|^2 = \sum_{i=1}^m (y_i - x_i)^2
\] 
is minimised.  This is called the \emph{closest lattice point problem} (or \emph{closest vector problem}) and a solution is called a \emph{closest lattice point} (or simply \emph{closest point}) to $y$. %Equivalently we wish to minimise
%\[
%\| y - \sum_{i=1}^n b_i w_i \|^2
%\]
%over integers $w_1,\dots,w_n$.  
A related problem is to find a lattice point of minimum nonzero Euclidean length, that is, a lattice point of length
\[
\min_{x\in \Lambda \backslash \{ \zerobf \} } \| x \|^2,
\]
where $\Lambda \backslash  \{\zerobf\}$ denotes the set of lattice points not equal to the origin $\zerobf$.  This is called the \emph{shortest vector problem} and a solution is called a~\emph{short vector}.

The closest lattice point problem and the shortest vector problem have interested mathematicians and computer scientists due to their relationship with integer programming~\cite{Lenstra_integerprogramming1983,Kannan1987_fast_general_np,Babai1986}, the factoring of polynomials~\cite{Lenstra1982}, and cryptanalysis~\cite{Joux_toolbox_cryptanal1998,NyguyenStern_two_faces_crypto,Micciancio_lattice_based_post_quantum_crypto}.  
Solutions of the closest lattice point problem have engineering applications.  For example, if a lattice is used as a vector quantiser then the closest lattice point corresponds to the minimum distortion point~\cite{Conway1983VoronoiCodes,Conway1982VoronoiRegions,Conway1982FastQuantDec}.  If the lattice is used as a code, then the closest lattice point corresponds to what is called \emph{lattice decoding}\index{lattice decoding} and has been shown to yield arbitrarily good codes~\cite{Erex2004_lattice_decoding,Erez2005}.  The closest lattice point problem also occurs in communications systems involving multiple antennas~\cite{Ryan2008,Wubben_2011}.  The unwrapping of phase data for location estimation can also be posed as a closest lattice point problem and this has been applied to the global positioning system~\cite{Teunissen_GPS_1995,Hassibi_GPS_1998}.  The problem has also found applications to circular statistics~\cite{McKilliam_mean_dir_est_sq_arc_length2010}, single frequency estimation~\cite{McKilliamFrequencyEstimationByPhaseUnwrapping2009}, and related signal processing problems~\cite{McKilliam2007,Clarkson2007,McKilliam2009IndentifiabliltyAliasingPolyphase,Quinn_sparse_noisy_SSP_2012}.

The closest lattice point problem is known to be NP-hard in general~\cite{micciancio_hardness_2001, Dinur2003_approximating_CVP_NP_hard,Regev_2004_inappox_lattice_with_preprocessing,feige_inapproximability_2004,Jalden2005_sphere_decoding_complexity}. Nevertheless, algorithms exist that can compute a closest lattice point in reasonable time if the dimension is small~\cite{Pohst_sphere_decoder_1981,Kannan1987_fast_general_np,Agrell2002}.  These algorithms require a number of operations that grows as $O(n^{O(n)})$ or $O(n^{O(n^2)})$ where $n$ is the dimension of the lattice.  Recently, Micciancio and Voulgaris~\cite{MicciancioVoulgaris_deterministic_jv_2013} described a solution for the closest lattice point problem that requires a number of operations that grows as $O(2^{2n})$.  This single exponential growth in complexity is the best known. 

Although the problem is NP-hard in general, fast algorithms are known for specific highly regular lattices, such the integer lattice $\ints^{n}$, the root lattices $A_n$ and $D_n$, their dual lattices $A_n^*$ and $D_n^*$, and the related Coxeter lattices~\cite[Chap.~4]{SPLAG}\cite{Conway1982FastQuantDec, McKilliam2008, McKilliam2009CoxeterLattices}.  In this paper we consider a particular class of lattices, those of \emph{Voronoi's first kind}~\cite{ConwaySloane1992_voronoi_lattice_3d_obtuse_superbases,Valentin2003_coverings_tilings_low_dimension,Voronoi1908_main_paper}.  Each lattice of Voronoi's first kind has what is called an \emph{obtuse superbasis}.  We show that if the obtuse superbasis is known, then a closest lattice point can be computed in $O(n^4)$ operations.  This is achieved by enumerating a series of \emph{relevant vectors} of the lattice.  Each relevant vector in the series can be computed in $O(n^3)$ operations using an algorithm for computing a minimum cut in an undirected flow network~\cite{Picard_min_cuts_1974,Sankaran_solving_CDMA_mincut_1998,Ulukus_cdma_mincut_1998,Cormen2001}.  We show that the series converges to a closest lattice point after at most $n$ terms, resulting in $O(n^4)$ operations in total.  This result extends upon a recent result by some of the authors showing that a short vector in a lattice of Voronoi's first kind can be found by computing a minimum cut in a weighted graph~\cite{McKilliam_short_vectors_first_type_isit_2012}.

%BLERG incase I do want to introduce graph Laplacian in the introduction.
%We show that if the obtuse superbasis is known, then a closest lattice point can be computed in $O(n^4)$ operations.  This is achieved by enumerating a series of \emph{relevant vectors} of the lattice.  We show that the series converges to a closest lattice point after at most $n$ terms and, using a connection between the obtuse superbasis and the Laplacian of a graph with positive weights, we show how each relevant vector in the series can be computed in $O(n^3)$ operations by an algorithm that computes a minimum cut in an undirected flow network~\cite{Picard_min_cuts_1974,Sankaran_solving_CDMA_mincut_1998,Ulukus_cdma_mincut_1998,Cormen2001}.

The paper is structured as follows.  Section~\ref{sec:voron-cells-relev} describes the relevant vectors and the \emph{Voronoi cell} of a lattice. Section~\ref{sec:iterative-slicer} describes a procedure to find a closest lattice point by enumerating a series of relevant vectors.  The series is guaranteed to converge to a closest point after a finite number of terms.  In general the procedure might be computationally expensive because the number of terms required might be large and because computation of each relevant vector in the series might be expensive.  Section~\ref{sec:latt-voron-first} describes lattices of Voronoi's first kind and their obtuse superbasis.  In Section~\ref{sec:seri-relev-vect} it is shown that for these lattices, the series of relevant vectors results in a closest lattice point after at most $n$ terms.  Section~\ref{sec:comp-clos-relev} shows that each relevant vector in the series can be computed in $O(n^3)$ operations by computing a minimum cut in an undirected flow network.  Section~\ref{sec:discussion} discusses some potential applications of this algorithm and poses some interesting questions for future research.

\section{Voronoi cells and relevant vectors}\label{sec:voron-cells-relev}
\newcommand{\calR}{\mathcal{R}}
The (closed) \term{Voronoi cell}, denoted $\vor(\Lambda)$, of a lattice $\Lambda$ in $\reals^m$ is the subset of $\reals^m$ containing all points closer or of equal distance (here with respect to the Euclidean norm) to the lattice point at the origin than to any other lattice point. The Voronoi cell is an $m$-dimensional convex polytope that is symmetric about the origin. %Here we will always assume the Euclidean norm (or 2-norm), so $\vor(\Lambda)$ contains those points nearer in Euclidean distance to the origin. %If $x \in \Lambda$ it follows that $\vor(\Lambda) + x$ is the subset of $\reals^n$ that is nearer to $x$ than any other lattice point in $\Lambda$. Figure \ref{lattices:fig:vorregion} is an example of the Voronoi cell.
 
Equivalently, the Voronoi cell can be defined as the intersection of the half spaces 
\begin{align*}
H_{v} &= \{x \in \reals^n \mid \|x\| \leq \|x - v\| \} \\
&= \{x \in \reals^n \mid \dotprod{x}{v} \leq \tfrac{1}{2}\dotprod{v}{v} \}
\end{align*}
for all $v \in \Lambda \backslash  \{\zerobf\}$.  %We denote by $x^\prime$ the transpose of the vector $x$ and so $\dotprod{x}{v}$ is the inner product between column vectors $x$ and $v$.  
We denote by $\dotprod{x}{v}$ the inner product between vectors $x$ and $v$.
It is not necessary to consider all $v \in \Lambda \backslash  \{\zerobf\}$ to define the Voronoi cell.   %Figure \ref{lattices:fig:vorregion} depicts the relevant vectors of the lattice with generator given by~\eqref{lattices:eq:genB}.  A lattice point in $\calR$ is called \term{relevant}. The following remark follow trivially from the preceding discussion.
The \emph{relevant vectors} are those lattice points $v \in \Lambda \backslash  \{\zerobf\}$ for which  
\[
\dotprod{v}{x} \leq \dotprod{x}{x} \qquad \text{for all $x \in \Lambda$}.
\]
This definition includes what Conway and Sloane call the `lax' relevant vectors~\cite{ConwaySloane1992_voronoi_lattice_3d_obtuse_superbases}.  If the definition above is replaced by one with strict inequality then only the `strict' relevant vectors would be considered. It will be of benefit to include all of the strict and lax relevant vectors here.  Denote by $\relevant(\Lambda)$ the set of relevant vectors of the lattice $\Lambda$.  The Voronoi cell is the intersection of the halfspaces corresponding with the relevant vectors, that is, 
\[
%\begin{equation}\label{eq:defvortermsrelvecs}
\vor(\Lambda) = \cap_{v\in\relevant(\Lambda)}{H_{v}}.
%\end{equation}
\]
The closest lattice point problem and the Voronoi cell are related in that $x\in\Lambda$ is a closest lattice point to $y$ if and only $y - x \in \vor(\Lambda)$, that is, if and only if
\begin{equation}\label{eq:relvectnearpointieq}
\dotprod{(y - x)}{v} \leq \tfrac{1}{2} \dotprod{v}{v} \qquad \text{for all $v \in \relevant(\Lambda)$}.  
\end{equation} 

% Every short vector in a lattice is also a relevant vector, because, if a lattice point $s$ is not relevant there exists a lattice point $x$ not equal to $s$ or $\zerobf$ such that $\dotprod{x}{s} \geq \dotprod{x}{x}$ or equivalently $\dotprod{\tfrac{x}{\|x\|}}{s} \geq \|x\|$ after dividing by $\|x\|$.  Since both $\tfrac{x}{\|x\|}$ and $\tfrac{s}{\|s\|}$ are unit vectors and $s \neq x$ then,
% \[
% \|s\| = \dotprod{\tfrac{s}{\|s\|}}{s} >  \dotprod{\tfrac{x}{\|x\|}}{s} \geq \|x\|,
% \]
% i.e, $\|s\| > \|x\|$, so $s$ is not a short vector.  

If $s$ is a short vector in a lattice $\Lambda$ then 
\[
\rho = \frac{\|s\|}{2} = \frac{1}{2} \min_{x \in \Lambda / \{\zerobf\} } \|x\|
\]
is called the \emph{packing radius} (or \emph{inradius}) of $\Lambda$~\cite{SPLAG}.  The packing radius is the minimum distance between the boundary of the Voronoi cell and the origin.  It is also the radius of the largest sphere that can be placed at every lattice point such that no two spheres intersect (see Figure~\ref{lattices:fig:vorregion}).  The following well known results will be useful.

%  \begin{proposition}\label{prop:latticepointinscaledtranslatedvorcell}
%  Let $\Lambda \subset \reals^{m}$ be an $n$-dimensional lattice with Voronoi cell $\vor(\Lambda)$.  Let $D$ be a positive real number and let $\ceil{D}$ denote the smallest integer larger than $D$.  Denote by 
%  \[
%   D\vor(\Lambda) + t = \left\{ x \in \reals^{m} \ \mid \frac{x-t}{D} \in \vor(\Lambda) \right\}
%  \]
% the Voronoi cell scaled by $D$ and translated by $t \in \reals^{m}$.  The number of lattice points from $\Lambda$ in $ D\vor(\Lambda) + t$ is less than or equal to $\ceil{D}^{n}$. 
%  \end{proposition}
%  \begin{proof}
% See, for example,~\cite[Lemma~3.7]{Micciancio09adeterministic}.  The result also follows from~\cite[vol.~134, p.~277]{Voronoi1908_main_paper} or \cite[Theorem~2]{ConwaySloane1992_voronoi_lattice_3d_obtuse_superbases}.
%  \end{proof}

\begin{proposition}\label{eq:latticepointsinvorcvell}
Let $\Lambda \subset \reals^{m}$ be an $n$-dimensional lattice.  For $r\in \reals$ let $\floor{r}$ denote the largest integer less than or equal to $r$.  Let $t \in \reals^m$.  The number of lattice points inside the scaled and translated Voronoi cell $r \vor(\Lambda) + t$ is at most $(\floor{r}+1)^n$.
\end{proposition}
\begin{proof}
%It is convenient to modify the boundary of the Voronoi cell so that it tessellates $\reals^m$ under translations by $\Lambda$.  With this aim we 
Let $V \subset \vor(\Lambda)$ contain all those points from the interior of the closed Voronoi cell $\vor(\Lambda)$, but with boundaries defined so that $V$ tessellates $\reals^m$ under translations by $\Lambda$.  That is, $\reals^{m} = \cup_{x \in \Lambda}(V + x)$ and the intersection $(V + x)\cap(V+y)$ is empty for distinct lattice points $x$ and $y$.  For positive integer $k$, the scaled and translated cell $kV + t$ contains precisely one coset representative for each element of the quotient group $\Lambda/k\Lambda$~\cite[Sec.~2.4]{McKilliam2010thesis}.  There are $k^n$ coset representatives.  Thus, the number of lattice points inside $r \vor(\Lambda) + t \subset (\floor{r}+1)V + t$ is at most $(\floor{r}+1)^n$.
\end{proof}
 

\begin{proposition}\label{eq:latticepointsinsphere}
Let $\Lambda \subset \reals^{m}$ be an $n$-dimensional lattice with packing radius $\rho$.  Let $S$ be an $m$-dimensional hypersphere of radius $r$ centered at $t \in \reals^{m}$.  The number of lattice points from $\Lambda$ in the sphere $S$ is at most $(\floor{r/\rho}+1)^{n}$. %where $\ceil{\cdot}$ denotes the smallest integer strictly larger than its argument. 
 \end{proposition}
 \begin{proof}
The packing radius $\rho$ is the Euclidean length of a point on the boundary of the Voronoi cell $\vor(\Lambda)$ that is closest to the origin. Therefore, the sphere $S$ is a subset of $\vor(\Lambda)$ scaled by $r/\rho$ and translated by $t$.  That is, $S \subset r/\rho \vor(\Lambda) + t$.  The proof follows because the number of lattice points in $r/\rho \vor(\Lambda) + t$ is at most $(\floor{r/\rho}+1)^n$ by Proposition~\ref{eq:latticepointsinvorcvell}.
\end{proof}
 

% \footnote{These are the `strict' relevant vectors according to Conway and Sloane~\cite{ConwaySloane1992_voronoi_lattice_3d_obtuse_superbases}.  If the inequality $\dotprod{v}{x} < \dotprod{x}{x}$ is replaced by $\dotprod{v}{x} \leq \dotprod{x}{x}$ then this would also include the `lax' relevant vectors.  The short vectors are always strict so we only have use of the strict relevant vectors here.}
 
\begin{figure}[p] 
	\centering      
		\includegraphics{figs/latticefigures-1.mps} 
		\caption{The $2$-dimensional lattice with basis vectors $(3,0.6)$ and $(0.6,3)$.  The lattice points are represented by dots and the relevant vectors are circled.  The Voronoi cell $\vor(\Lambda)$ is the shaded region and the packing radius $\rho$ and corresponding sphere packing (circles) are depicted.
}     
		\label{lattices:fig:vorregion}   
\end{figure} 

\begin{figure}[p] 
	\centering      
		\includegraphics{figs/iterativeseriesexample-1.mps} 
		\caption{Example of the iterative procedure described in~\eqref{eq:relvectsminimslicer} to compute a closest lattice point to $y = (4,3.5)$ (marked with a cross) in the $2$-dimensional lattice generated by basis vectors $(2,0.4)$ and $(0.4,2)$.  The initial lattice point for the iteration is $x_0 = (-4.4,-2.8)$.  The shaded region is the Voronoi cell surrounding the closest lattice point $x_6 = (4.4,2.8)$.}       
		\label{lattices:fig:iterativeexample} 
\end{figure} 


\section{Finding a closest lattice point by a series of relevant vectors} \label{sec:iterative-slicer}

Let $\Lambda$ be a lattice in $\reals^m$ and let $y \in \reals^m$. A simple method to compute a lattice point $x \in \Lambda$ closest to $y$ is as follows.  Let $x_0$ be some lattice point from $\Lambda$, for example the origin.  Motivated by Shalvi~et.~al.~\cite{Shalvi_iterativeslicer_2009} and  Micciancio and Voulgaris~\cite{MicciancioVoulgaris_deterministic_jv_2013}, we consider the following iteration,
\begin{align}
x_{k+1} &= x_k + v_k \nonumber \\
v_k &= \arg\min_{ v \in \relevant(\Lambda) \cup \{\zerobf\} } \|y - x_k - v \|, \label{eq:relvectsminimslicer}
\end{align} 
where $\relevant(\Lambda) \cup \{\zerobf\}$ is the set of relevant vectors of $\Lambda$ including the origin.  The minimum over $\relevant(\Lambda) \cup \{\zerobf\}$ may not be unique, that is, there may be multiple vectors from $\relevant(\Lambda) \cup \{\zerobf\}$ that are closest to $y - x_k$.  In this case, any one of the minimisers may be chosen.  The results that we will describe do not depend on this choice. We make the following straightforward propositions.

\begin{proposition}\label{obs:1}
%After a finite number, say $K$ iterations, the above proceedure The above proceedure reaches a stationary point after a finite number $K$ of iterations
At the $k$th iteration either $x_k$ is a closest lattice point to $y$ or $\|y - x_k\| > \| y - x_{k+1} \|$.
\end{proposition}
\begin{proof}
If $x_k$ is a closest lattice point to $y$ then $\|y - x_k\| \leq \| y - x_{k+1} \|$ by definition.  On the other hand if $x_k$ is not a closest lattice point to $y$ we have $y - x_k \notin \vor(\Lambda)$ and from~\eqref{eq:relvectnearpointieq} there exists a relevant vector $v$ such that
\[
0 > \dotprod{v}{v} - 2\dotprod{(y - x_k)}{v}.
\]
Adding $\|y - x_k\|^2$ to both sides of this inequality gives
\begin{align*}
\|y - x_k\|^2 &> \dotprod{v}{v} - 2\dotprod{(y - x_k)}{v} + \|y - x_k\|^2 \\
&= \|y - x_k - v\|^2 \\
&\geq \arg\min_{ v \in \relevant(\Lambda) \cup \{\zerobf\}}\|y - x_k - v \|^2 \\
&= \|y - x_k - v_k \|^2 \\
&= \|y - x_{k+1}\|^2. 
\end{align*}
\end{proof} 

% \begin{observation}\label{obs:1}
% At the $k$th iteration either $x_k$ is a closest lattice point to $y$ or 
% \[
%  \max_{ v \in \relevant(\Lambda)}\frac{\dotprod{(y - x_k)}{v}}{\dotprod{v}{v}} > \frac{1}{2}.
% \]
% \end{observation}
% \begin{proof}
% If $x_k$ is a closest lattice point to $y$ then $\dotprod{(y - x)}{v} \leq \tfrac{1}{2} \dotprod{v}{v}$ for all $v \in \relevant(\Lambda)$ by definition of the Voronoi cell~\eqref{eq:relvecsvorcellnearpoint} and so
% \[
%  \max_{ v \in \relevant(\Lambda)}\frac{\dotprod{(y - x_k)}{v}}{\dotprod{v}{v}} \leq \frac{1}{2}.
% \]
% Conversely if $x_k$ is not the closest point to $y$ there exists a relevant vector $r$ such that $\dotprod{(y - x)}{r} > \tfrac{1}{2} \dotprod{r}{r}$ and so
% \[
%  \max_{ v \in \relevant(\Lambda)}\frac{\dotprod{(y - x_k)}{v}}{\dotprod{v}{v}} \geq \frac{\dotprod{(y - x_k)}{r}}{\dotprod{r}{r}} > \frac{1}{2}.
% \]
% \end{proof}

 \begin{proposition}\label{obs:2}
 There is a finite number $K$ such that $x_K, x_{K+1}, x_{K+2}, \dots$ are all closest points to $y$.
 \end{proposition}
 \begin{proof}
Suppose no such finite $K$ exists, then
\[
\|y - x_0\| >  \|y - x_1\| > \|y - x_2\| > \dots
\]
and so $x_0,x_1,\dots$ is an infinite sequence of distinct (due to the strict inequality) lattice points  all contained inside an $n$-dimensional hypersphere of radius $r = \|y - x_0\|$ centered at $y$.  This is a contradiction because, if $\rho$ is the packing radius of the lattice, then less than $(\floor{r/\rho}+1)^n$ lattice points lie inside this sphere by Proposition~\ref{eq:latticepointsinsphere}. 
\end{proof}
 
Proposition~\ref{obs:2} asserts that after some finite number $K$ of iterations the procedure arrives at $x_K$, a closest lattice point to $y$.  Using Proposition~\ref{obs:1} we can detect that $x_K$ is a closest lattice point by checking whether $\|y - x_K\| \leq \| y - x_{K+1} \|$.
This simple iterative approach to compute a closest lattice point is related to what is called the \emph{iterative slicer}~\cite{Shalvi_iterativeslicer_2009}.  Micciancio and Voulgaris~\cite{MicciancioVoulgaris_deterministic_jv_2013} describe a related, but more sophisticated, iterative algorithm that can compute a closest point in a number of operations that grows exponentially as $O(2^{2 n})$.  This single exponential growth in complexity is the best known.  %Most popular algorithms for computed closest lattice points require a number of operations of order $O(n^n)$\cite{Agrell2002,Viterbo_sphere_decoder_1999,Pohst_sphere_decoder_1981}. 

Two factors contribute to the computational complexity of this iterative approach to compute a closest lattice point.  The first factor is computing the minimum over the set $\relevant(\Lambda) \cup \{\zerobf\}$ in~\eqref{eq:relvectsminimslicer}.  In general a lattice can have as many as $2^{n+1}-2$ relevant vectors so computing a minimiser directly can require a number of operations that grows exponentially with $n$.  To add to this it is often the case that the set of relevant vectors $\relevant(\Lambda)$ must be stored in memory so the algorithm can require an amount of memory that grows exponentially with $n$~\cite{MicciancioVoulgaris_deterministic_jv_2013}\cite{Shalvi_iterativeslicer_2009}.  We will show that for a lattice of Voronoi's first kind the set of relevant vectors has a compact representation in terms of what is called its \emph{obtuse superbasis}.  To store the obtuse superbasis requires an amount of memory of order $O(n^2)$ in the worst case.  We also show that for a lattice of Voronoi's first kind the minimisation over $\relevant(\Lambda) \cup \{\zerobf\}$ in~\eqref{eq:relvectsminimslicer} can be solved efficiently by computing a minimum cut in an undirected flow network.  Using known algorithms a minimiser can be computed in $O(n^3)$ operations~\cite{Goldberg:1986:NAM:12130.12144,EdmondsKarp_max_flow,Cormen2001}. 

The other factor affecting the complexity is the number of iterations required before the algorithm arrives at a closest lattice point, that is, the size of $K$.  Proposition~\ref{eq:latticepointsinsphere} suggests that this number might be as large as $(\floor{r/\rho}+1)^n$ where $r = \|y - x_0\|^2$ and $\rho$ is the packing radius of the lattice.  Thus, the number of iterations required might grow exponentially with $n$.  The number of iterations required depends on the lattice point that starts the iteration $x_0$.  It is helpful for $x_0$ to be, in some sense, a close approximation of the closest point $x_K$.  Unfortunately, computing close approximations of a closest lattice point is known to be computationally difficult~\cite{feige_inapproximability_2004,Regev_2004_inappox_lattice_with_preprocessing,Aharonov_Regev_2005,Aleknovish_hardness_with_preprocessing_2011,Dadush_cvp_with_distance_guarantee_2014}.  We will show that for a lattice of Voronoi's first kind a simple and easy to compute choice for $x_0$ ensures that a closest lattice point is reached in at most $n$ iterations and so $K \leq n$.  Combining this with the fact that each iteration of the algorithm requires $O(n^3)$ operations results in an algorithm that requires $O(n^4)$ operations to compute a closest point in a lattice of Voronoi's first kind. 


\section{Lattices of Voronoi's first kind} \label{sec:latt-voron-first}

An $n$-dimensional lattice $\Lambda$ is said to be of \emph{Voronoi's first kind} if it has what is called an \emph{obtuse superbasis}~\cite{ConwaySloane1992_voronoi_lattice_3d_obtuse_superbases}.  That is, there exists a set of $n+1$ vectors $b_1,\dots,b_{n+1}$ such that $b_1,\dots,b_n$ are a basis for $\Lambda$,
\begin{equation}\label{eq:superbasecond}
b_1 + b_2 \dots + b_{n+1} = 0
\end{equation}
(the \emph{superbasis} condition), and the inner products satisfy
\begin{equation}\label{eq:obtusecond}
q_{ij} = b_i \cdot b_j \leq 0, \qquad \text{for} \qquad i,j = 1,\dots,n+1, i \neq j
\end{equation}
(the \emph{obtuse} condition).  The $q_{ij}$ are called the \emph{Selling parameters}~\cite{Selling1874}.  It is known that all lattices in dimensions less than or equal to $3$ are of Voronoi's first kind~\cite{ConwaySloane1992_voronoi_lattice_3d_obtuse_superbases}.  
%%%%%%%%%%%%%%%%%%%%
% There is a one-to-one correspondence between the obtuse superbasis of a lattice of Vornoi's first kind and what is called the \emph{Laplacian} of a weighted graph.  To see this, let
% \[
% B = (b_1\,\,b_2\,\,\dots\,\,b_{n+1})
% \]
% be the matrix with columns given by $b_1,\dots,b_{n+1}$ and define the $n+1$ by $n+1$ symmetric \emph{Gram matrix} $Q = B^\prime B$.  The elements of $Q$ are given by the Selling parameters, that is, $Q_{ij} = q_{ij}$.  Because of the superbasis condition the row and column sums of $Q$ are zero.  In other words, $Q\onebf = \zerobf$ and $\onebf^\prime Q = \zerobf^\prime$ where $\onebf$ and $\zerobf$ denote column vectors of ones and zeros respectively.  Let $G$ be a graph with $n+1$ vertices $v_1,\dots,v_{n+1}$ and edges $e_{ij}$ connecting $v_i$ to $v_j$.  To each edge assign the \emph{weight} $w_{ij} = -q_{ij} \leq 0$.  Let $d_{ii} = \sum_{j \neq i} w_{ij}$ be the sum of weights of edges connected to vertex $v_i$. The Laplacian of the graph $G$ is the $n+1$ by $n+1$ matrix with elements
% \[
% L_{ij} = \begin{cases}
% d_{ii} & i = j \\
% -w_{ij} & i \neq j. 
% \end{cases}
% \]
%%%%%%%%%%%%%%%%%%%
% Lattices of Vornoi's first kind have some other interesting properties.  Let
% \[
% B = (b_1\,\,b_2\,\,\dots\,\,b_{n+1})
% \]
% be the matrix with columns given by $b_1,\dots,b_{n+1}$ and define the $n+1$ by $n+1$ symmetric \emph{Gram matrix} $Q = B^\prime B$ where superscript $^\prime$ indicates the vector or matrix transpose.  The elements of $Q$ are given by the Selling parameters, that is, $Q_{ij} = q_{ij}$.  Because of the superbasis condition the row and column sums of $Q$ are zero, i.e., 
% \[
% \sum_{i = 1}^n q_{ij} = \sum_{j=1}^n q_{ij} = 0.
% \]  
% Equivalently, $Q\onebf = \zerobf$ and $\onebf^\prime Q = \zerobf^\prime$ where $\onebf$ and $\zerobf$ denote column vectors of ones and zeros respectively.  The following formula will be useful.  For $x\in\reals^{n+1}$ and $y \in \reals^{n+1}$, the quadratic form
% \begin{align}
% x^\prime Q y &= \frac{1}{2} ( x^\prime Qy + y^\prime Qx ) \nonumber \\ 
% &= \frac{1}{2} \sum_{i=1}^{n+1} \sum_{j=1}^{n+1} q_{ij}(x_iy_j + x_jy_i) \nonumber \\
% &= \frac{1}{2} \sum_{i=1}^{n+1} \sum_{j=1}^{n+1} q_{ij}\big( x_iy_i + x_jy_j - (x_i - x_j)(y_i - y_j) \big) \nonumber \\
% &= -\sum_{i=1}^{n+1} \sum_{j=i+1}^{n+1} q_{ij}(x_i - x_j)(y_i - y_j). \label{eq:extendedsellings}
% \end{align}
% The first line in the above equation follows from the symmetry of $Q$.  The last line follows because
% \[
% \sum_{i=1}^{n+1} \sum_{j=1}^{n+1} q_{ij}x_jy_j = \sum_{i=1}^{n+1} \sum_{j=1}^{n+1} q_{ij}x_iy_i = \sum_{i=1}^{n+1} x_iy_i \sum_{j=1}^{n+1} q_{ij} = 0
% \]
% because the row and column sums of $Q$ are zero.  Putting $y = x$ yields what is called~\emph{Selling's formula}~\cite{Selling1874,ConwaySloane1992_voronoi_lattice_3d_obtuse_superbases}\cite[Proposition~2.3.1]{Valentin2003_coverings_tilings_low_dimension},
% \begin{equation}\label{eq:sellings}
% \|Bx\|^2 = x^\prime Q x = -\sum_{i=1}^{n+1} \sum_{j=i+1}^{n+1} q_{ij}(x_i - x_j)^2.
% \end{equation}
%%%%%%%%%%%%%%%%%%%%%
An interesting property of lattices of Voronoi's first kind is that their relevant vectors have a straightforward description.

\begin{theorem} \label{thm:revvecssuperbase} (Conway and Sloane~\cite[Theorem~3]{ConwaySloane1992_voronoi_lattice_3d_obtuse_superbases})
The relevant vectors of $\Lambda$ are of the form,
\[
\sum_{i \in I} b_i
\]
where $I$ is a strict subset of $\{1, 2, \dots, n+1\}$ that is not empty, i.e., $I \subset \{1, 2, \dots, n+1\}$ and $I \neq \emptyset$.
\end{theorem}  
 
% \begin{align*}
% x^\prime Q y &= \sum_{i=1}^n \sum_{j=1}^n q_{ij} x_i y_j \\
% &= \sum_{i=1}^n \sum_{j=1}^n q_{ij} x_i y_j  - 2
% \end{align*}
% \[
% (x-y)^\prime Q (x-y) = \sum_{i=1}^n \sum_{j=1}^n q_{ij} (x_i-y_i)^2
% \]

%BLERG: It will be instructive for us to give a proof of this theorem.  In doing so we will introduce some of the notation that will be used in Section~\ref{}.

% A number of important lattices are of Voronoi's first kind.  For example, the \emph{root lattices} $\ints^n, A_n, E_6, E_7$ and $E_8$ are of Voronoi's first kind~\cite{SPLAG}.  The root lattice $E_8$ has obtuse superbasis
% \[

% \]

% The root lattice $D_n$ has obtuse superbasis given by the vectors
% \begin{align*}
% b_1 &= -e_1 - e_2 \\
% b_i &= e_{i} - e_{i+1}, \qquad i = 2,\dots,n \\
% b_{n+1} &= e_{2} + e_{n}.
% \end{align*}

Classical examples of lattices of Voronoi's first kind are the $n$ dimensional root lattice $A_n$ and its dual lattice $A_n^*$~\cite{SPLAG}.  % These lattices have convenient bases represented in $n+1$ dimensional space.  An obtuse superbasis for $A_n$ is of the form
% \begin{align*}
% b_i &= e_i - e_{i+1}, \qquad i = 1, \dots, n \\
% b_{n+1} &= e_{n+1} - e_{1},
% \end{align*}
% where $e_i$ is a vector of length $n+1$ with elements equal to zero except for the $i$th element that is equal to one.  The dual lattice $A_{n}^*$ is also of Voronoi's first kind.  The $i$th vector $b_i$ of the obtuse superbasis has $j$th element equal to
% \[
% b_{ij} = \begin{cases}
% \frac{n}{n+1} & j = i \\
% -\frac{1}{n+1} & j \neq i.
% \end{cases}
% \]
For $A_n$ and $A_n^*$ there exist efficient algorithms that can compute a closest lattice point in $O(n)$ operations~\cite{McKilliam2009CoxeterLattices,Conway1982FastQuantDec}. For this reason we do not recommend using the algorithm described in this paper for $A_n$ and $A_n^*$.  
The fast algorithms for $A_n$ and $A_n^*$ rely of the special structure of these lattices and are not applicable to other lattices.  In contrast, the algorithm we describe here works for all lattices of Voronoi's first kind.  Questions that arise are: how ``large'' (in some sense) is the set of lattices of Voronoi's first kind?  Are there lattices of Voronoi's first kind that are useful in applications such as coding, quantisation, or signal processing?  We discuss these questions in Section~\ref{sec:discussion}.  We now focus on the problem of computing a closest lattice point in a lattice of Voronoi's first kind. 

%However, there exist a large number of lattice of first
%A lattice of Voronoi's first kind appears nontrivial.  One approach would be to construct the matrix $Q = B^\prime B$ containing the Selling parameters $q_{ij}$.  The matrix $Q$ requires to have nonpositive off diagonal entries and to be positive semidefinite with null space containing the vector $\onebf$ of all ones.  The explicit construction of such matrices is an interesting question for future research.  A related question is to determine, in some sense, how `large' the set of matrices satisfying these criteria is.  This would shed some light on how `large' the set of lattices of Voronoi's first kind is.  


\section{A series of relevant vectors from a lattice of Voronoi's first kind}\label{sec:seri-relev-vect}

We are interested in solving the closest lattice point problem for lattices of Voronoi's first kind.  Let $\Lambda \subset \reals^m$ be an $n$ dimensional lattice of Voronoi's first kind with obtuse superbasis $b_1,\dots,b_{n+1}$ and let $y \in \reals^m$.  We want to find $n$ integers $w_1,\dots,w_n$ that minimise
\[
\| y - \sum_{i=1}^n b_i w_i \|^2.
\]
We can equivalently find $n+1$ integers $w_1,\dots,w_{n+1}$ that minimise
\[
\| y - \sum_{i=1}^{n+1} b_i w_i \|^2.
\]
The iterative procedure described in~\eqref{eq:relvectsminimslicer} will be used to do this.  In what follows it is assumed that $y$ lies in the space spanned by the basis vectors $b_1,\dots,b_{n}$.  This assumption is without loss of generality because $x$ is a closest lattice point to $y$ if and only if $x$ is a the closest lattice point to the orthogonal projection of $y$ into the space spanned by $b_1,\dots,b_{n}$.  Let
\[
B = (b_1\,\,b_2\,\,\dots\,\,b_{n+1})
\]
be the $m$ by $n+1$ matrix with columns given by $b_1,\dots,b_{n+1}$ and let $z \in \reals^{n+1}$ be a column vector such that $y = Bz$.  We now want to find a column vector $w = (w_1,\dots,w_{n+1})^\prime$ of integers such that
\begin{equation}\label{eq:tominimise}
\| B(z  -  w) \|^2
\end{equation}
is minimised.  Define the column vector $u_0 = \floor{z}$ where $\floor{\cdot}$ operates on vectors elementwise. In view of Theorem~\ref{thm:revvecssuperbase} the iterative procedure~\eqref{eq:relvectsminimslicer} to compute a closest lattice point can be written in the form
\begin{align}
x_{k+1} &= B u_{k+1} \label{eq:xseqfirsttype}  \\
u_{k+1} &= u_k + t_k \nonumber \\
t_k &= \arg\min_{t \in \{0,1\}^{n+1}}\| B(z - u_k - t) \|^2, \label{eq:pvecmin}
\end{align}
where $\{0,1\}^{n+1}$ denotes the set of column vectors of length $n+1$ with elements equal to zero or one.  The procedure is initialised at the lattice point $x_0 = Bu_0 = B\floor{z}$.  This choice of initial lattice point is important.  In Section~\ref{sec:comp-clos-relev} we show how minimisation over $\{0,1\}^{n+1}$ in~\eqref{eq:pvecmin} can be computed efficiently in $O(n^3)$ operations by computing a minimum cut in an undirected flow network.  The minimiser may not be unique corresponding with the existence of multiple minimum cuts.  In this case any one of the minimisers may be chosen.  Our results do not depend on this choice.   In the remainder of this section we prove that this iterative procedure results in a closest lattice point after at most $n$ iterations.  That is, we show that there exists a positive integer $K \leq n$ such that $x_K$ is a closest lattice point to $y = Bz$.

\newcommand{\rng}{\operatorname{rng}}
\newcommand{\subrng}{\operatorname{subr}}
\newcommand{\decrng}{\operatorname{decrng}}

It is necessary to introduce some notation.  For $S$ a subset of indices $\{1,\dots,n+1\}$ let $\onebf_S$ denote the column vector of length $n+1$ with $i$th element equal to one if $i \in S$ and zero otherwise. %Thus, for example, if $p \in \reals^{n+1}$, then $p+\onebf_S$ denotes the vector $p$ with those elements having index from $S$ incremented by one.  
For $S \subseteq \{1,\dots,n+1\}$ and $p \in \reals^{n+1}$ we define the function
\[
\Phi(S, p) = \sum_{i \in S}\sum_{j \notin S}q_{ij}(1 + 2p_i - 2p_j)
\]
where $q_{ij} = \dotprod{b_i}{b_j}$ are the Selling parameters from~\eqref{eq:obtusecond}.  We denote by $\bar{S}$ the complement of the set of indices $S$, that is $\bar{S} = \{ i \in \{1,\dots,n+1\} \mid i \notin S\}$. 

\begin{lemma}\label{lem:decSellings}
Let $p \in \reals^{n+1}$ and let $S$ and $T$ be subsets of the indices of $p$.  The following equalities hold:
\begin{enumerate}
\item  $\|Bp\|^2 - \|B(p + \onebf_S)\|^2 = \Phi(S, p)$, \label{eq:lem:decSellingsinc}
\item  $\|Bp\|^2 - \|B(p - \onebf_S)\|^2 = \Phi(\bar{S}, p)$, \label{eq:lem:decSellingsdec}
\item  ${\displaystyle \|Bp\|^2 - \|B(p + \onebf_S - \onebf_T)\|^2 = \Phi(S, p) + \Phi(\bar{T},p) + 2\sum_{i\in S}\sum_{j\in T} q_{ij} }$.  \label{eq:lem:decSellingsincdec}
\end{enumerate}
\end{lemma}
\begin{proof}
Part~\ref{eq:lem:decSellingsincdec} follows immediately from parts~\ref{eq:lem:decSellingsinc} and~\ref{eq:lem:decSellingsdec} because
\begin{align*}
\|Bp\|^2 - &\|B(p + \onebf_S - \onebf_T)\|^2 \\
&= \|Bp\|^2-\|B(p + \onebf_S)\|^2 +  \|Bp\|^2-\|B(p - \onebf_T)\|^2 + 2\sum_{i\in S}\sum_{j\in T} q_{ij}.
\end{align*}
We give a proof for part~\ref{eq:lem:decSellingsinc}.  The proof for part~\ref{eq:lem:decSellingsdec} is similar.  
Put $Q = B^\prime B$ where superscript $^\prime$ indicates the vector or matrix transpose.  The $n+1$ by $n+1$ matrix $Q$ has elements given by the Selling parameters, that is, $Q_{ij} = q_{ij}=\dotprod{b_i}{b_j}$.  Denote by $\onebf$ the column vector of length $n+1$ containing all ones.  Now $B\onebf = \sum_{i=1}^{n+1}b_i = \zerobf$ as a result of the superbasis condition~\eqref{eq:superbasecond} and so $Q\onebf = \zerobf$.  Since $\onebf_S = \onebf - \onebf_{\bar{S}}$ it follows that $Q\onebf_S = -Q\onebf_{\bar{S}}$.  With $\circ$ the elementwise vector product, i.e., the Schur or Hadamard product, we have
 \begin{align*}
 \|Bp\|^2 - \|B(p + \onebf_S)\|^2 &= -\onebf^\prime_S Q \onebf_S - 2 p^\prime Q \onebf_S \\
 &= \onebf^\prime_S Q \onebf_{\bar{S}} - 2 p^\prime Q \onebf_S \\
&= \onebf^\prime_S Q \onebf_{\bar{S}} - 2 (p\circ\onebf_{\bar{S}})^\prime Q \onebf_S - 2 (p\circ\onebf_S)^\prime Q \onebf_S \\
&= \onebf^\prime_S Q \onebf_{\bar{S}} - 2 (p\circ\onebf_{\bar{S}})^\prime Q \onebf_{S} + 2 (p\circ\onebf_S)^\prime Q \onebf_{\bar{S}}
 \end{align*}
which is precisely $\Phi(S,p)$.
% % Observe that
% % \[
% % \sum_{i=1}^{n+1}\sum_{j=1}^{n+1} q_{ij} p_i^2 =  \sum_{i=1}^{n+1}p_i^2\sum_{j=1}^{n+1} q_{ij} = 0
% % \]
% % and
% % \[
% % \sum_{i=1}^{n+1}\sum_{j=1}^{n+1} q_{ij} p_j^2 =  \sum_{j=1}^{n+1}p_j^2\sum_{i=1}^{n+1} q_{ij} = 0
% % \]
% % since $\sum_{i=1}^{n+1} q_{ij} = \sum_{j=1}^{n+1} q_{ij} = 0$ as a result of the superbasis condition~\eqref{eq:superbasecond}.  Now
% % \begin{align*}
% % \|Bp\|^2 &= \sum_{i=1}^{n+1} \sum_{j=1}^{n+1} q_{ij} p_i p_j \\
% % &= \sum_{i=1}^{n+1} \sum_{j=1}^{n+1} q_{ij} (\tfrac{1}{2}p_j^2 + p_i p_j -  \tfrac{1}{2}p_i^2) \\
% % &= -\frac{1}{2} \sum_{i=1}^{n+1} \sum_{j=1}^{n+1} q_{ij} (p_i - p_j)^2.
% % \end{align*}
% % This is closely related to what is known as Sellings formula~\cite[Proposition 2.3.1]{Valentin2003_coverings_tilings_low_dimension}~\cite{Selling1874}.  Put $g = p - \onebf_S$.  It follows that
% % \[
% % \|Bp\|^2 - \|Bg\|^2 = \frac{1}{2} \sum_{i=1}^{n+1}\sum_{j=1}^{n+1}q_{ij}d_{ij},
% % \]
% % where $d_{ij} = (g_i - g_j)^2 - (p_i - p_j)^2$.  If both $i \notin S$ and $j \notin S$ then $p_i = g_i$, $p_j = g_j$ and so $d_{ij} = 0$.  Similarly if both $i \in S$ and $j \in S$ then 
% % \[
% % d_{ij} = (p_i-1 - p_j+1)^2 - (p_i - p_j)^2 = 0.
% % \]
% % If $i \in S$ and $j \notin S$ then
% % \[
% % d_{ij} = (p_i-1 - p_j)^2 - (p_i - p_j)^2= 1 + 2p_j - 2p_i,
% % \]
% % while if $i \notin S$ and $j \in S$ then
% % \[
% % d_{ij} = (p_i - p_j+1)^2 - (p_i - p_j)^2 = 1 + 2p_i - 2p_j.
% % \]
% % We thus have
% % \begin{align*}
% % \|Bp\|^2 - \|Bg\|^2 &= \frac{1}{2} \sum_{i \in S}\sum_{j \notin S}q_{ij}(1 + 2p_j - 2p_i) + \frac{1}{2} \sum_{i \notin S}\sum_{j \in S}q_{ij}(1 + 2p_i - 2p_j) \\
% % &= \sum_{i \in S}\sum_{j \notin S}q_{ij}(1 + 2p_j - 2p_i),
% % \end{align*}
% % where the last line holds because $q_{ij} = q_{ji}$.
\end{proof}

% \begin{lemma}\label{lem:decSellingsincanddec}
% With the same symbols as in Lemma~\ref{lem:decSellings} and with $T$ a subset of the indices of $p$,
% \[
% \|Bp\|^2 - \|B(p + \onebf_S + \onebf_T)\|^2 = \sum_{i \in S}\sum_{j \notin S}q_{ij}(1 + 2p_i - 2p_j) + \sum_{i \in T}\sum_{j \notin T}q_{ij}(1 + 2p_i - 2p_j) + \sum_{i \in S}\sum_{j \in T} 2q_{ij}
% \]
% \end{lemma}

Denote by $\min(p)$ and $\max(p)$ the minimum and maximum values obtained by the elements of the vector $p$ and define the function
\[
\rng(p) = \max(p) - \min(p).
\] 
%to return the difference between the maximum and minimum of $p$.  %For example, if $p = (2,-1,4) \in \ints^3$ then $\rng(p) = 4 - (-1)=5$.  We might also 
%For example $\rng(2,-1,4)= 4 - (-1) = 5$.  
Observe that $\rng(p)$ cannot be negative and that if $\rng(p) = 0$ then all of the elements of $p$ are equal.  We define the function $\subrng(p)$ to return the largest subset, say $S$, of the indices of $p$ such that $\min\{p_i, i \in S\} - \max\{p_i, i \notin S\} \geq 2.$  If no such subset exists then $\subrng(p)$ is the empty set $\emptyset$.  For example, 
\[
\subrng(2,-1,4) = \{1,3\}, \qquad \subrng(2,1,3) = \emptyset, \quad  \subrng(1,3,1) = \{2\}.
\]  
To make the definition of $\subrng$ clear we give the following alternative and equivalent definition.  Let $p \in \reals^n$ and let $\sigma$ be the permutation of the indices $\{1,\dots,n\}$ that puts the elements of $p$ in ascending order, that is
\[
p_{\sigma(1)} \leq p_{\sigma(2)} \leq \dots \leq p_{\sigma(n)}.
\]  
Let $T$ be the smallest integer from $\{2,\dots,n\}$ such that $p_{\sigma(T)} - p_{\sigma(T-1)} \geq 2$.  If no such integer $T$ exists then $\subrng(p) = \emptyset$.  Otherwise 
\[
\subrng(p) =  \{ \sigma(T), \sigma(T+1), \dots, \sigma(n) \}.
\]
The following straightforward property of $\subrng$ will be useful.

\begin{proposition}\label{prop:subrrngsmall}
Let $p \in \ints^{n+1}$.  If $\subrng(p) = \emptyset$ then $\rng(p) \leq n$.
\end{proposition}
\begin{proof}
Let $\sigma$ be the permutation of the indices $\{1,\dots,n+1\}$ that puts the elements of $p$ in ascending order.  Because $\subrng(p) = \emptyset$ and because the elements of $p$ are integers we have $p_{\sigma(i+1)} \leq p_{\sigma(i)} + 1$ for all $i=1,\dots,n$.  It follows that
\[
p_{\sigma(n+1)} \leq p_{\sigma(n)} + 1 \leq p_{\sigma(n-1)} + 2 \leq \dots \leq p_{\sigma(1)} + n.
\]
and so $\rng(p) = p_{\sigma(n+1)} - p_{\sigma(1)} \leq n$.
\end{proof}

Finally we define the function
\[
\decrng(p) = p -  \onebf_{\subrng(p)}
\]
that decrements those elements from $p$ with indices from $\subrng(p)$.  If $\subrng(p) = \emptyset$, then $\decrng(p) = p$, that is, $\decrng$ does not modify $p$.  On the other hand, if $\subrng(p) \neq \emptyset$ then
\[
\rng\big(\decrng(p)\big) = \rng(p) - 1
\]
because $\subrng(p)$ contains all those indices $i$ such that $p_i = \max(p)$.  By repeatedly applying $\decrng$ to a vector one eventually obtains a vector for which further application of $\decrng$ has no effect.  For example,
\begin{align*}
\decrng(2,-1,4) &= (2,-1,4) - \onebf_{\subrng(2,-1,4)} = (2,-1,4) - \onebf_{\{1,3\}} = (1,-1,3) \\ 
\decrng(1,-1,3) &= (1,-1,3) - \onebf_{\{1,3\}} = (0,-1,2) \\ 
\decrng(0,-1,2) &= (0,-1,2) - \onebf_{\{3\}} = (0,-1,1) \\ 
\decrng(0,-1,1) &= (0,-1,1) - \onebf_{\emptyset} = (0,-1,1).
\end{align*}
This will be a useful property so we state it formally in the following proposition.

\begin{proposition} \label{lem:repeatappdecrange} Let $p \in \reals^{n+1}$ and define the infinite sequence $d_0,d_1,d_2,\dots$ of vectors according to $d_0=p$ and $d_{k+1} = \decrng(d_k)$.  There is a finite integer $T$ such that $d_T=d_{T+1}=d_{T+2}=\dots$.
\end{proposition}
\begin{proof}
Assume that no such $T$ exists.  Then $\decrng(d_k) \neq d_k$ for all positive integers $k$ and so 
\[
\rng(d_{k}) = \rng(d_{k-1}) - 1 = \rng(d_{k-2}) - 2 = \dots = \rng(p) - k.
\]  
Choosing $k > \rng(p)$ we have $\rng(d_{k}) < 0$ contradicting that $\rng(d_k)$ is nonegative.
\end{proof}

We are now ready to study properties of a closest lattice point in a lattice of Voronoi's first kind.

\begin{lemma}\label{lem:decrngpreservesclosestpoints}
If $v \in \ints^{n+1}$ is such that $B(\floor{z} + v)$ is a closest lattice point to $y = Bz$, then $B\big(\floor{z} + \decrng(v)\big)$ is also a closest lattice point to $y$.
\end{lemma}
\begin{proof}
The lemma is trivial if $\subrng(v) = \emptyset$ so that $\decrng(v) = v$.  It remains to prove the lemma when $\subrng(v) \neq \emptyset$.  In this case put $S = \subrng(v)$ and put 
\[
u = \decrng(v) = v - \onebf_S.
\] 
Let $\zeta = z - \floor{z}$ be the column vector containing the fractional parts of the elements of $z$.  We have $\zeta - u = \zeta - v + \onebf_S$.  Applying part~\ref{eq:lem:decSellingsinc} of Lemma~\ref{lem:decSellings} with $p = \zeta - v$ we obtain
\begin{align}
\|B(\zeta - v)\|^2 - \|B(\zeta - u)\|^2 &= \Phi(S, \zeta-v) \nonumber \\
&= \sum_{i \in S}\sum_{j \notin S}q_{ij}\big(1 + 2(\zeta_i-\zeta_j) - 2(v_i - v_j)\big). \label{eq:sumsumBzBu}
\end{align}
Observe that $\zeta_i =  z_i - \floor{z_i} \in [0,1)$ for all $i=1,\dots,n+1$ and so $-1 < \zeta_i-\zeta_j < 1$ for all $i,j=1,\dots,n+1$.  Also, for $i \in S$ and $j\notin S$ we have 
\[
v_i - v_j \geq \min\{ v_i, i \in S\} - \max\{v_j, j \notin S\} \geq 2
\]
by definition of $\subrng(v) = S$.  Thus,
\[
1 + 2(\zeta_i-\zeta_j) - 2(v_i - v_j) < 1 + 2 - 4 = -1 < 0 \qquad \text{for $i \in S$ and $j \notin S$}.
\]
Substituting this inequality into~\eqref{eq:sumsumBzBu} and using that $q_{ij} \leq 0$ for $i \neq j$ (the obtuse condition~\eqref{eq:obtusecond}) we find that
\[
\|B(z - \floor{z} - v)\|^2 - \|B(z - \floor{z} - u)\|^2 \geq 0.
\]
It follows that $B(\floor{z} + u) = B\big(\floor{z} + \decrng(v)\big)$ is a closest lattice point to $y = Bz$ whenever $B(\floor{z} + v)$ is.
\end{proof}

\begin{lemma}\label{lem:roundzclose}
There exists a closest lattice point to $y = Bz$ in the form $B(\floor{z} + v)$ where $v \in \ints^{n+1}$ with $\rng(v) \leq n$.
\end{lemma}
\begin{proof}
Let $d_0 \in \ints^{n+1}$ be such that $B(\floor{z} + d_0)$ is a closest lattice point to $y$. Define the sequence of vectors $d_0,d_1,\dots$ from $\ints^{n+1}$ according to the recursion $d_{k+1} = \decrng(d_k)$.  It follows from Lemma~\ref{lem:decrngpreservesclosestpoints} that $B(\floor{z} + d_{k})$ is a closest lattice point for all positive integers $k$.  By Proposition~\ref{lem:repeatappdecrange} there is a finite $T$ such that 
\[
d_{T+1}=d_T=\decrng(d_T).
\]  
Thus $\subrng(d_T) = \emptyset$ and $\rng(d_T) \leq n$ by Proposition~\ref{prop:subrrngsmall}.  The proof follows with $v = d_T$.   
\end{proof}

Let $\ell$ be a nonegative integer.  We say that a lattice point $x$ is $\ell$-\emph{close} to $y$ if there exists a $v \in \ints^{n+1}$ with $\rng(v) = \ell$ such that $x + Bv$ is a closest lattice point to $y$.  Lemma~\ref{lem:roundzclose} asserts that the lattice point $x_0 = B\floor{z}$ that initialises the iterative procedure~\eqref{eq:xseqfirsttype} is $K$-close to $y$ where $K \leq n$.  From Lemma~\ref{lem:rngdecreases} stated below it will follow that if the lattice point $x_k$ obtained on the $k$th iteration of the procedure is $\ell$-close, then the lattice point $x_{k+1}$ obtained on the next iteration is $(\ell-1)$-close.  Since $x_0$ is $K$-close it will then follow that after $K \leq n$ iterations the lattice point $x_K$ is $0$-close.  At this stage it is guaranteed that $x_{K}$ itself is a closest lattice point to $y$.  This is shown in the following lemma.  

\begin{lemma}\label{lem:rngzeroclosestpoint}
If $x$ is a lattice point that is $0$-close to $y$, then $x$ is a closest lattice point to $y$.
\end{lemma}
\begin{proof}
Because $x$ is $0$-close there exists a $v \in \ints^{n+1}$ with $\rng(v) = 0$ such that $x + Bv$ is a closest lattice point to $y$.  Because $\rng(v) = 0$ all elements from $v$ are identical, that is, $v_1=v_2=\dots=v_{n+1}$.  In this case $Bv = \sum_{i=1}^{n+1} v_n b_n = v_1\sum_{i=1}^{n+1}b_n = 0$
as a result of the superbasis condition~\eqref{eq:superbasecond}.  Thus $x = x + Bv$ is a closest point to $y$. 
\end{proof}

Before giving the proof of Lemma~\ref{lem:rngdecreases} we require the following simple result.

\begin{lemma}\label{eq:integergreaterless}
Let $h \in \{0,1\}^{n+1}$ and $v \in \ints^{n+1}$ with $\rng(v) \geq 1$.  Suppose that
$h_i = 0$ whenever $v_i = \min(v)$ and that $h_i = 1$ whenever $v_i = \max(v)$.  Then
\[
h_i - h_j \leq v_i - v_j
\]
when either $v_i = \max(v)$ or $v_j = \min(v)$.
\end{lemma}
\begin{proof}
If $v_i = \max(v)$ then $h_i = 1$ and we need only show that $1-h_j \leq \max(v) - v_j$ for all $j$.  If $v_j = \max(v)$ then $h_j = 1$ and the results holds since $1 - h_j = 0 = \min(v) - v_j$.  Otherwise if $v_j < \max(v)$ then $1-h_j \leq 1 \leq \max(v) - v_j$ because $\max(v)$ and $v_j$ are integers.

Now, if $v_j = \min(v)$ then $h_j = 0$ and we need only show that $h_i \leq v_i - \min(v)$ for all $i$.  If $v_i = \min(v)$ then $h_i = 0 = v_i - \min(v)$ and the results holds.  Otherwise if $v_i > \min(v)$ then $h_i \leq 1 \leq v_i - \min(v)$ because $\min(v)$ and $v_i$ are integers.
\end{proof}

\begin{lemma}\label{lem:rngdecreases}
Let $Bu$ with $u \in \ints^{n+1}$ be a lattice point that is $\ell$-close to $y = Bz$ where $\ell > 0$.  Let $g \in \{0,1\}^{n+1}$ be such that
\begin{equation}\label{eq:qismin}
\|B(z - u - g)\|^2 = \min_{t \in \{0,1\}^{n+1}}\|B(z - u - t)\|^2.
\end{equation}
The lattice point $B(u+g)$ is $(\ell-1)$-close to $y$.
\end{lemma}
\begin{proof}
Because $Bu$ is $\ell$-close to $y$ there exists $v \in \ints^{n+1}$ with $\rng(v) = \ell$ such that $B(u+v)$ is a closest lattice point to $y$.  Define subsets of indices
\[
S = \{i \mid g_i = 0, v_i = \max(v) \}, \qquad T = \{i \mid g_i = 1, v_i = \min(v) \}, 
\]
and put $h = g + \onebf_S - \onebf_T \in \{0,1\}^{n+1}$ and $w = v - h$.  Observe that $h_i = 0$ whenever $v_i = \min(v)$ and so $\min(w) = \min(v-h) = \min(v)$.  Also, $h_i = 1$ whenever $v_i = \max(v)$ and so $\max(w) = \max(v-h) = \max(v) - 1$.  Thus,
\[
\rng(w) = \max(v) - 1 - \min(v) = \rng(v) - 1 = \ell - 1.
\]
The lemma will follow if we show that $B(u+g+w)$ is a closest lattice point to $y$ since then the lattice point $B(u+g)$ with be $(\ell-1)$-close to $y$.  The proof is by contradiction.  Suppose $B(u+g+w)$ is not a closest point to $y$, that is, suppose
\[
\|B(z-u-g-w)\|^2 > \|B(z-u-v)\|^2.
\]
Putting $p = z-u-v$ we have
\[
\|B(p+\onebf_S-\onebf_T)\|^2 > \|Bp\|^2.
\]
By part~\ref{eq:lem:decSellingsincdec} of Lemma~\ref{lem:decSellings} we obtain
\begin{equation}\label{eq:BpBpdecincineq}
\|Bp\|^2 - \|B(p+\onebf_S-\onebf_T)\|^2 = \Phi(S,p) + \Phi(\bar{T}, p) + 2\sum_{i\in S}\sum_{j \in T} q_{ij} < 0.
\end{equation}

As stated already $h_i = 0$ whenever $v_i = \min(v)$ and $h_i = 1$ whenever $v_i = \max(v)$.  It follows from Lemma~\ref{eq:integergreaterless} that
\begin{equation}\label{eq:hijvijieq}
h_{i} - h_j \leq v_{i} - v_j
\end{equation}
when either $v_i = \max(v)$ or $v_j = \min(v)$.  Since $v_i = \max(v)$ for $i \in S$ and $v_j = \min(v)$ for $j \in T$ the inequality~\eqref{eq:hijvijieq} holds when either $i \in S$ or $j \in T$.

Put $r = z - u - h$.  By~\eqref{eq:hijvijieq} we have $r_i - r_j \geq p_{i} - p_{j}$ when either $i \in S$ or $j \in T$.  Now, since $q_{ij} \leq 0$ for $i \neq j$,
\[
\Phi(S,r) = \sum_{i \in S}\sum_{j \notin S}q_{ij}(1 + 2r_i - 2r_j) \leq \sum_{i \in S}\sum_{j \notin S}q_{ij}(1 + 2p_i - 2p_j) = \Phi(S,p),
\]
and
\[
\Phi(\bar{T},r) = \sum_{i \notin T}\sum_{j \in T}q_{ij}(1 + 2r_i - 2r_j) \leq \sum_{i \notin T}\sum_{j \in T}q_{ij}(1 + 2p_i - 2p_j) = \Phi(\bar{T},p).
\]
Using part~\ref{eq:lem:decSellingsincdec} of Lemma~\ref{lem:decSellings} again,
\begin{align*}
\|B(z-u-h)\|^2 - \|B(z-u-g)\|^2 &= \|Br\|^2 - \|B(r+\onebf_S - \onebf_T)\|^2 \\
&= \Phi(S,r) + \Phi(\bar{T},r) + 2\sum_{i\in S}\sum_{j \in T} q_{ij} \\
&\leq \Phi(S,p) + \Phi(\bar{T}, p) + 2\sum_{i\in S}\sum_{j \in T} q_{ij} < 0 
\end{align*}
as a result of~\eqref{eq:BpBpdecincineq}.  However, $h \in \{0,1\}^{n+1}$ and so this implies
\[
\|B(z-u-g)\|^2 > \|B(z-u-h)\|^2 \geq \min_{t \in \{0,1\}^{n+1}}\|B(z - u - t)\|^2
\]
contradicting~\eqref{eq:qismin}.  Thus, our original supposition is false and $B(u+g+w)$ is a closest lattice point to $y$. Because $\rng(w) = \ell-1$ the lattice point $B(u+g)$ is $(\ell-1)$-close to $y$.
 \end{proof}

The next theorem asserts that the iterative procedure~\eqref{eq:xseqfirsttype} converges to a closest lattice point in $K \leq n$ iterations.  This is the primary result of this section.

\begin{theorem}
Let $x_0,x_1,\dots$ be the sequence of lattice points given by the iterative procedure~\eqref{eq:xseqfirsttype}.  There exists $K \leq n$ such that $x_K$ is a closest lattice point to $y = Bz$.
\end{theorem}
\begin{proof}
Let $x_k = B u_k$ be the lattice point obtained on the $k$th iteration of the procedure.  Suppose that $x_k$ is $\ell$-close to $y=Bz$ with $\ell > 0$.  The procedure computes $t_{k} \in \{0,1\}^{n+1}$ satisfying
\[
\|B(z - u_k - t_{k})\|^2 = \min_{t \in \{0,1\}^{n+1}}\|B(z - u_k - t)\|^2
\]
and puts $x_{k+1} = B(u_k + t_k)$.  It follows from Lemma~\ref{lem:rngdecreases} that $x_{k+1}$ is $(\ell-1)$-close to $y$.  By Lemma~\ref{lem:roundzclose} the lattice point that initialises the procedure $x_0 = B \floor{z}$ is $K$-close to $y$ where $K \leq n$.  Thus, $x_1$ is $(K-1)$-close, $x_2$ is $(K-2)$-close, and so on until $x_K$ is $0$-close.  That $x_K$ is a closest lattice point to $y$ follows from Lemma~\ref{lem:rngzeroclosestpoint}.
\end{proof}


\section{Computing a closest relevant vector}\label{sec:comp-clos-relev}

In the previous section we showed that the iterative procedure~\eqref{eq:xseqfirsttype} results in a closest lattice point in at most $n$ iterations.  It remains to show that each iteration of the procedure can be computed efficiently.  Specifically, it remains to show that the minimisation over the set of binary vectors $\{0,1\}^{n+1}$ described in~\eqref{eq:pvecmin} can be computed efficiently.  Putting $p = z - u_k$ in~\eqref{eq:pvecmin} we require an efficient method to compute a $t \in \{0,1\}^{n+1}$ such that the binary quadratic form
\[
\| B(p - t) \|^2 = \| \sum_{i=1}^{n+1} b_i (p_i - t_i) \|^2
\]
is minimised.  Expanding this quadratic form gives
\[
\| \sum_{i=1}^{n+1} b_i (p_i - t_i) \|^2 =  \sum_{i=1}^{n+1}\sum_{j=1}^{n+1} q_{ij}p_i p_j -  2\sum_{i=1}^{n+1}\sum_{j=1}^{n+1} q_{ij}p_j t_i + \sum_{i=1}^{n+1}\sum_{j=1}^{n+1} q_{ij} t_i t_j.
\]
The first sum above is independent of $t$ and can be ignored for the purpose of minimisation.  Letting $s_i = -2\sum_{j=1}^{n+1} q_{ij}p_j$, we can equivalently minimise the binary quadratic form
\begin{equation}\label{eq:quadformnp}
Q(t) = \sum_{i=1}^{n+1} s_i t_i + \sum_{i=1}^{n+1}\sum_{j=1}^{n+1} q_{ij} t_i t_j.
\end{equation}
We will show that a minimiser of $Q(t)$ can be found efficiently by computing a minimum cut in an undirected flow network.
%\section{Quadratic $\{0,1\}$ programs and minimium cuts in graphs}
%This section describes how a binary quadratic program can be mapped into the problem of computing a minimum cut in a weighted graph.  
This technique has appeared previously~\cite{Picard_min_cuts_1974,Sankaran_solving_CDMA_mincut_1998,Ulukus_cdma_mincut_1998,Cormen2001} but we include the derivation here so that this paper is self contained.  At the core of this technique is the fact that a one-to-one correspondence exists between the obtuse superbasis of a lattice of Voronoi's first kind and the~\emph{Laplacian matrix}~\cite{Chung_spectral_graph_theory_1997,Cvetković_spectra_graphs_1998} of a simple weighted graph with $n+1$ vertices and positive edge weights equal to the negated Selling parameters $-q_{ij}$.  

Let $G$ be an undirected graph with $n+3$ vertices $v_0, \dots, v_{n+2}$ contained in the set $V$ and edges $e_{ij}$ connecting $v_i$ to $v_j$.  To each edge we assign a \emph{weight} $w_{ij} \in \reals$.  The graph is undirected so the weights are symmetric, that is, $w_{ij} = w_{ji}$.  By calling the vertex $v_0$ the \emph{source} and the vertex $v_{n+2}$ the \emph{sink} the graph $G$ is what is called a \emph{flow network}.  The flow network is \emph{undirected} since the weights assigned to each edge are undirected.  A \emph{cut} in the flow network $G$ is a subset $C \subset V$ of vertices with its complement $\bar{C} \subset V$ such that the source vertex $v_0 \in C$ and the sink vertex $v_{n+2} \in \bar{C}$.  %The vertex $v_0$ is typically called the \emph{source} and the vertex $v_{n+1}$ is typically called the \emph{sink}.

The weight of a cut is
\[
W(C,\bar{C}) = \sum_{i \in I} \sum_{j \in J} w_{ij}, 
\]
where $I = \{ i \mid v_i \in C\}$ and $J = \{j \mid v_j \in \bar{C}\}$.  That is, $W(C,\bar{C})$ is the sum of the weights on the edges crossing from the vertices in $C$ to the vertices in $\bar{C}$.  In what follows we will often drop the argument and write $W$ rather than $W(C,\bar{C})$.  A \emph{minimum cut} is a $C$ and $\bar{C}$ that minimise the weight $W$.  If all of the edge weights $w_{ij}$ for $i \neq j$ are nonnegative, a minimum cut can be computed in 
order $O(n^3)$ arithmetic operations~\cite{Cormen2001,Even_graph_algorithms_1979}.

We require some properties of the weights $w_{ij}$ in relation to $W$.  If the graph is allowed to contain loops, that is, edges from a vertex to itself, then the weight of these edges $w_{ii}$ has no effect on the weight of any cut.  We may choose any values for the $w_{ii}$ without affecting $W$.  We will find it convenient to set $w_{0,0} = w_{n+2,n+2} = 0$.  The remaining $w_{ii}$ we shall specify shortly.  The edge $e_{0,n+2}$ is in every cut.  If a constant is added to the weight of this edge, that is, $w_{0,n+2}$ is replaced by $w_{0,n+2} + c$ then $W$ is replaced by $W + c$ for every $C$ and $\bar{C}$.  In particular, the subsets $C$ and $\bar{C}$ corresponding to a minimum cut are not changed.  We will find it convenient to choose $w_{0,n+2} = w_{n+2,0} = 0$.  

If vertex $v_i$ is in $C$ then edge $e_{i,n+2}$ contributes to the weight of the cut.  If $v_i \notin C$, i.e., $v_i \in \bar{C}$, then edge $e_{0,i}$ contributes to the weight of the cut.  So, either $e_{0,i}$ or $e_{i,n+2}$ \emph{but not both} contribute to every cut.  If a constant, say $c$, is added to the weights of these edges, that is, $w_{0,i}$ and $w_{i,n+2}$ are replaced by $w_{0,i} + c$ and $w_{i,n+2} + c$, then $W$ is replaced by $W + c$ for every $C$ and $\bar{C}$.  The $C$ and $\bar{C}$ corresponding to a minimum cut are unchanged.  In this way, the minimum cut is only affected by the differences 
\[
d_i = w_{i,n+2} - w_{0,i}
\]
for each $i$ and not the specific values of the weights $w_{i,n+2}$ and $w_{0,i}$.  %For the purpose of computing the minimum cut it is necessary to choose the difference so that the weights $w_{0,i}$ and $w_{i,n+1}$ are nonnegative.

% It is convenient to transform $W(C)$ as follows.  Let $D = C \backslash 0$ and $\bar{D} = \bar{C} \backslash (n+1)$ where $C \backslash 0$ denotes the set $C$ with the element $0$ removed and $\bar{C} \backslash (n+1)$ denotes the set $\bar{C}$ with the element $n+1$ removed.  Now we can write the weight as,
% \[
% W(C) =  \sum_{i \in D}w_{i,n+1} + \sum_{j \in \bar{d}}w_{0,j} +  \sum_{i \in D} \sum_{j \in \bar{D}} w_{ij}
% \]
% becuase $w_{0,0},w_{0,n+2}$ and $w_{n+2,n+2}$ are zero. 

We now show how $W(C,\bar{C})$ can be represented as a binary quadratic form.  Put $t_0 = 1$ and $t_{n+2} = 0$ and
\[
t_i = \begin{cases}
1, & i \in C \\
0, & i \in \bar{C}
\end{cases}
\]
for $i = 1,2,\dots,n+1$.  Observe that
\[
t_i(1 - t_j) = \begin{cases}
1, & i \in C, j \in \bar{C} \\
0, & \text{otherwise}.
\end{cases}
\]
The weight can now be written as
\begin{align*}
W(C,\bar{C}) = \sum_{i \in C} \sum_{j \in \bar{C}} w_{ij} = \sum_{i =0}^{n+2} \sum_{j =0}^{n+2} w_{ij} t_i (1 - t_j) = F(t),
\end{align*}
say.  Finding a minimum cut is equivalent to finding the binary vector $t = (t_1, \dots, t_{n+1})$ that minimises $F(t)$.  Write,
\[
F(t) =  \sum_{i=0}^{n+2} \sum_{j =0}^{n+2} w_{ij}t_i - \sum_{i=0}^{n+2} \sum_{j =0}^{n+2} w_{ij} t_it_j.
\]
Letting $k_i = \sum_{j =0}^{n+2} w_{ij}$, and using that $t_0 = 1$ and $t_{n+2} = 0$,
\[
F(t) = \sum_{i=0}^{n+1}k_it_i  - w_{00} - \sum_{i=1}^{n+1} w_{i0} t_i - \sum_{j=1}^{n+1} w_{0j} t_j - \sum_{i=1}^{n+1} \sum_{j =1}^{n+1} w_{ij} t_it_j.
\]
Because $w_{00} = 0$ and $w_{ij} = w_{ji}$ we have
\[
F(t) = k_0 + \sum_{i=1}^{n+1}( k_i  - 2 w_{i0}) t_i  - \sum_{i=1}^{n+1} \sum_{j =1}^{n+1} w_{ij} t_it_j.
\]
The constant term $k_0$ is unimportant for the purpose of minimisation so finding a minimum cut is equivalent to minimising the binary quadratic form
\[
\sum_{i=1}^{n+1}g_i t_i  - \sum_{i=1}^{n+1} \sum_{j =1}^{n+1} w_{ij} t_it_j,
\]
where $g_i = k_i  - 2 w_{i0} = d_i + \sum_{j=1}^{n+1} w_{ij}$.  It only remains to observe the equivalence of this quadratic form and $Q(t)$ from~\eqref{eq:quadformnp} when the weights are assigned to satisfy,
\begin{align*}
&q_{ij} = - w_{ij} \qquad i,j = 1,\dots,n+1 \\
&s_i = g_i = d_i + \sum_{j=1}^{n+1} w_{ij}.
\end{align*}
Because the $q_{ij}$ are nonpositive for $i \neq j$ the weights $w_{ij}$ are nonnegative for all $i \neq j$ with $i,j = 1,\dots,n+1$.  As discussed the value of the weights $w_{ii}$ have no effect on the weight of any cut $W$ so setting $q_{ii} = - w_{ii}$ for  $i = 1,\dots,n+1$ is of no consequence.  Finally the weights $w_{i,n+2}$ and $w_{0,i}$ can be chosen so that both are nonnegative and 
\[
w_{i,n+2} - w_{0,i} = d_i = s_i + \sum_{j=1}^{n+1} q_{ij} = s_i
\]  
because $\sum_{j=1}^{n+1} q_{ij} = 0$ due to the superbase condition~\eqref{eq:superbasecond}.  That is, we choose $w_{i,n+2} = s_i$ and $w_{0,i} = 0$ when $s_i \geq 0$ and $w_{i,n+2}=0$ and $w_{0,i} = -s_i$ when $s_i < 0$.  With these choices, all the weights $w_{ij}$ for $i \neq j$ are nonnegative.  A minimiser of $Q(t)$, and correspondingly a solution of~\eqref{eq:pvecmin} can be computed in $O(n^3)$ operations by computing a minimum cut in the undirected flow network $G$ assigned with these nonnegative weights~\cite{Picard_min_cuts_1974,Sankaran_solving_CDMA_mincut_1998,Ulukus_cdma_mincut_1998,Cormen2001}.  %The closest lattice point is then computed using~\eqref{eq:compnp}.


\section{Discussion}\label{sec:discussion} 

The closest lattice point problem has a number of applications, for example, channel coding and data quantisation~\cite{Conway1983VoronoiCodes,Conway1982VoronoiRegions,Conway1982FastQuantDec,Erex2004_lattice_decoding,Erez2005}.  A significant hurdle in the practical application of lattices as codes or as quantisers is that computing a closest lattice point is computationally difficult in general~\cite{micciancio_hardness_2001}.  The best known general purpose algorithms require a number of operations of order $O(2^{2n})$~\cite{MicciancioVoulgaris_deterministic_jv_2013}.  In this paper we have focused on the class of lattices of Voronoi's first kind.  We have shown that computing a closest point in a lattice of Voronoi's first kind can be achieved in a comparatively modest number of operations of order $O(n^4)$.  Besides being of theoretical interest, the algorithm has potential for practical application.

A question of immediate interest to communications engineers is: do there exist lattices of Voronoi's first kind that produce good codes or good quantisers?  Since lattices that produce good codes and quantisers often also describe dense~\emph{sphere packings}~\cite{SPLAG}, a related question is: do there exist lattices of Voronoi's first kind that produce dense sphere packings?  These questions do not appear to have trivial answers.  The questions have heightened importance due to the algorithm described in this paper.

It is straightforward to construct an `arbitrary' lattice of Voronoi's first kind.  One approach is to construct the $n+1$ by $n+1$ symmetric matrix $Q = B^\prime B$ with elements $Q_{ij} = q_{ij} = \dotprod{b_{i}}{b_j}$ given by the Selling parameters.  Choose the off diagonal entries of $Q$ to be nonpositive with $q_{ij}=q_{ji}$ and set the diagonal elements $q_{ii} = -\sum_{j \neq i} q_{ij}$.  %We then have $Q\onebf = \zerobf$ and so $\onebf$ is in the null space of $Q$.  
The matrix $Q$ is diagonally dominant, that is, $\abs{q_{ii}} \geq \sum_{j \neq i} \abs{q_{ij}}$, and so $Q$ is positive semidefinite.  A rank deficient Cholesky decomposition~\cite{Higham90analysisof} can now be used to recover a matrix $B$ such that $B^\prime B  = Q$. The columns of $B$ are vectors of the obtuse superbasis.

A number applications such as phase unwrapping~\cite{Teunissen_GPS_1995,Hassibi_GPS_1998}, single frequency estimation~\cite{McKilliamFrequencyEstimationByPhaseUnwrapping2009}, and related signal processing problems~\cite{McKilliam2007,Clarkson2007,McKilliam2009IndentifiabliltyAliasingPolyphase,Quinn_sparse_noisy_SSP_2012} also require computing a closest lattice point.  In these applications the particular lattice arises from the signal processing problem under consideration.  If that lattice happens to be of Voronoi's first kind then our algorithm can be used.  An example where this occurs is the problem of computing the \emph{sample intrinsic mean} in circular statistics~\cite{McKilliam_mean_dir_est_sq_arc_length2010}.  In this particular problem the lattice $A_n^*$ is involved.  A fast closest point algorithm requiring only $O(n)$ operations exists for $A_n^*$~\cite{McKilliam2009CoxeterLattices,McKilliam2008b} and so the algorithm described in this paper is not needed in this particular case.  However, there may exist other signal processing problems where lattices of Voronoi's first kind arise.

Another interesting question is: are there subfamilies of Voronoi's first kind that admit even faster algorithms?  Both $A_n$ and $A_n^*$ are examples of this, but there might exist other subfamilies with algorithms faster than $O(n^4)$.  A related question is: can the techniques developed in this paper be applied to other families of lattices, i.e., beyond just those of Voronoi's first kind?  %More generally, do there exist large and useful familes of lattices where computing a closest point has polynomial complexity?   

A final remark is that our algorithm assumes that the obtuse superbasis is known in advance.  It is known that all lattices of dimension less than or equal to 3 are of Voronoi's first kind and an algorithm exists to recover the obtuse superbasis in this case~\cite{SPLAG}.  Lattices of dimension larger than 3 need not be of Voronoi's first kind.  An interesting question is: given a lattice, is it possible to efficiently decide whether it is of Voronoi's first kind?  A related question is: is it possible to efficiently find a obtuse superbasis if it exists?  It is suspected that the answer to both of these questions is no.  An efficient solution to either question would immediately yield a solution to a known problem, that of determining whether a lattice is rectangular (has a basis consisting of pairwise orthogonal vectors) given an arbitrary basis~\cite{Lenstra_Silverberg_revisting_gentra_szydlo_2014}.  It is suspected that this simpler problem is difficult.

%In some applications, for example, the detection of communications signals involving multiple anntanea~\cite{Ryan2008,Wubben_2011} the lattice basis is not known in advance.  It is then of interest whether the 

\section{Conclusion}

The paper describes an algorithm to compute a closest lattice point in a lattice of Voronoi's first kind when the obtuse superbasis is known~\cite{ConwaySloane1992_voronoi_lattice_3d_obtuse_superbases}.  The algorithm requires $O(n^4)$ operations where $n$ is the dimension of the lattice.  The algorithm iteratively computes a series of relevant vectors that converges to a closest lattice point after at most $n$ terms.   Each relevant vector in the series can be efficiently computed in $O(n^3)$ operations by computing a minimum cut in an undirected flow network.  The algorithm has potential application in communications engineering problems such as coding and quantisation.  An interesting problem for future research is to find lattices of Voronoi's first kind that produce good codes, good quantisers, or dense sphere packings~\cite{SPLAG,Conway1982VoronoiRegions}.

%\small
\bibliography{bib}

\end{document}

%%% Local Variables: 
%%% mode: latex
%%% TeX-master: t
%%% End: 
 


\end{document}





















