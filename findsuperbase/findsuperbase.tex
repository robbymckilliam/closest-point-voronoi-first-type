\documentclass[a4paper,10pt]{article}

\usepackage{../mathbf-abbrevs}

%\newcommand {\tbf}[1] {\textbf{#1}}
%\newcommand {\tit}[1] {\textit{#1}}
%\newcommand {\tmd}[1] {\textmd{#1}}
%\newcommand {\trm}[1] {\textrm{#1}}
%\newcommand {\tsc}[1] {\textsc{#1}}
%\newcommand {\tsf}[1] {\textsf{#1}}
%\newcommand {\tsl}[1] {\textsl{#1}}
%\newcommand {\ttt}[1] {\texttt{#1}}
%\newcommand {\tup}[1] {\textup{#1}}
%
%\newcommand {\mbf}[1] {\mathbf{#1}}
%\newcommand {\mmd}[1] {\mathmd{#1}}
%\newcommand {\mrm}[1] {\mathrm{#1}}
%\newcommand {\msc}[1] {\mathsc{#1}}
%\newcommand {\msf}[1] {\mathsf{#1}}
%\newcommand {\msl}[1] {\mathsl{#1}}
%\newcommand {\mtt}[1] {\mathtt{#1}}
%\newcommand {\mup}[1] {\mathup{#1}}

%some math functions and symbols
\newcommand{\reals}{{\mathbb R}}
\newcommand{\ints}{{\mathbb Z}}
\newcommand{\complex}{{\mathbb C}}
\newcommand{\integers}{{\mathbb Z}}
\newcommand{\sign}[1]{\mathtt{sign}\left( #1 \right)}
\newcommand{\NP}{\operatorname{NPt}}
\newcommand{\NS}{\operatorname{NearestSet}}
\newcommand{\bres}{\operatorname{Bres}}
\newcommand{\vol}{\operatorname{vol}}
\newcommand{\vor}{\operatorname{Vor}}
\newcommand{\relevant}{\operatorname{Rel}}

\newcommand{\term}{\emph}
\newcommand{\var}{\operatorname{var}}
\newcommand{\prob}{\operatorname{P}}

%distribution fucntions
\newcommand{\projnorm}{\operatorname{ProjectedNormal}}
\newcommand{\vonmises}{\operatorname{VonMises}}
\newcommand{\wrapnorm}{\operatorname{WrappedNormal}}
\newcommand{\wrapunif}{\operatorname{WrappedUniform}}

\newcommand{\selectindicies}{\operatorname{selectindices}}
\newcommand{\sortindicies}{\operatorname{sortindices}}
\newcommand{\largest}{\operatorname{largest}}
\newcommand{\quickpartition}{\operatorname{quickpartition}}
\newcommand{\quickpartitiontwo}{\operatorname{quickpartition2}}

%some commonly used underlined and
%hated symbols
\newcommand{\uY}{\ushort{\mbf{Y}}}
\newcommand{\ueY}{\ushort{Y}}
\newcommand{\uy}{\ushort{\mbf{y}}}
\newcommand{\uey}{\ushort{y}}
\newcommand{\ux}{\ushort{\mbf{x}}}
\newcommand{\uex}{\ushort{x}}
\newcommand{\uhx}{\ushort{\mbf{\hat{x}}}}
\newcommand{\uehx}{\ushort{\hat{x}}}

% Brackets
\newcommand{\br}[1]{{\left( #1 \right)}}
\newcommand{\sqbr}[1]{{\left[ #1 \right]}}
\newcommand{\cubr}[1]{{\left\{ #1 \right\}}}
\newcommand{\abr}[1]{\left< #1 \right>}
\newcommand{\abs}[1]{{\left| #1 \right|}}
\newcommand{\floor}[1]{{\left\lfloor #1 \right\rfloor}}
\newcommand{\ceiling}[1]{{\left\lceil #1 \right\rceil}}
\newcommand{\ceil}[1]{\lceil #1 \rceil}
\newcommand{\round}[1]{{\left\lfloor #1 \right\rceil}}
\newcommand{\magn}[1]{\left\| #1 \right\|}
\newcommand{\fracpart}[1]{\left< #1 \right>}

% Referencing
\newcommand{\refeqn}[1]{\eqref{#1}}
\newcommand{\reffig}[1]{Figure~\ref{#1}}
\newcommand{\reftable}[1]{Table~\ref{#1}}
\newcommand{\refsec}[1]{Section~\ref{#1}}
\newcommand{\refappendix}[1]{Appendix~\ref{#1}}
\newcommand{\refchapter}[1]{Chapter~\ref{#1}}

\newcommand {\figwidth} {100mm}
\newcommand {\Ref}[1] {Reference~\cite{#1}}
\newcommand {\Sec}[1] {Section~\ref{#1}}
\newcommand {\App}[1] {Appendix~\ref{#1}}
\newcommand {\Chap}[1] {Chapter~\ref{#1}}
\newcommand {\Lem}[1] {Lemma~\ref{#1}}
\newcommand {\Thm}[1] {Theorem~\ref{#1}}
\newcommand {\Cor}[1] {Corollary~\ref{#1}}
\newcommand {\Alg}[1] {Algorithm~\ref{#1}}
\newcommand {\etal} {\emph{~et~al.}}
\newcommand {\bul} {$\bullet$ }   % bullet
\newcommand {\fig}[1] {Figure~\ref{#1}}   % references Figure x
\newcommand {\imp} {$\Rightarrow$}   % implication symbol (default)
\newcommand {\impt} {$\Rightarrow$}   % implication symbol (text mode)
\newcommand {\impm} {\Rightarrow}   % implication symbol (math mode)
\newcommand {\vect}[1] {\mathbf{#1}} 
\newcommand {\hvect}[1] {\hat{\mathbf{#1}}}
\newcommand {\del} {\partial}
\newcommand {\eqn}[1] {Equation~(\ref{#1})} 
\newcommand {\tab}[1] {Table~\ref{#1}} % references Table x
\newcommand {\half} {\frac{1}{2}} 
\newcommand {\ten}[1] {\times10^{#1}}
\newcommand {\bra}[2] {\mbox{}_{#2}\langle #1 |}
\newcommand {\ket}[2] {| #1 \rangle_{#2}}
\newcommand {\Bra}[2] {\mbox{}_{#2}\left.\left\langle #1 \right.\right|}
\newcommand {\Ket}[2] {\left.\left| #1 \right.\right\rangle_{#2}}
\newcommand {\im} {\mathrm{Im}}
\newcommand {\re} {\mathrm{Re}}
\newcommand {\braket}[4] {\mbox{}_{#3}\langle #1 | #2 \rangle_{#4}} 
%\newcommand{\dotprod}[2]{ #1^\prime #2}
\newcommand{\dotprod}[2]{ #1 \cdot #2}
\newcommand {\trace}[1] {\text{tr}\left(#1\right)}

% spell things correctly
\newenvironment{centre}{\begin{center}}{\end{center}}
\newenvironment{itemise}{\begin{itemize}}{\end{itemize}}

%%%%% set up the bibliography style
\bibliographystyle{siam}
%\bibliographystyle{uqthesis}  % uqthesis bibliography style file, made
			      % with makebst

%%%%% optional packages
\usepackage[square,comma,numbers,sort]{natbib}
		% this is the natural sciences bibliography citation
		% style package.  The options here give citations in
		% the text as numbers in square brackets, separated by
		% commas, citations sorted and consecutive citations
		% compressed 
		% output example: [1,4,12-15]

%\usepackage{cite}		
			
\usepackage{units}
	%nice looking units
		
\usepackage{booktabs}
		%creates nice looking tables
		
\usepackage{ifpdf}
\ifpdf
  \usepackage[pdftex]{graphicx}
  %\usepackage{thumbpdf}
  \usepackage[naturalnames]{hyperref}
\else
	\usepackage{graphicx}% standard graphics package for inclusion of
		      % images and eps files into LaTeX document
\fi

\usepackage{amsmath,amsfonts,amssymb} % this is handy for mathematicians and physicists
% see http://www.ams.org/tex/amslatex.html
%\let\proof\relax
%\let\endproof\relax
%\usepackage{amsmath,amsfonts,amssymb, amsthm, bm} % this is handy for mathematicians and physicists
			      % see http://www.ams.org/tex/amslatex.html

		 
%\usepackage[vlined, linesnumbered]{algorithm2e}
	%algorithm writing package
	
\usepackage{mathrsfs}
%fancy math script

%\usepackage{ushort}
%enable good underlining in math mode

%------------------------------------------------------------
% Theorem like environments
%
 \newtheorem{theorem}{Theorem}
% %\theoremstyle{plain}
% \newtheorem{acknowledgement}{Acknowledgement}
% %\newtheorem{algorithm}{Algorithm}
% \newtheorem{axiom}{Axiom}
% \newtheorem{case}{Case}
% \newtheorem{claim}{Claim}
% \newtheorem{conclusion}{Conclusion}
% \newtheorem{observation}{Observation}
% \newtheorem{condition}{Condition}
% \newtheorem{conjecture}{Conjecture}
 \newtheorem{corollary}{Corollary}
% \newtheorem{criterion}{Criterion}
% \newtheorem{definition}{Definition}
% \newtheorem{example}{Example}
% \newtheorem{exercise}{Exercise}
% \newtheorem{lemma}{Lemma}
% \newtheorem{notation}{Notation}
% \newtheorem{problem}{Problem}
% \newtheorem{proposition}{Proposition}
% \newtheorem{remark}{Remark}
% \newtheorem{solution}{Solution}
% \newtheorem{summary}{Summary}
%\numberwithin{equation}{section}
%--------------------------------------------------------


\usepackage{tikz}
\usepackage{pgfplots}
\usetikzlibrary{calc}

%opening
\title{Finding a obtuse superbase}
\author{Robby McKilliam}
\date{\today}

\begin{document}

\maketitle

This document describes an algorithm to determine whether a lattice, specified by a basis matrix, is of Voronoi's first kind.  The algorithm returns an obtuse superbase for the lattice if the lattice is of Vornoi's first kind.

An $n$-dimensional lattice $\Lambda$ is said to be of \emph{Voronoi's first kind} if it has what is called an \emph{obtuse superbasis}~\cite{ConwaySloane1992_voronoi_lattice_3d_obtuse_superbases}.  That is, there exists a set of $n+1$ vectors $b_1,\dots,b_{n+1}$ such that $b_1,\dots,b_n$ are a basis for $\Lambda$,
\begin{equation}\label{eq:superbasecond}
b_1 + b_2 \dots + b_{n+1} = 0
\end{equation}
(the \emph{superbasis} condition), and the inner products satisfy
\begin{equation}\label{eq:obtusecond}
q_{ij} = b_i \cdot b_j \leq 0, \qquad \text{for} \qquad i,j = 1,\dots,n+1, i \neq j
\end{equation}
(the \emph{obtuse} condition).  The $q_{ij}$ are called the \emph{Selling parameters}~\cite{Selling1874}.  It is known that all lattices in dimensions less than $4$ are of Voronoi's first kind~\cite{ConwaySloane1992_voronoi_lattice_3d_obtuse_superbases}.  An interesting property of lattices of Voronoi's first kind is that their relevant vectors have a straightforward description.

\begin{theorem} \label{thm:revvecssuperbase} (Conway and Sloane~\cite[Theorem~3]{ConwaySloane1992_voronoi_lattice_3d_obtuse_superbases})
The relevant vectors of $\Lambda$ are of the form,
\[
\sum_{i \in I} b_i
\]
where $I$ is a strict subset of $\{1, 2, \dots, n+1\}$ that is not empty, i.e., $I \subset \{1, 2, \dots, n+1\}$ and $I \neq \emptyset$.
\end{theorem}  

Notes:
\begin{itemize}
\item Need definition of `strict' and `lax' relevant vectors. 
\item How many lax relevant vectors can there be (presumably this comes the integer lattice, so $3^{n}-1$?)
\item Are vectors from a superbase always strict?  Can probably show this using the arguments in Conway and Sloane's paper.
\end{itemize}

Our algorithm begins by enumerating a superset $R$ of the set of relevant vectors $\relevant(\Lambda)$.  This can be achieved using the algorithm of Micciancio and Voulgaris~\cite{Micciancio09adeterministic} or the algorithm described by Agrell~et.~al.~\cite[Sec.~VI.C]{Agrell2002}.  In many cases these algorithms will return precisely the set of relevant vectors and so $R = \relevant(\Lambda)$, but for what follows it is sufficient if $R \supseteq \relevant(\Lambda)$.

If $\Lambda$ is of first kind then, by Theorem~\ref{thm:revvecssuperbase}, the vectors $b_1,\dots,b_{n+1}$ from the obtuse superbase are contained in $R$.  A straightforward procedure to determine whether the lattice is of first kind consists of enumerating all $\binom{\abs{R}}{n+1}$ combinations of $n+1$ vectors from $R$ and checking whether any combination forms a superbase.  That is, let $v_1,\dots,v_{n+1}$ be a set of $n+1$ distinct vectors from $R$.  If both $v_1+\dots + v_{n+1} = 0$ and $\dotprod{v_{i}}{v_{j}} \leq 0$ for all $i \neq j$ then the lattice is of first kind with superbase is $v_1,\dots,v_{n+1}$.  If no such combination of vectors exists in $R$, then the lattice is not of first kind.

In general there are as many as $2^{n+1}-2$ relevant vectors (BLERG strict, lax etc?)

\begin{figure}
  \begin{tikzpicture}
    \begin{axis}      
      \addplot[mark=*,color=black,only marks,/pgfplots/error bars/.cd,
        x dir=none,
        y dir=both,
        y explicit] table[x=n,y=proportion,y error=stddev] {code/data/testname};
    \end{axis}
  \end{tikzpicture}
\end{figure}





\bibliography{../bib}

\end{document}
