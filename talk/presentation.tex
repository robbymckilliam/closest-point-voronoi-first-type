\documentclass{beamer}

\input{presentationdefs}

\input{../../../bib/mathbf-abbrevs.sty}

\usepackage{ulem}

\usetheme{default}

\setbeamertemplate{footline}[frame number]
\setbeamertemplate{navigation symbols}{}

%\usepackage{beamerthemesplit}

%\usetheme[width=2.8cm, hideothersubsections]{Goettingen}
%\usetheme[width=2.8cm]{Goettingen}

\setbeamercovered{transparent=20}


\title{Finding short vectors in a lattice of Voronoi's first kind}
\author{Robby McKilliam and Alex Grant \\ 
\vspace{0.5cm}
Institute for Telecommunications Research \\ University of South Australia
}

\date{\today}


\begin{document}

\frame{\titlepage}


\frame{\tableofcontents}


\section{Lattices and short vectors}

\frame{
\frametitle{Lattices and short vectors}
An $n$-dimensional \term{lattice} $\Lambda$ is a discrete set of vectors from $\reals^m$, $m \geq n$, given by
\[
  \Lambda = \{ x = b_1 u_1 + b_2u_2 + \dots + b_n u_n \mid u_1, \dots u_n \in \ints \},
\] 
where $b_1, \dots, b_n \in \reals^m$ are \term{basis vectors} of $\Lambda$.
}

\frame{
\begin{figure}[tp]
	\centering 
		\includegraphics[width=\linewidth]{preslatticefigures-1.mps}
		\caption{A 2-dimensional lattice.}
		\label{fig:latfig1}
\end{figure}
}

\frame{
\frametitle{Lattices and short vectors}
Those points in the lattice with smallest non-zero length are called \term{short vectors}.  That is, the short vectors have squared length
\[
\min_{x\in \Lambda \backslash \{ 0 \} } \| x \|^2.
\]
}

\frame{
\begin{figure}[tp]
	\centering 
		\includegraphics[width=\linewidth]{preslatticefigures-1.mps}
		\caption{A 2-dimensional lattice.}
		\label{fig:latfig2}
\end{figure}
}

\frame{
\begin{figure}[tp]
	\centering 
		\includegraphics[width=\linewidth]{preslatticefigures-3.mps}
		\caption{A 2-dimensional lattice.  There are 4 short vectors.}
		\label{fig:latfigandpacking1}
\end{figure}
}

\frame{
\frametitle{Lattices and short vectors}
\begin{itemize}
\item Computing a short vector is called the \term{shortest vector problem}.
\item Applications in cryptography and number theory.
\item NP-hard for arbitrary lattices.
\item Easier for specific lattices.
\item For example, short vectors are easy to find in the \term{root lattices} $\ints^n$, $A_n$ and $D_n$.
\item We will show that the problem is relatively easy to solve for lattices of \term{Voronoi's first kind}.
\end{itemize}
}



\section{Lattices of Voronoi's first kind}
\frame{
\frametitle{Lattices of Voronoi's first kind}
An $n$-dimensional lattice $\Lambda$ is of \term{Voronoi's first kind} if it has an \term{obtuse superbase}, that is, a set of $n+1$ vectors 
\[
b_1,\dots,b_{n+1}
\] 
such that
\begin{itemize}
\item $b_1,\dots,b_n$ are a basis for $\Lambda$,
\item $b_1 + b_2 \dots + b_{n+1} = 0$ \hspace{0.2cm} (the superbase condition),
\item $q_{ij} = b_i \cdot b_j \leq 0$ whenever $i \neq j$ \hspace{0.2cm} (the obtuse condition).
\end{itemize}
\vspace{0.2cm}
The $q_{ij}$ are called \term{Selling parameters}.
%\begin{itemize}
%\item Conway and Sloane have shown that every lattice of dimensional less than 3 is of Vornoi's first kind.
%\end{itemize}
}

\frame{
\frametitle{An example}
Consider the 3-dimensional lattice with basis
\[
\begin{split}
b_1 &= \left[ \begin{array}{rrr} 2 & -1 & 0 \end{array}\right] \\
b_2 &= \left[ \begin{array}{rrr}  -1 & 2 & 0  \end{array}\right] \\
b_3 &= \left[ \begin{array}{rrr}  0 & 0 & 2   \end{array}\right].
\end{split}
\]
Define a $4$th vector as
\[
b_4 = -b_1 - b_2 - b_3 =  \left[ \begin{array}{rrr} -1 &  -1  & -2 \end{array}\right],
\]
so that $b_1, b_2, b_3, b_4$ satisfy the superbase condition.
}

\frame{
\frametitle{An example}
The Selling parameters can be written in a matrix
\[
\left[ \begin{array}{cccc} 
q_{11} & q_{12} & q_{13} & q_{14}\\
q_{21} & q_{22} & q_{23} & q_{24} \\
q_{31} & q_{32} & q_{33} & q_{34} \\
q_{41} & q_{42} & q_{43} & q_{44}
 \end{array}\right]
=
\left[ \begin{array}{rrrr} 
5 & -4 & 0 & -1 \\
-4 & 5 & 0 & -1 \\
0 & 0 & 4 & -4 \\
-1 & -1 & -4 & 6
 \end{array}\right].
\]
The off diagonal elements are not positive so the obtuse condition is satisfied.
%\begin{itemize}
%\item All lattices of dimension less than 4 are of Vornoi's first kind.
%\end{itemize}
} 

% \frame{
% \frametitle{Example: The root lattice $A_n$}
% \begin{itemize}
% \item The root lattice $A_n$ is of Voronoi's first kind.
% \item An obtuse superbase is all cyclic shifts of the vector 
% \[
% \left[ \begin{array}{rrrrrrr} 1 & -1 & 0 & 0 & \cdots & 0 \end{array}\right]
% \] 
% from $\reals^{n+1}$, that is,
% \begin{align*}
% b_1 &= \left[ \begin{array}{rrrrrrr} 1 & -1 & 0 & 0 & \cdots & 0 \end{array}\right] \\
% b_2 &= \left[ \begin{array}{rrrrrrr} 0 & 1 & -1 & 0 & \cdots & 0 \end{array}\right] \\
% &\vdots \\
% b_{n+1} &= \left[ \begin{array}{rrrrrrr} -1 & 0 & 0 & \cdots & 0 & 1 \end{array}\right].
% \end{align*}
% \item The Selling parameters are 
% \[
% q_{ij} = \dotprod{b_i}{b_j} = \begin{cases} 2, & i = j \\
% -1, & i - j \equiv 1 \bmod{n+1} \\
% 0, & \text{otherwise}.
% \end{cases}
% \]
% \end{itemize}
% }



\frame{
\frametitle{Lattices of Voronoi's first kind}
\begin{theorem}
Let $\Lambda$ be a $n$-dimensional lattice of Voronoi's first kind with obtuse superbase 
\[
b_1, \dots, b_{n+1}.
\]  
The short vectors in $\Lambda$ are of the form 
\[
\sum_{i \in I} b_i
\] 
where $I$ is a strict subset of $\{1, 2, \dots, n+1\}$ and $I$ is not empty.
\end{theorem}
}

\frame{
\frametitle{Lattices of Voronoi's first kind}
A na\"{i}ve way to compute a short vector is to compute
\[
\| \sum_{i \in I} b_i \|^2
\]
for all of the $2^{n+1} - 2$ possible subsets $I$.  
\begin{itemize}
\item Requires a number of operations that grows exponentially with the dimension $n$.
\item We can improve this using a \term{minimum cut algorithm}.
\end{itemize}
}

\section{Graphs, cuts, and minimum cuts}
\frame{
\frametitle{Graphs, cuts, and minimum cuts}
Let $G$ be a weighted graph with:
\begin{itemize}
\item $n+1$ vertices $v_1, \dots, v_{n+1}$,  
\item edges $e_{ij}$ connecting vertex $v_i$ to vertex $v_j$,
\item edge weights $w_{ij} \in \reals$.
\end{itemize} 
}

\frame{
\begin{figure}[tp]
	\centering 
        \includegraphics{graphs-3.mps}
		\caption{A graph with 4 vertices and 4 weighted edges.}
		\label{fig:graphexample}
\end{figure}
}

\frame{
\frametitle{Graphs, cuts, and minimum cuts}
A \term{cut} in $G$ is a partition of the vertices into two nonempty sets $C$ and its complement $\bar{C}$.
\begin{itemize}
\item The \term{weight} of a cut is the sum of the weights on the edges crossing from the vertices in $C$ to the vertices in $\bar{C}$.%, that is,
%\[
%W(C,\bar{C}) = \sum_{i \in I} \sum_{j \in J} w_{ij}, 
%\]
%where $I = \{ i \mid v_i \in C\}$ and $J = \{j \mid v_j \in \bar{C}\}$.
%\item \term{Loops} and zero weight edges don't affect the weight of any cut, so we ignore them.
\item A \term{minimum cut} is a pair ($C$,$\bar{C}$) with smallest weight. %that minimise $W(C,\bar{C})$.  
\end{itemize} 
}

\frame{
\begin{figure}[tp]
	\centering 
        \includegraphics{graphs-4.mps}
		\caption{A graph with 4 vertices and 4 weighted edges.}
		\label{fig:graphexample2}
\end{figure}
}

\frame{
\begin{figure}[tp]
	\centering 
        \includegraphics{graphs-5.mps} 
		\caption{The cut $C = \{v_2\}$ and $\bar{C} = \{v_1,v_3,v_4 \}$}
		\label{fig:graphcut1}
\end{figure}
}

\frame{
\begin{figure}[tp]
	\centering 
        \includegraphics{graphs-6.mps}
		\caption{The cut $C = \{v_2\}$ and $\bar{C} = \{v_1,v_3,v_4 \}$ has weight 5.}
		\label{fig:graphcut2}
\end{figure}
} 

\frame{
\begin{figure}[tp]
	\centering 
        \includegraphics{graphs-7.mps}
		\caption{The minimum cut $C = \{v_3,v_4\}$ and $\bar{C} = \{v_1,v_2 \}$}
		\label{fig:graphcut3}
\end{figure}
}

\frame{
\begin{figure}[tp]
	\centering 
        \includegraphics{graphs-8.mps}
		\caption{The minimum cut $C = \{v_3,v_4\}$ and $\bar{C} = \{v_1,v_2 \}$ has weight 2.}
		\label{fig:graphcut4}
\end{figure}
} 

\frame{
\frametitle{Graphs, cuts, and minimum cuts}
If the edge weights $w_{ij}$ are all nonnegative, a minimum cut can be computed:
\begin{itemize}
\item deterministically in $O(n^3)$ operations using the algorithm of Stoer and Wagner,
\item with high probability in $O(n^2 \log(n)^3)$ operations using the randomised algorithm of Karger and Stien.
\end{itemize}	
}
 
     
\frame{
\begin{theorem}
Let $\Lambda$ be a $n$-dimensional lattice of Voronoi's first kind with obtuse superbase 
\[
b_1, \dots, b_{n+1}.
\]  
Let $G$ be a graph with $n+1$ vertices $v_{1}, \dots, v_{n+1}$ and edge weights 
\[
w_{ij} = -q_{ij} = -b_i \cdot b_j \geq 0 \qquad i \neq j.
\]  
Let $(C, \bar{C})$ be a minimum cut in $G$.  A short vector in $\Lambda$ is 
\[
\sum_{i \in I} b_i \;\;\; \text{where} \;\;\; I = \{ i \mid v_i \in C\}.
\]  
The squared length of the short vector is given by the weight of the minimum cut. %$W(C,\bar{C})$.
\end{theorem}
}

\section{Examples}

% \frame{
% \frametitle{Example: The root lattice $A_n$}
% \begin{itemize} 
% \item Recall the root lattice $A_n$ of Voronoi's first kind.
% \item The Selling parameters are
% \[ 
% q_{ij} = \dotprod{b_i}{b_j} = \begin{cases} 2, & i = j \\
% -1, & i - j \equiv 1 \bmod{n+1} \\
% 0, & \text{otherwise}.
% \end{cases}
% \]  
% \item The corresponding graph has positive edge weights
% \[
% w_{ij} = 1 \qquad i - j \equiv 1 \bmod{n+1}.
% \]
% \item This is the cycle graph with $n+1$ vertices and weight 1 on every edge.
% \end{itemize}
% }

% \frame{
% \begin{figure}[tp]
% 	\centering 
%         \includegraphics{cyclegraph-1.mps}
% 		\caption{The cycle graph corresponding to $A_n$.}
% 		\label{fig:graphcycleAn}  
% \end{figure}
% }

% \frame{
% \begin{figure}[tp]
% 	\centering 
%         \includegraphics{cyclegraph-2.mps}
% 		\caption{Cut corresponding with short vector $b_1 = [1,-1,0,\dots,0]$}
% 		\label{fig:graphcycleAn}  
% \end{figure}
% } 

% \frame{
% \begin{figure}[tp]
% 	\centering 
%         \includegraphics{cyclegraph-3.mps} 
% 		\caption{Cut corresponding with $b_1+b_{2}+b_3 = [1,0,0,-1\dots,0]$}
% 		\label{fig:graphcycleAn}  
% \end{figure}
% } 

\frame{
\frametitle{An example}
Consider again the 3-dimensional lattice with obtuse superbase
\[
\begin{split}
b_1 &= \left[ \begin{array}{rrr} 2 & -1 & 0 \end{array}\right] \\
b_2 &= \left[ \begin{array}{rrr}  -1 & 2 & 0  \end{array}\right] \\
b_3 &= \left[ \begin{array}{rrr}  0 & 0 & 2   \end{array}\right] \\
b_4 &= \left[ \begin{array}{rrr} -1 &  -1  & -2 \end{array}\right].
\end{split} 
\]
The Selling parameters are given in matrix form as
\[
\left[ \begin{array}{cccc} 
q_{11} & q_{12} & q_{13} & q_{14}\\
q_{21} & q_{22} & q_{23} & q_{24} \\
q_{31} & q_{32} & q_{33} & q_{34} \\
q_{41} & q_{42} & q_{43} & q_{44}
 \end{array}\right]
=
\left[ \begin{array}{rrrr} 
5 & -4 & 0 & -1 \\
-4 & 5 & 0 & -1 \\
0 & 0 & 4 & -4 \\
-1 & -1 & -4 & 6
 \end{array}\right].
\]
}

\frame{
\begin{figure}[tp]
	\centering 
        \includegraphics{graphs-4.mps}
		\caption{We have seen this graph before!}
		%\label{fig:graphcut4}
\end{figure}
} 


\frame{
\begin{figure}[tp]
	\centering 
        \includegraphics{graphs-8.mps}
		\caption{The minimum cut $C = \{v_3,v_4\}$ and $\bar{C} = \{v_1,v_2 \}$ has weight 2.}
		%\label{fig:graphcut4}
\end{figure}
} 

\frame{
\frametitle{Another example}
The minimum cut corresponds with the short vectors
\[
b_1 + b_2 = [1,1,0]
\]
and 
\[
b_3 + b_4 = -b_1 - b_2 = [-1,-1,0]
\]
of squared length 2.
} 


\section{What's next}
\frame{
\frametitle{What's next}

\begin{itemize}
\item Is it possible to efficiently decide whether a lattice is of Voronoi's first kind?
\item Is it possible to efficiently find an obtuse superbase if it exists?
\item Can a similar approach be taken to solve the \term{nearest lattice point problem}?
\end{itemize}

}

% \frame{
% \frametitle{The nearest point problem}
% \begin{itemize}
% \item Given some vector $y \in \reals^m$ we wish to find a lattice point $x \in \Lambda$ such that the squared Euclidean distance
% \[
% \| y - x \|^2
% \] 
% is minimised.  
% \item Equivalently we can minimise
% \[
% \| y - \sum_{i=1}^{n+1} b_i u_i \|^2
% \]
% over integers $u_1,\dots,u_{n+1}$.
% \end{itemize}
% }

% \frame{
% \begin{theorem}
% Let $z \in \reals^{n+1}$ be a column vector such that $y = \sum_{i=1}^{n+1}b_i z_i$.  A nearest point $x \in \Lambda$ to $y$ is of the form
% \[
% x = \sum_{i=1}^{n+1} b_i (\floor{z_i} + u_i)
% \]
% where the $u_i \in \{-1, 0, 1\}$ and $\floor{z_i}$ denotes the largest integer smaller that $z_i$.
% \end{theorem}

% %So for a lattice of Voronoi's first kind, the nearest point problem can be turned into a quadratic minimisation problem over vectors in $u_i \in \{-1,0,1\}$ rather that $u_i \in \ints$.

% }



\end{document}