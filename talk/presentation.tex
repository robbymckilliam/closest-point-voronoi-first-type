\documentclass{beamer}

%some math functions and symbols
\newcommand{\round}[1]{\left\lceil #1 \right\rfloor}
\newcommand{\floor}[1]{\left\lfloor #1 \right\rfloor}
\newcommand{\ceil}[1]{\left\lceil #1 \right\rceil}
\newcommand{\reals}{{\mathbb R}}
\newcommand{\ints}{{\mathbb Z}}
\newcommand{\complex}{{\mathbb C}}
\newcommand{\integers}{{\mathbb Z}}
\newcommand{\sign}{\mathtt{sign}}
\newcommand{\NP}{\operatorname{NearestPt}}
\newcommand{\NS}{\operatorname{NearestSet}}
\newcommand{\bres}{\operatorname{Bres}}
\newcommand{\vol}{\operatorname{vol}}
\newcommand{\vor}{\operatorname{Vor}}
\newcommand{\relevant}{\operatorname{Rel}}
\newcommand{\coef}{\operatorname{coef}}
\newcommand{\eval}{\operatorname{eval}}
\newcommand{\Int}{\operatorname{Int}}
\newcommand{\pvec}{\operatorname{vec}}
\newcommand{\rem}{\operatorname{rem}}
\newcommand{\var}{\operatorname{var}}
\newcommand{\covar}{\operatorname{covar}}
\newcommand{\erf}{\operatorname{erf}}
\newcommand{\adj}{\operatorname{adj}}
\newcommand{\pad}{\operatorname{pad}}
\newcommand{\dealias}{\operatorname{dealias}}


%distribution fucntions
\newcommand{\projnorm}{\operatorname{ProjectedNormal}}
\newcommand{\vonmises}{\operatorname{VonMises}}
\newcommand{\wrapnorm}{\operatorname{WrappedNormal}}
\newcommand{\wrapunif}{\operatorname{WrappedUniform}}

%sorting and selecting
\newcommand{\selectindicies}{\operatorname{selectindices}}
\newcommand{\sortindicies}{\operatorname{sortindices}}
\newcommand{\largest}{\operatorname{largest}}
\newcommand{\quickpartition}{\operatorname{quickpartition}}
\newcommand{\quickpartitiontwo}{\operatorname{quickpartition2}}

%caligraphic letters, b means bold.
\newcommand{\bcalL}{\bm{\mathcal{L}}}
\newcommand{\bcalX}{\bm{\mathcal{X}}}
\newcommand{\bcalP}{\bm{\mathcal{P}}}
\newcommand{\calP}{\mathcal{P}}
\newcommand{\calR}{\mathcal{R}}
\newcommand{\calV}{\mathcal{V}}

%Brackets
\newcommand{\br}[1]{{\left( #1 \right)}}
\newcommand{\sqbr}[1]{{\left[ #1 \right]}}
\newcommand{\cubr}[1]{{\left\{ #1 \right\}}}
\newcommand{\abr}[1]{\left< #1 \right>}
\newcommand{\abs}[1]{{\left| #1 \right|}}
\newcommand{\ceiling}[1]{{\left\lceil #1 \right\rceil}}
\newcommand{\magn}[1]{\left\| #1 \right\|}
\newcommand{\fracpart}[1]{\left< #1 \right>}
\newcommand{\dotprod}[2]{ #1 \cdot #2}

\definecolor{darkgreen}{rgb}{0,0.5,0}
\newcommand{\term}[1]{{\color{darkgreen}\textbf{#1}}}
\definecolor{red}{rgb}{0.8,0.0,0}
\definecolor{gray}{rgb}{0.7,0.7,0.7}
\newcommand{\redsix}{{\color{red}\textbf{6}}}

%some commonly used underlined and
%hated symbols
\newcommand{\uY}{\ushort{\mbf{Y}}}
\newcommand{\ueY}{\ushort{Y}}
\newcommand{\uy}{\ushort{\mbf{y}}}
\newcommand{\uey}{\ushort{y}}
\newcommand{\ux}{\ushort{\mbf{x}}}
\newcommand{\uex}{\ushort{x}}
\newcommand{\uhx}{\ushort{\mbf{\hat{x}}}}
\newcommand{\uehx}{\ushort{\hat{x}}}

\newcommand {\figwidth} {100mm}
\newcommand {\Ref}[1] {Reference~\cite{#1}}
\newcommand {\Sec}[1] {Section~\ref{#1}}
\newcommand {\App}[1] {Appendix~\ref{#1}}
\newcommand {\Chap}[1] {Chapter~\ref{#1}}
\newcommand {\etal} {\emph{~et~al.}}
\newcommand {\bul} {$\bullet$ }   % bullet
\newcommand {\fig}[1] {Figure~\ref{#1}}   % references Figure x
\newcommand {\imp} {$\Rightarrow$}   % implication symbol (default)
\newcommand {\impt} {$\Rightarrow$}   % implication symbol (text mode)
\newcommand {\impm} {\Rightarrow}   % implication symbol (math mode)
\newcommand {\vect}[1] {\mathbf{#1}} 
\newcommand {\hvect}[1] {\hat{\mathbf{#1}}}
\newcommand {\del} {\partial}
\newcommand {\eqn}[1] {Equation~(\ref{#1})} 
\newcommand {\tab}[1] {Table~\ref{#1}} % references Table x
\newcommand {\half} {\frac{1}{2}} 
\newcommand {\ten}[1] {\times10^{#1}}
\newcommand {\bra}[2] {\mbox{}_{#2}\langle #1 |}
\newcommand {\ket}[2] {| #1 \rangle_{#2}}
\newcommand {\Bra}[2] {\mbox{}_{#2}\left.\left\langle #1 \right.\right|}
\newcommand {\Ket}[2] {\left.\left| #1 \right.\right\rangle_{#2}}
\newcommand {\im} {\mathrm{Im}}
\newcommand {\re} {\mathrm{Re}}
\newcommand {\braket}[4] {\mbox{}_{#3}\langle #1 | #2 \rangle_{#4}} 
%\newcommand {\dotprod}[4] {\mbox{}_{#3}\langle #1 | #2 \rangle_{#4}} 
\newcommand {\trace}[1] {\text{tr}\left(#1\right)}

% spell things correctly
\newenvironment{centre}{\begin{center}}{\end{center}}
\newenvironment{itemise}{\begin{itemize}}{\end{itemize}}


%%%%% set up the bibliography style
%\bibliographystyle{../../bib/IEEEbib}
%\bibliographystyle{uqthesis}  
						% uqthesis bibliography style file, made
			      % with makebst

%%%%% optional packages
\usepackage[square,comma,numbers,sort&compress]{natbib}
		% this is the natural sciences bibliography citation
		% style package.  The options here give citations in
		% the text as numbers in square brackets, separated by
		% commas, citations sorted and consecutive citations
		% compressed 
		% output example: [1,4,12-15]	
%\usepackage[notocbib]{apacite}
		
\usepackage{booktabs}
		%creates nice looking tables
		
%\usepackage[nottoc]{tocbibind}  
				% allows the table of contents, bibliography
				% and index to be added to the table of
				% contents if desired, the option used
				% here specifies that the table of
				% contents is not to be added.
				% tocbibind needs to be after natbib
				% otherwise bits of it get trampled.

\usepackage{amsmath,amsfonts,amssymb, amsthm, bm} % this is handy for mathematicians and physicists
			      % see http://www.ams.org/tex/amslatex.html

%\usepackage[intoc]{nomencl}
%\usepackage{showkeys} % this shows what labels you are using for cross
		      % references
		      
		 
\usepackage[vlined, linesnumbered]{algorithm2e}
	%algorithm writing package
	
\usepackage{mathrsfs}
%fancy math script

\usepackage{ushort, units}
%enable good underlining in math mode

%------------------------------------------------------------
% Theorem like environments
%
%\newtheorem{theorem}{Theorem}
%\theoremstyle{plain}
%\newtheorem{acknowledgement}{Acknowledgement}
%%\newtheorem{algorithm}{Algorithm}
%\newtheorem{axiom}{Axiom}
%\newtheorem{case}{Case}
%\newtheorem{claim}{Claim}
%\newtheorem{conclusion}{Conclusion}
%\newtheorem{condition}{Condition}
%\newtheorem{conjecture}{Conjecture}
%\newtheorem{corollary}{Corollary}
%\newtheorem{criterion}{Criterion}
%\newtheorem{definition}{Definition}
%\newtheorem{example}{Example}
%\newtheorem{exercise}{Exercise}
%\newtheorem{lemma}{Lemma}
%\newtheorem{notation}{Notation}
%\newtheorem{problem}{Problem}
%\newtheorem{proposition}{Proposition}
%\newtheorem{remark}{Remark}
%\newtheorem{solution}{Solution}
%\newtheorem{summary}{Summary}
%\numberwithin{equation}{section}
%--------------------------------------------------------


\newcommand{\abf}{\mathbf{a}}
\newcommand{\bbf}{\mathbf{b}}
\newcommand{\cbf}{\mathbf{c}}
\newcommand{\dbf}{\mathbf{d}}
\newcommand{\ebf}{\mathbf{e}}
\newcommand{\fbf}{\mathbf{f}}
\newcommand{\gbf}{\mathbf{g}}
\newcommand{\hbf}{\mathbf{h}}
\newcommand{\ibf}{\mathbf{i}}
\newcommand{\jbf}{\mathbf{j}}
\newcommand{\kbf}{\mathbf{k}}
\newcommand{\lbf}{\mathbf{l}}
\newcommand{\mbf}{\mathbf{m}}
\newcommand{\nbf}{\mathbf{n}}
\newcommand{\obf}{\mathbf{o}}
\newcommand{\pbf}{\mathbf{p}}
\newcommand{\qbf}{\mathbf{q}}
\newcommand{\rbf}{\mathbf{r}}
\newcommand{\sbf}{\mathbf{s}}
\newcommand{\tbf}{\mathbf{t}}
\newcommand{\ubf}{\mathbf{u}}
\newcommand{\vbf}{\mathbf{v}}
\newcommand{\wbf}{\mathbf{w}}
\newcommand{\xbf}{\mathbf{x}}
\newcommand{\ybf}{\mathbf{y}}
\newcommand{\zbf}{\mathbf{z}}
\newcommand{\Abf}{\mathbf{A}}
\newcommand{\Bbf}{\mathbf{B}}
\newcommand{\Cbf}{\mathbf{C}}
\newcommand{\Dbf}{\mathbf{D}}
\newcommand{\Ebf}{\mathbf{E}}
\newcommand{\Fbf}{\mathbf{F}}
\newcommand{\Gbf}{\mathbf{G}}
\newcommand{\Hbf}{\mathbf{H}}
\newcommand{\Ibf}{\mathbf{I}}
\newcommand{\Jbf}{\mathbf{J}}
\newcommand{\Kbf}{\mathbf{K}}
\newcommand{\Lbf}{\mathbf{L}}
\newcommand{\Mbf}{\mathbf{M}}
\newcommand{\Nbf}{\mathbf{N}}
\newcommand{\Obf}{\mathbf{O}}
\newcommand{\Pbf}{\mathbf{P}}
\newcommand{\Qbf}{\mathbf{Q}}
\newcommand{\Rbf}{\mathbf{R}}
\newcommand{\Sbf}{\mathbf{S}}
\newcommand{\Tbf}{\mathbf{T}}
\newcommand{\Ubf}{\mathbf{U}}
\newcommand{\Vbf}{\mathbf{V}}
\newcommand{\Wbf}{\mathbf{W}}
\newcommand{\Xbf}{\mathbf{X}}
\newcommand{\Ybf}{\mathbf{Y}}
\newcommand{\Zbf}{\mathbf{Z}}
\newcommand{\alphabf}{\boldsymbol{\alpha}}
\newcommand{\betabf}{\boldsymbol{\beta}}
\newcommand{\gammabf}{\boldsymbol{\gamma}}
\newcommand{\deltabf}{\boldsymbol{\delta}}
\newcommand{\epsilonbf}{\boldsymbol{\epsilon}}
\newcommand{\varepsilonbf}{\boldsymbol{\varepsilon}}
\newcommand{\zetabf}{\boldsymbol{\zeta}}
\newcommand{\etabf}{\boldsymbol{\eta}}
\newcommand{\thetabf}{\boldsymbol{\theta}}
\newcommand{\varthetabf}{\boldsymbol{\vartheta}}
\newcommand{\iotabf}{\boldsymbol{\iota}}
\newcommand{\kappabf}{\boldsymbol{\kappa}}
\newcommand{\lambdabf}{\boldsymbol{\lambda}}
\newcommand{\mubf}{\boldsymbol{\mu}}
\newcommand{\nubf}{\boldsymbol{\nu}}
\newcommand{\xibf}{\boldsymbol{\xi}}
\newcommand{\pibf}{\boldsymbol{\pi}}
\newcommand{\varpibf}{\boldsymbol{\varpi}}
\newcommand{\rhobf}{\boldsymbol{\rho}}
\newcommand{\varrhobf}{\boldsymbol{\varrho}}
\newcommand{\sigmabf}{\boldsymbol{\sigma}}
\newcommand{\varsigmabf}{\boldsymbol{\varsigma}}
\newcommand{\taubf}{\boldsymbol{\tau}}
\newcommand{\upsilonbf}{\boldsymbol{\upsilon}}
\newcommand{\phibf}{\boldsymbol{\phi}}
\newcommand{\varphibf}{\boldsymbol{\varphi}}
\newcommand{\chibf}{\boldsymbol{\chi}}
\newcommand{\psibf}{\boldsymbol{\psi}}
\newcommand{\omegabf}{\boldsymbol{\omega}}
\newcommand{\Gammabf}{\boldsymbol{\Gamma}}
\newcommand{\Deltabf}{\boldsymbol{\Delta}}
\newcommand{\Thetabf}{\boldsymbol{\Theta}}
\newcommand{\Lambdabf}{\boldsymbol{\Lambda}}
\newcommand{\Xibf}{\boldsymbol{\Xi}}
\newcommand{\Pibf}{\boldsymbol{\Pi}}
\newcommand{\Sigmabf}{\boldsymbol{\Sigma}}
\newcommand{\Upsilonbf}{\boldsymbol{\Upsilon}}
\newcommand{\Phibf}{\boldsymbol{\Phi}}
\newcommand{\Psibf}{\boldsymbol{\Psi}}
\newcommand{\Omegabf}{\boldsymbol{\Omega}}
\newcommand{\zerobf}{\boldsymbol{0}}
\newcommand{\onebf}{\boldsymbol{1}}

\newcommand{\acal}{\mathcal{a}}
\newcommand{\bcal}{\mathcal{b}}
\newcommand{\ccal}{\mathcal{c}}
\newcommand{\dcal}{\mathcal{d}}
\newcommand{\ecal}{\mathcal{e}}
\newcommand{\fcal}{\mathcal{f}}
\newcommand{\gcal}{\mathcal{g}}
\newcommand{\hcal}{\mathcal{h}}
\newcommand{\ical}{\mathcal{i}}
\newcommand{\jcal}{\mathcal{j}}
\newcommand{\kcal}{\mathcal{k}}
\newcommand{\lcal}{\mathcal{l}}
\newcommand{\mcal}{\mathcal{m}}
\newcommand{\ncal}{\mathcal{n}}
\newcommand{\ocal}{\mathcal{o}}
\newcommand{\pcal}{\mathcal{p}}
\newcommand{\qcal}{\mathcal{q}}
\newcommand{\rcal}{\mathcal{r}}
\newcommand{\scal}{\mathcal{s}}
\newcommand{\tcal}{\mathcal{t}}
\newcommand{\ucal}{\mathcal{u}}
\newcommand{\vcal}{\mathcal{v}}
\newcommand{\wcal}{\mathcal{w}}
\newcommand{\xcal}{\mathcal{x}}
\newcommand{\ycal}{\mathcal{y}}
\newcommand{\zcal}{\mathcal{z}}
\newcommand{\Acal}{\mathcal{A}}
\newcommand{\Bcal}{\mathcal{B}}
\newcommand{\Ccal}{\mathcal{C}}
\newcommand{\Dcal}{\mathcal{D}}
\newcommand{\Ecal}{\mathcal{E}}
\newcommand{\Fcal}{\mathcal{F}}
\newcommand{\Gcal}{\mathcal{G}}
\newcommand{\Hcal}{\mathcal{H}}
\newcommand{\Ical}{\mathcal{I}}
\newcommand{\Jcal}{\mathcal{J}}
\newcommand{\Kcal}{\mathcal{K}}
\newcommand{\Lcal}{\mathcal{L}}
\newcommand{\Mcal}{\mathcal{M}}
\newcommand{\Ncal}{\mathcal{N}}
\newcommand{\Ocal}{\mathcal{O}}
\newcommand{\Pcal}{\mathcal{P}}
\newcommand{\Qcal}{\mathcal{Q}}
\newcommand{\Rcal}{\mathcal{R}}
\newcommand{\Scal}{\mathcal{S}}
\newcommand{\Tcal}{\mathcal{T}}
\newcommand{\Ucal}{\mathcal{U}}
\newcommand{\Vcal}{\mathcal{V}}
\newcommand{\Wcal}{\mathcal{W}}
\newcommand{\Xcal}{\mathcal{X}}
\newcommand{\Ycal}{\mathcal{Y}}
\newcommand{\Zcal}{\mathcal{Z}}

\newcommand{\abb}{\mathbb{a}}
\newcommand{\bbb}{\mathbb{b}}
\newcommand{\cbb}{\mathbb{c}}
\newcommand{\dbb}{\mathbb{d}}
\newcommand{\ebb}{\mathbb{e}}
\newcommand{\fbb}{\mathbb{f}}
\newcommand{\gbb}{\mathbb{g}}
\newcommand{\hbb}{\mathbb{h}}
\newcommand{\ibb}{\mathbb{i}}
\newcommand{\jbb}{\mathbb{j}}
\newcommand{\kbb}{\mathbb{k}}
\newcommand{\lbb}{\mathbb{l}}
\newcommand{\mbb}{\mathbb{m}}
\newcommand{\nbb}{\mathbb{n}}
\newcommand{\obb}{\mathbb{o}}
\newcommand{\pbb}{\mathbb{p}}
\newcommand{\qbb}{\mathbb{q}}
\newcommand{\rbb}{\mathbb{r}}
\newcommand{\sbb}{\mathbb{s}}
\newcommand{\tbb}{\mathbb{t}}
\newcommand{\ubb}{\mathbb{u}}
\newcommand{\vbb}{\mathbb{v}}
\newcommand{\wbb}{\mathbb{w}}
\newcommand{\xbb}{\mathbb{x}}
\newcommand{\ybb}{\mathbb{y}}
\newcommand{\zbb}{\mathbb{z}}
\newcommand{\Abb}{\mathbb{A}}
\newcommand{\Bblb}{\mathbb{B}}
\newcommand{\Cbb}{\mathbb{C}}
\newcommand{\Dbb}{\mathbb{D}}
\newcommand{\Ebb}{\mathbb{E}}
\newcommand{\Fbb}{\mathbb{F}}
\newcommand{\Gbb}{\mathbb{G}}
\newcommand{\Hbb}{\mathbb{H}}
\newcommand{\Ibb}{\mathbb{I}}
\newcommand{\Jbb}{\mathbb{J}}
\newcommand{\Kbb}{\mathbb{K}}
\newcommand{\Lbb}{\mathbb{L}}
\newcommand{\Mbb}{\mathbb{M}}
\newcommand{\Nbb}{\mathbb{N}}
\newcommand{\Obb}{\mathbb{O}}
\newcommand{\Pbb}{\mathbb{P}}
\newcommand{\Qbb}{\mathbb{Q}}
\newcommand{\Rbb}{\mathbb{R}}
\newcommand{\Sbb}{\mathbb{S}}
\newcommand{\Tbb}{\mathbb{T}}
\newcommand{\Ubb}{\mathbb{U}}
\newcommand{\Vbb}{\mathbb{V}}
\newcommand{\Wbb}{\mathbb{W}}
\newcommand{\Xbb}{\mathbb{X}}
\newcommand{\Ybb}{\mathbb{Y}}
\newcommand{\Zbb}{\mathbb{Z}}

\newcommand{\mathhbf}[1]{\hat{\mathbf{#1}}}
\newcommand{\ahbf}{\mathhbf{a}}
\newcommand{\bhbf}{\mathhbf{b}}
\newcommand{\chbf}{\mathhbf{c}}
\newcommand{\dhbf}{\mathhbf{d}}
\newcommand{\ehbf}{\mathhbf{e}}
\newcommand{\fhbf}{\mathhbf{f}}
\newcommand{\ghbf}{\mathhbf{g}}
\newcommand{\hhbf}{\mathhbf{h}}
\newcommand{\ihbf}{\mathhbf{i}}
\newcommand{\jhbf}{\mathhbf{j}}
\newcommand{\khbf}{\mathhbf{k}}
\newcommand{\lhbf}{\mathhbf{l}}
\newcommand{\mhbf}{\mathhbf{m}}
\newcommand{\nhbf}{\mathhbf{n}}
\newcommand{\ohbf}{\mathhbf{o}}
\newcommand{\phbf}{\mathhbf{p}}
\newcommand{\qhbf}{\mathhbf{q}}
\newcommand{\rhbf}{\mathhbf{r}}
\newcommand{\shbf}{\mathhbf{s}}
\newcommand{\thbf}{\mathhbf{t}}
\newcommand{\uhbf}{\mathhbf{u}}
\newcommand{\vhbf}{\mathhbf{v}}
\newcommand{\whbf}{\mathhbf{w}}
\newcommand{\xhbf}{\mathhbf{x}}
\newcommand{\yhbf}{\mathhbf{y}}
\newcommand{\zhbf}{\mathhbf{z}}
\newcommand{\Ahbf}{\mathhbf{A}}
\newcommand{\Bhbf}{\mathhbf{B}}
\newcommand{\Chbf}{\mathhbf{C}}
\newcommand{\Dhbf}{\mathhbf{D}}
\newcommand{\Ehbf}{\mathhbf{E}}
\newcommand{\Fhbf}{\mathhbf{F}}
\newcommand{\Ghbf}{\mathhbf{G}}
\newcommand{\Hhbf}{\mathhbf{H}}
\newcommand{\Ihbf}{\mathhbf{I}}
\newcommand{\Jhbf}{\mathhbf{J}}
\newcommand{\Khbf}{\mathhbf{K}}
\newcommand{\Lhbf}{\mathhbf{L}}
\newcommand{\Mhbf}{\mathhbf{M}}
\newcommand{\Nhbf}{\mathhbf{N}}
\newcommand{\Ohbf}{\mathhbf{O}}
\newcommand{\Phbf}{\mathhbf{P}}
\newcommand{\Qhbf}{\mathhbf{Q}}
\newcommand{\Rhbf}{\mathhbf{R}}
\newcommand{\Shbf}{\mathhbf{S}}
\newcommand{\Thbf}{\mathhbf{T}}
\newcommand{\Uhbf}{\mathhbf{U}}
\newcommand{\Vhbf}{\mathhbf{V}}
\newcommand{\Whbf}{\mathhbf{W}}
\newcommand{\Xhbf}{\mathhbf{X}}
\newcommand{\Yhbf}{\mathhbf{Y}}
\newcommand{\Zhbf}{\mathhbf{Z}} 

\newcommand{\ah}{\hat{a}}
\newcommand{\bh}{\hat{b}}
\newcommand{\ch}{\hat{c}}
\newcommand{\dhat}{\hat{d}}
\newcommand{\eh}{\hat{e}}
\newcommand{\fh}{\hat{f}}
\newcommand{\gh}{\hat{g}}
\newcommand{\hh}{\hat{h}}
\newcommand{\ih}{\hat{i}}
\newcommand{\jh}{\hat{j}}
\newcommand{\kh}{\hat{k}}
\newcommand{\lh}{\hat{l}}
\newcommand{\mh}{\hat{m}}
\newcommand{\nh}{\hat{n}}
\newcommand{\oh}{\hat{o}}
\newcommand{\ph}{\hat{p}}
\newcommand{\qh}{\hat{q}}
\newcommand{\rh}{\hat{r}}
\newcommand{\sh}{\hat{s}}
\newcommand{\that}{\hat{t}}
\newcommand{\uh}{\hat{u}}
\newcommand{\vh}{\hat{v}}
\newcommand{\wh}{\hat{w}}
\newcommand{\xh}{\hat{x}}
\newcommand{\yh}{\hat{y}}
\newcommand{\zh}{\hat{z}}
\newcommand{\Ah}{\hat{A}}
\newcommand{\Bh}{\hat{B}}
\newcommand{\Ch}{\hat{C}}
\newcommand{\Dh}{\hat{D}}
\newcommand{\Eh}{\hat{E}}
\newcommand{\Fh}{\hat{F}}
\newcommand{\Gh}{\hat{G}}
\newcommand{\Hh}{\hat{H}}
\newcommand{\Ih}{\hat{I}}
\newcommand{\Jh}{\hat{J}}
\newcommand{\Kh}{\hat{K}}
\newcommand{\Lh}{\hat{L}}
\newcommand{\Mh}{\hat{M}}
\newcommand{\Nh}{\hat{N}}
\newcommand{\Oh}{\hat{O}}
\newcommand{\Ph}{\hat{P}}
\newcommand{\Qh}{\hat{Q}}
\newcommand{\Rh}{\hat{R}}
\newcommand{\Sh}{\hat{S}}
\newcommand{\Th}{\hat{T}}
\newcommand{\Uh}{\hat{U}}
\newcommand{\Vh}{\hat{V}}
\newcommand{\Wh}{\hat{W}}
\newcommand{\Xh}{\hat{X}}
\newcommand{\Yh}{\hat{Y}}
\newcommand{\Zh}{\hat{Z}} 
\newcommand{\thetah}{\hat{\theta}} 
\newcommand{\phih}{\hat{\phi}} 



\newcommand{\atbf}{\tilde{\mathbf{a}}}
\newcommand{\btbf}{\tilde{\mathbf{b}}}
\newcommand{\ctbf}{\tilde{\mathbf{c}}}
\newcommand{\dtbf}{\tilde{\mathbf{d}}}
\newcommand{\etbf}{\tilde{\mathbf{e}}}
\newcommand{\ftbf}{\tilde{\mathbf{f}}}
\newcommand{\gtbf}{\tilde{\mathbf{g}}}
\newcommand{\htbf}{\tilde{\mathbf{h}}}
\newcommand{\itbf}{\tilde{\mathbf{i}}}
\newcommand{\jtbf}{\tilde{\mathbf{j}}}
\newcommand{\ktbf}{\tilde{\mathbf{k}}}
\newcommand{\ltbf}{\tilde{\mathbf{l}}}
\newcommand{\mtbf}{\tilde{\mathbf{m}}}
\newcommand{\ntbf}{\tilde{\mathbf{n}}}
\newcommand{\otbf}{\tilde{\mathbf{o}}}
\newcommand{\ptbf}{\tilde{\mathbf{p}}}
\newcommand{\qtbf}{\tilde{\mathbf{q}}}
\newcommand{\rtbf}{\tilde{\mathbf{r}}}
\newcommand{\stbf}{\tilde{\mathbf{s}}}
\newcommand{\ttbf}{\tilde{\mathbf{t}}}
\newcommand{\utbf}{\tilde{\mathbf{u}}}
\newcommand{\vtbf}{\tilde{\mathbf{v}}}
\newcommand{\wtbf}{\tilde{\mathbf{w}}}
\newcommand{\xtbf}{\tilde{\mathbf{x}}}
\newcommand{\ytbf}{\tilde{\mathbf{y}}}
\newcommand{\ztbf}{\tilde{\mathbf{z}}}
\newcommand{\Atbf}{\tilde{\mathbf{A}}}
\newcommand{\Btbf}{\tilde{\mathbf{B}}}
\newcommand{\Ctbf}{\tilde{\mathbf{C}}}
\newcommand{\Dtbf}{\tilde{\mathbf{D}}}
\newcommand{\Etbf}{\tilde{\mathbf{E}}}
\newcommand{\Ftbf}{\tilde{\mathbf{F}}}
\newcommand{\Gtbf}{\tilde{\mathbf{G}}}
\newcommand{\Htbf}{\tilde{\mathbf{H}}}
\newcommand{\Itbf}{\tilde{\mathbf{I}}}
\newcommand{\Jtbf}{\tilde{\mathbf{J}}}
\newcommand{\Ktbf}{\tilde{\mathbf{K}}}
\newcommand{\Ltbf}{\tilde{\mathbf{L}}}
\newcommand{\Mtbf}{\tilde{\mathbf{M}}}
\newcommand{\Ntbf}{\tilde{\mathbf{N}}}
\newcommand{\Otbf}{\tilde{\mathbf{O}}}
\newcommand{\Ptbf}{\tilde{\mathbf{P}}}
\newcommand{\Qtbf}{\tilde{\mathbf{Q}}}
\newcommand{\Rtbf}{\tilde{\mathbf{R}}}
\newcommand{\Stbf}{\tilde{\mathbf{S}}}
\newcommand{\Ttbf}{\tilde{\mathbf{T}}}
\newcommand{\Utbf}{\tilde{\mathbf{U}}}
\newcommand{\Vtbf}{\tilde{\mathbf{V}}}
\newcommand{\Wtbf}{\tilde{\mathbf{W}}}
\newcommand{\Xtbf}{\tilde{\mathbf{X}}}
\newcommand{\Ytbf}{\tilde{\mathbf{Y}}}
\newcommand{\Ztbf}{\tilde{\mathbf{Z}}}
\newcommand{\alphatbf}{\tilde{\boldsymbol{\alpha}}}
\newcommand{\betatbf}{\tilde{\boldsymbol{\beta}}}
\newcommand{\gammatbf}{\tilde{\boldsymbol{\gamma}}}
\newcommand{\deltatbf}{\tilde{\boldsymbol{\delta}}}
\newcommand{\epsilontbf}{\tilde{\boldsymbol{\epsilon}}}
\newcommand{\varepsilontbf}{\tilde{\boldsymbol{\varepsilon}}}
\newcommand{\zetatbf}{\tilde{\boldsymbol{\zeta}}}
\newcommand{\etatbf}{\tilde{\boldsymbol{\eta}}}
\newcommand{\thetatbf}{\tilde{\boldsymbol{\theta}}}
\newcommand{\varthetatbf}{\tilde{\boldsymbol{\vartheta}}}
\newcommand{\iotatbf}{\tilde{\boldsymbol{\iota}}}
\newcommand{\kappatbf}{\tilde{\boldsymbol{\kappa}}}
\newcommand{\lambdatbf}{\tilde{\boldsymbol{\lambda}}}
\newcommand{\mutbf}{\tilde{\boldsymbol{\mu}}}
\newcommand{\nutbf}{\tilde{\boldsymbol{\nu}}}
\newcommand{\xitbf}{\tilde{\boldsymbol{\xi}}}
\newcommand{\pitbf}{\tilde{\boldsymbol{\pi}}}
\newcommand{\varpitbf}{\tilde{\boldsymbol{\varpi}}}
\newcommand{\rhotbf}{\tilde{\boldsymbol{\rho}}}
\newcommand{\varrhotbf}{\tilde{\boldsymbol{\varrho}}}
\newcommand{\sigmatbf}{\tilde{\boldsymbol{\sigma}}}
\newcommand{\varsigmatbf}{\tilde{\boldsymbol{\varsigma}}}
\newcommand{\tautbf}{\tilde{\boldsymbol{\tau}}}
\newcommand{\upsilontbf}{\tilde{\boldsymbol{\upsilon}}}
\newcommand{\phitbf}{\tilde{\boldsymbol{\phi}}}
\newcommand{\varphitbf}{\tilde{\boldsymbol{\varphi}}}
\newcommand{\chitbf}{\tilde{\boldsymbol{\chi}}}
\newcommand{\psitbf}{\tilde{\boldsymbol{\psi}}}
\newcommand{\omegatbf}{\tilde{\boldsymbol{\omega}}}
\newcommand{\Gammatbf}{\tilde{\boldsymbol{\Gamma}}}
\newcommand{\Deltatbf}{\tilde{\boldsymbol{\Delta}}}
\newcommand{\Thetatbf}{\tilde{\boldsymbol{\Theta}}}
\newcommand{\Lambdatbf}{\tilde{\boldsymbol{\Lambda}}}
\newcommand{\Xitbf}{\tilde{\boldsymbol{\Xi}}}
\newcommand{\Pitbf}{\tilde{\boldsymbol{\Pi}}}
\newcommand{\Sigmatbf}{\tilde{\boldsymbol{\Sigma}}}
\newcommand{\Upsilontbf}{\tilde{\boldsymbol{\Upsilon}}}
\newcommand{\Phitbf}{\tilde{\boldsymbol{\Phi}}}
\newcommand{\Psitbf}{\tilde{\boldsymbol{\Psi}}}
\newcommand{\Omegatbf}{\tilde{\boldsymbol{\Omega}}}
\newcommand{\zerotbf}{\tilde{\boldsymbol{0}}}
\newcommand{\onetbf}{\tilde{\boldsymbol{1}}}


\newcommand{\at}{\tilde{a}}
\newcommand{\bt}{\tilde{b}}
\newcommand{\ct}{\tilde{c}}
\newcommand{\dt}{\tilde{d}}
\newcommand{\et}{\tilde{e}}
\newcommand{\ft}{\tilde{f}}
\newcommand{\gt}{\tilde{g}}
\newcommand{\hti}{\tilde{h}}
\newcommand{\iti}{\tilde{i}}
\newcommand{\jt}{\tilde{j}}
\newcommand{\kt}{\tilde{k}}
\newcommand{\lt}{\tilde{l}}
\newcommand{\mt}{\tilde{m}}
\newcommand{\nt}{\tilde{n}}
\newcommand{\ot}{\tilde{o}}
\newcommand{\pt}{\tilde{p}}
\newcommand{\qt}{\tilde{q}}
\newcommand{\rt}{\tilde{r}}
\newcommand{\st}{\tilde{s}}
\newcommand{\tti}{\tilde{t}}
\newcommand{\ut}{\tilde{u}}
\newcommand{\vt}{\tilde{v}}
\newcommand{\wt}{\tilde{w}}
\newcommand{\xt}{\tilde{x}}
\newcommand{\yt}{\tilde{y}}
\newcommand{\zt}{\tilde{z}}
\newcommand{\At}{\tilde{A}}
\newcommand{\Bt}{\tilde{B}}
\newcommand{\Ct}{\tilde{C}}
\newcommand{\Dt}{\tilde{D}}
\newcommand{\Et}{\tilde{E}}
\newcommand{\Ft}{\tilde{F}}
\newcommand{\Gt}{\tilde{G}}
\newcommand{\Ht}{\tilde{H}}
\newcommand{\It}{\tilde{I}}
\newcommand{\Jt}{\tilde{J}}
\newcommand{\Kt}{\tilde{K}}
\newcommand{\Lt}{\tilde{L}}
\newcommand{\Mt}{\tilde{M}}
\newcommand{\Nt}{\tilde{N}}
\newcommand{\Ot}{\tilde{O}}
\newcommand{\Pt}{\tilde{P}}
\newcommand{\Qt}{\tilde{Q}}
\newcommand{\Rt}{\tilde{R}}
\newcommand{\St}{\tilde{S}}
\newcommand{\Tt}{\tilde{T}}
\newcommand{\Ut}{\tilde{U}}
\newcommand{\Vt}{\tilde{V}}
\newcommand{\Wt}{\tilde{W}}
\newcommand{\Xt}{\tilde{X}}
\newcommand{\Yt}{\tilde{Y}}
\newcommand{\Zt}{\tilde{Z}} 
\newcommand{\phit}{\tilde{\phi}} 

\newcommand{\aubbf}{\mathbf{\underline{a}}}
\newcommand{\bubbf}{\mathbf{\underline{b}}}
\newcommand{\cubbf}{\mathbf{\underline{c}}}
\newcommand{\dubbf}{\mathbf{\underline{d}}}
\newcommand{\eubbf}{\mathbf{\underline{e}}}
\newcommand{\fubbf}{\mathbf{\underline{f}}}
\newcommand{\gubbf}{\mathbf{\underline{g}}}
\newcommand{\hubbf}{\mathbf{\underline{h}}}
\newcommand{\iubbf}{\mathbf{\underline{i}}}
\newcommand{\jubbf}{\mathbf{\underline{j}}}
\newcommand{\kubbf}{\mathbf{\underline{k}}}
\newcommand{\lubbf}{\mathbf{\underline{l}}}
\newcommand{\mubbf}{\mathbf{\underline{m}}}
\newcommand{\nubbf}{\mathbf{\underline{n}}}
\newcommand{\oubbf}{\mathbf{\underline{o}}}
\newcommand{\pubbf}{\mathbf{\underline{p}}}
\newcommand{\qubbf}{\mathbf{\underline{q}}}
\newcommand{\rubbf}{\mathbf{\underline{r}}}
\newcommand{\subbf}{\mathbf{\underline{s}}}
\newcommand{\tubbf}{\mathbf{\underline{t}}}
\newcommand{\uubbf}{\mathbf{\underline{u}}}
\newcommand{\vubbf}{\mathbf{\underline{v}}}
\newcommand{\wubbf}{\mathbf{\underline{w}}}
\newcommand{\xubbf}{\mathbf{\underline{x}}}
\newcommand{\yubbf}{\mathbf{\underline{y}}}
\newcommand{\zubbf}{\mathbf{\underline{z}}}
\newcommand{\Aubbf}{\mathbf{\underline{A}}}
\newcommand{\Bubbf}{\mathbf{\underline{B}}}
\newcommand{\Cubbf}{\mathbf{\underline{C}}}
\newcommand{\Dubbf}{\mathbf{\underline{D}}}
\newcommand{\Eubbf}{\mathbf{\underline{E}}}
\newcommand{\Fubbf}{\mathbf{\underline{F}}}
\newcommand{\Gubbf}{\mathbf{\underline{G}}}
\newcommand{\Hubbf}{\mathbf{\underline{H}}}
\newcommand{\Iubbf}{\mathbf{\underline{I}}}
\newcommand{\Jubbf}{\mathbf{\underline{J}}}
\newcommand{\Kubbf}{\mathbf{\underline{K}}}
\newcommand{\Lubbf}{\mathbf{\underline{L}}}
\newcommand{\Mubbf}{\mathbf{\underline{M}}}
\newcommand{\Nubbf}{\mathbf{\underline{N}}}
\newcommand{\Oubbf}{\mathbf{\underline{O}}}
\newcommand{\Pubbf}{\mathbf{\underline{P}}}
\newcommand{\Qubbf}{\mathbf{\underline{Q}}}
\newcommand{\Rubbf}{\mathbf{\underline{R}}}
\newcommand{\Subbf}{\mathbf{\underline{S}}}
\newcommand{\Tubbf}{\mathbf{\underline{T}}}
\newcommand{\Uubbf}{\mathbf{\underline{U}}}
\newcommand{\Vubbf}{\mathbf{\underline{V}}}
\newcommand{\Wubbf}{\mathbf{\underline{W}}}
\newcommand{\Xubbf}{\mathbf{\underline{X}}}
\newcommand{\Yubbf}{\mathbf{\underline{Y}}}
\newcommand{\Zubbf}{\mathbf{\underline{Z}}}
\newcommand{\alphaubbf}{\underline{\boldsymbol{\alpha}}}
\newcommand{\betaubbf}{\underline{\boldsymbol{\beta}}}
\newcommand{\gammaubbf}{\underline{\boldsymbol{\gamma}}}
\newcommand{\deltaubbf}{\underline{\boldsymbol{\delta}}}
\newcommand{\epsilonubbf}{\underline{\boldsymbol{\epsilon}}}
\newcommand{\varepsilonubbf}{\underline{\boldsymbol{\varepsilon}}}
\newcommand{\zetaubbf}{\underline{\boldsymbol{\zeta}}}
\newcommand{\etaubbf}{\underline{\boldsymbol{\eta}}}
\newcommand{\thetaubbf}{\underline{\boldsymbol{\theta}}}
\newcommand{\varthetaubbf}{\underline{\boldsymbol{\vartheta}}}
\newcommand{\iotaubbf}{\underline{\boldsymbol{\iota}}}
\newcommand{\kappaubbf}{\underline{\boldsymbol{\kappa}}}
\newcommand{\lambdaubbf}{\underline{\boldsymbol{\lambda}}}
\newcommand{\muubbf}{\underline{\boldsymbol{\mu}}}
\newcommand{\nuubbf}{\underline{\boldsymbol{\nu}}}
\newcommand{\xiubbf}{\underline{\boldsymbol{\xi}}}
\newcommand{\piubbf}{\underline{\boldsymbol{\pi}}}
\newcommand{\varpiubbf}{\underline{\boldsymbol{\varpi}}}
\newcommand{\rhoubbf}{\underline{\boldsymbol{\rho}}}
\newcommand{\varrhoubbf}{\underline{\boldsymbol{\varrho}}}
\newcommand{\sigmaubbf}{\underline{\boldsymbol{\sigma}}}
\newcommand{\varsigmaubbf}{\underline{\boldsymbol{\varsigma}}}
\newcommand{\tauubbf}{\underline{\boldsymbol{\tau}}}
\newcommand{\upsilonubbf}{\underline{\boldsymbol{\upsilon}}}
\newcommand{\phiubbf}{\underline{\boldsymbol{\phi}}}
\newcommand{\varphiubbf}{\underline{\boldsymbol{\varphi}}}
\newcommand{\chiubbf}{\underline{\boldsymbol{\chi}}}
\newcommand{\psiubbf}{\underline{\boldsymbol{\psi}}}
\newcommand{\omegaubbf}{\underline{\boldsymbol{\omega}}}
\newcommand{\Gammaubbf}{\underline{\boldsymbol{\Gamma}}}
\newcommand{\Deltaubbf}{\underline{\boldsymbol{\Delta}}}
\newcommand{\Thetaubbf}{\underline{\boldsymbol{\Theta}}}
\newcommand{\Lambdaubbf}{\underline{\boldsymbol{\Lambda}}}
\newcommand{\Xiubbf}{\underline{\boldsymbol{\Xi}}}
\newcommand{\Piubbf}{\underline{\boldsymbol{\Pi}}}
\newcommand{\Sigmaubbf}{\underline{\boldsymbol{\Sigma}}}
\newcommand{\Upsilonubbf}{\underline{\boldsymbol{\Upsilon}}}
\newcommand{\Phiubbf}{\underline{\boldsymbol{\Phi}}}
\newcommand{\Psiubbf}{\underline{\boldsymbol{\Psi}}}
\newcommand{\Omegaubbf}{\underline{\boldsymbol{\Omega}}}
\newcommand{\zeroubbf}{\underline{\boldsymbol{0}}}
\newcommand{\oneubbf}{\underline{\boldsymbol{1}}}


\newcommand{\aub}{\underline{a}}
\newcommand{\bub}{\underline{b}}
\newcommand{\cub}{\underline{c}}
\newcommand{\dub}{\underline{d}}
\newcommand{\eub}{\underline{e}}
\newcommand{\fub}{\underline{f}}
\newcommand{\gub}{\underline{g}}
\newcommand{\hub}{\underline{h}}
\newcommand{\iub}{\underline{i}}
\newcommand{\jub}{\underline{j}}
\newcommand{\kub}{\underline{k}}
\newcommand{\lub}{\underline{l}}
\newcommand{\mub}{\underline{m}}
\newcommand{\nub}{\underline{n}}
\newcommand{\oub}{\underline{o}}
\newcommand{\pub}{\underline{p}}
\newcommand{\qub}{\underline{q}}
\newcommand{\rub}{\underline{r}}
\newcommand{\sub}{\underline{s}}
\newcommand{\tub}{\underline{t}}
\newcommand{\uub}{\underline{u}}
\newcommand{\vub}{\underline{v}}
\newcommand{\wub}{\underline{w}}
\newcommand{\xub}{\underline{x}}
\newcommand{\yub}{\underline{y}}
\newcommand{\zub}{\underline{z}}
\newcommand{\Aub}{\underline{A}}
\newcommand{\Bub}{\underline{B}}
\newcommand{\Cub}{\underline{C}}
\newcommand{\Dub}{\underline{D}}
\newcommand{\Eub}{\underline{E}}
\newcommand{\Fub}{\underline{F}}
\newcommand{\Gub}{\underline{G}}
\newcommand{\Hub}{\underline{H}}
\newcommand{\Iub}{\underline{I}}
\newcommand{\Jub}{\underline{J}}
\newcommand{\Kub}{\underline{K}}
\newcommand{\Lub}{\underline{L}}
\newcommand{\Mub}{\underline{M}}
\newcommand{\Nub}{\underline{N}}
\newcommand{\Oub}{\underline{O}}
\newcommand{\Pub}{\underline{P}}
\newcommand{\Qub}{\underline{Q}}
\newcommand{\Rub}{\underline{R}}
\newcommand{\Sub}{\underline{S}}
\newcommand{\Tub}{\underline{T}}
\newcommand{\Uub}{\underline{U}}
\newcommand{\Vub}{\underline{V}}
\newcommand{\Wub}{\underline{W}}
\newcommand{\Xub}{\underline{X}}
\newcommand{\Yub}{\underline{Y}}
\newcommand{\Zub}{\underline{Z}}
\newcommand{\alphaub}{\underline{\alpha}}
\newcommand{\betaub}{\underline{\beta}}
\newcommand{\gammaub}{\underline{\gamma}}
\newcommand{\deltaub}{\underline{\delta}}
\newcommand{\epsilonub}{\underline{\epsilon}}
\newcommand{\varepsilonub}{\underline{\varepsilon}}
\newcommand{\zetaub}{\underline{\zeta}}
\newcommand{\etaub}{\underline{\eta}}
\newcommand{\thetaub}{\underline{\theta}}
\newcommand{\varthetaub}{\underline{\vartheta}}
\newcommand{\iotaub}{\underline{\iota}}
\newcommand{\kappaub}{\underline{\kappa}}
\newcommand{\lambdaub}{\underline{\lambda}}
\newcommand{\muub}{\underline{\mu}}
\newcommand{\nuub}{\underline{\nu}}
\newcommand{\xiub}{\underline{\xi}}
\newcommand{\piub}{\underline{\pi}}
\newcommand{\varpiub}{\underline{\varpi}}
\newcommand{\rhoub}{\underline{\rho}}
\newcommand{\varrhoub}{\underline{\varrho}}
\newcommand{\sigmaub}{\underline{\sigma}}
\newcommand{\varsigmaub}{\underline{\varsigma}}
\newcommand{\tauub}{\underline{\tau}}
\newcommand{\upsilonub}{\underline{\upsilon}}
\newcommand{\phiub}{\underline{\phi}}
\newcommand{\varphiub}{\underline{\varphi}}
\newcommand{\chiub}{\underline{\chi}}
\newcommand{\psiub}{\underline{\psi}}
\newcommand{\omegaub}{\underline{\omega}}
\newcommand{\Gammaub}{\underline{\Gamma}}
\newcommand{\Deltaub}{\underline{\Delta}}
\newcommand{\Thetaub}{\underline{\Theta}}
\newcommand{\Lambdaub}{\underline{\Lambda}}
\newcommand{\Xiub}{\underline{\Xi}}
\newcommand{\Piub}{\underline{\Pi}}
\newcommand{\Sigmaub}{\underline{\Sigma}}
\newcommand{\Upsilonub}{\underline{\Upsilon}}
\newcommand{\Phiub}{\underline{\Phi}}
\newcommand{\Psiub}{\underline{\Psi}}
\newcommand{\Omegaub}{\underline{\Omega}}
\newcommand{\zeroub}{\underline{0}}
\newcommand{\oneub}{\underline{1}}

\newcommand{\xbfML}{\xbf_{\text{ML}}}
\newcommand{\hbfML}{\hbf_{\text{ML}}}
\newcommand{\XbfML}{\Xbf_{\text{ML}}}
\newcommand{\HbfML}{\Hbf_{\text{ML}}} 

\usepackage{ulem}

\usetheme{default}

\setbeamertemplate{footline}[frame number]
\setbeamertemplate{navigation symbols}{}

%\usepackage{beamerthemesplit}

%\usetheme[width=2.8cm, hideothersubsections]{Goettingen}
%\usetheme[width=2.8cm]{Goettingen}

\setbeamercovered{transparent=20}


\title{Finding shortest and closest vectors in a lattice of Voronoi's first kind}
\author{Robby McKilliam, Alex Grant, and I. Vaughan L. Clarkson \\ 
\vspace{0.5cm}
{\small
Institute for Telecommunications Research \\ University of South Australia 
\\ \vspace{0.2cm}
School of Information Technology and Electrical Engineering \\ The University of Queensland
}
}

\date{\today}


\begin{document}

\frame{\titlepage}


\frame{\tableofcontents}


\section{Lattices}

\frame{
\frametitle{Lattices}
An $n$-dimensional \term{lattice} $\Lambda$ is a discrete set of vectors from $\reals^m$, $m \geq n$, given by
\[
  \Lambda = \{ b_1 u_1 + b_2u_2 + \dots + b_n u_n \mid u_1, \dots u_n \in \ints \},
\] 
where $b_1, \dots, b_n \in \reals^m$ are \term{basis vectors} of $\Lambda$.
}

\frame{
\begin{figure}[tp]
	\centering 
		\includegraphics{figs/lattice-1.mps}
		\caption{A 2-dimensional lattice.}
		\label{fig:latfig1}
\end{figure}
}

\frame{
\frametitle{Short vectors}
Those lattice points with smallest non-zero length are called \term{short vectors}.  That is, the short vectors have squared length
\[
\min_{x\in \Lambda \backslash \{ 0 \} } \| x \|^2.
\]
}

\frame{
\begin{figure}[tp]
	\centering 
		\includegraphics{figs/lattice-1.mps}
		\caption{A 2-dimensional lattice.}
		\label{fig:latfig2}
\end{figure}
}

\frame{
\begin{figure}[tp]
	\centering 
		\includegraphics{figs/latticewithspherepacking-1.mps}
		\caption{A 2-dimensional lattice.  There are 4 short vectors.}
		\label{fig:latfigandpacking1}
\end{figure}
}

\frame{
\frametitle{The shortest vector problem}
\begin{itemize}
\item Computing a short vector is called the \term{shortest vector problem}.
\item Applications in cryptography and number theory.
\item NP-hard for arbitrary lattices.
\item Easier for specific lattices.
\item For example, short vectors are easy to find in the \term{root lattices} $\ints^n$, $A_n$ and $D_n$.
\item We will show that the problem is relatively easy to solve for lattices of \term{Voronoi's first kind}.
\end{itemize}
}
 
\frame{ 
\frametitle{The Voronoi cell}

The \term{Voronoi cell} of a lattice $\Lambda \subset \reals^m$ is the subset of $\reals^m$ with points at least as close to the origin than to any lattice point,
\[
\vor(\Lambda) = \{ x \in \reals^m \mid \|x\| \leq \|x - y\|, y \in \Lambda \}.
\]

}  

\frame{ 
\begin{figure}[tp]
	\centering 
		\includegraphics{figs/voronoicell-1.mps}
		\caption{A 2-dimensional lattice and its Voronoi cell.}
		\label{fig:latfigandpacking1} 
\end{figure}
}

\frame{ 
\frametitle{Relevant vectors}

The \term{relevant vectors} of a lattice $\Lambda$ are those which contribute a face to the Voronoi cell.  

}

\frame{
\begin{figure}[tp] 
	\centering 
		\includegraphics{figs/relevantvectors-1.mps}  
		\caption{A 2-dimensional lattice with 6 relevant vectors.}
		\label{fig:latfigandpacking1}
\end{figure}
}

\frame{ 
\frametitle{Relevant vectors}
\begin{itemize}
\item The set of relevant vectors $\relevant(\Lambda)$ contains those $v \in \Lambda$ such that  
\[
\dotprod{v}{x} \leq \dotprod{x}{x}
\]
for all $x \in \Lambda$ with $x \neq v$ and $x \neq \zerobf$.  
\item The Voronoi cell can be defined using the relevant vectors, 
\[
\vor(\Lambda) = \{ x \in\reals^m \mid \|x\| \leq \|x - v\|, v \in\relevant(\Lambda) \}
\]
\item Short vectors are relevant vectors.
\end{itemize}
}

\frame{
\begin{figure}[tp] 
	\centering 
		\includegraphics{figs/relevantvectorsandspherepacking-1.mps}   
		\caption{A lattice with 6 relevant vectors and 4 short vectors.}
		\label{fig:latfigandpacking1}
\end{figure}
}

\frame{
\frametitle{Closest lattice points}

Given a lattice $\Lambda \subset \reals^m$ and a vector $y \in \reals^m$, a problem of interest is to find $x \in \Lambda$ such that
\[
\| y - x \|^2
\] 
is minimised.  

\begin{itemize}
\item This is called the \term{closest lattice point problem} and a solution is called a \term{closest lattice point} to $y$
\end{itemize}

}



\section{Lattices of Voronoi's first kind}
\frame{
\frametitle{Lattices of Voronoi's first kind}
An $n$-dimensional lattice $\Lambda$ is of \term{Voronoi's first kind} if it has an \term{obtuse superbase}, that is, a set of $n+1$ vectors 
\[
b_1,\dots,b_{n+1}
\] 
such that
\begin{itemize}
\item $b_1,\dots,b_n$ are a basis for $\Lambda$,
\item $b_1 + b_2 \dots + b_{n+1} = 0$ \hspace{0.2cm} (the superbase condition),
\item $q_{ij} = b_i \cdot b_j \leq 0$ whenever $i \neq j$ \hspace{0.2cm} (the obtuse condition).
\end{itemize}
\vspace{0.2cm}
The $q_{ij}$ are called \term{Selling parameters}.
%\begin{itemize}
%\item Conway and Sloane have shown that every lattice of dimensional less than 3 is of Vornoi's first kind.
%\end{itemize}
}

\frame{
\frametitle{An example}
Consider the 3-dimensional lattice with basis
\[
\begin{split}
b_1 &= \left[ \begin{array}{rrr} 2 & -1 & 0 \end{array}\right] \\
b_2 &= \left[ \begin{array}{rrr}  -1 & 2 & 0  \end{array}\right] \\
b_3 &= \left[ \begin{array}{rrr}  0 & 0 & 2   \end{array}\right].
\end{split}
\]
Define a $4$th vector as
\[
b_4 = -b_1 - b_2 - b_3 =  \left[ \begin{array}{rrr} -1 &  -1  & -2 \end{array}\right],
\]
so that $b_1, b_2, b_3, b_4$ satisfy the superbase condition.
}

\frame{
\frametitle{An example}
The Selling parameters can be written in a matrix
\[
\left[ \begin{array}{cccc} 
q_{11} & q_{12} & q_{13} & q_{14}\\
q_{21} & q_{22} & q_{23} & q_{24} \\
q_{31} & q_{32} & q_{33} & q_{34} \\
q_{41} & q_{42} & q_{43} & q_{44}
 \end{array}\right]
=
\left[ \begin{array}{rrrr} 
5 & -4 & 0 & -1 \\
-4 & 5 & 0 & -1 \\
0 & 0 & 4 & -4 \\
-1 & -1 & -4 & 6
 \end{array}\right].
\]
The off diagonal elements are not positive so the obtuse condition is satisfied.
%\begin{itemize}
%\item All lattices of dimension less than 4 are of Vornoi's first kind.
%\end{itemize}
} 

% \frame{
% \frametitle{Example: The root lattice $A_n$}
% \begin{itemize}
% \item The root lattice $A_n$ is of Voronoi's first kind.
% \item An obtuse superbase is all cyclic shifts of the vector 
% \[
% \left[ \begin{array}{rrrrrrr} 1 & -1 & 0 & 0 & \cdots & 0 \end{array}\right]
% \] 
% from $\reals^{n+1}$, that is,
% \begin{align*}
% b_1 &= \left[ \begin{array}{rrrrrrr} 1 & -1 & 0 & 0 & \cdots & 0 \end{array}\right] \\
% b_2 &= \left[ \begin{array}{rrrrrrr} 0 & 1 & -1 & 0 & \cdots & 0 \end{array}\right] \\
% &\vdots \\
% b_{n+1} &= \left[ \begin{array}{rrrrrrr} -1 & 0 & 0 & \cdots & 0 & 1 \end{array}\right].
% \end{align*}
% \item The Selling parameters are 
% \[
% q_{ij} = \dotprod{b_i}{b_j} = \begin{cases} 2, & i = j \\
% -1, & i - j \equiv 1 \bmod{n+1} \\
% 0, & \text{otherwise}.
% \end{cases}
% \]
% \end{itemize}
% }



\frame{
\frametitle{Lattices of Voronoi's first kind}
\begin{theorem}
Let $\Lambda$ be a $n$-dimensional lattice of Voronoi's first kind with obtuse superbase $b_1, \dots, b_{n+1}$.  The relevant vectors in $\Lambda$ are of the form 
\[
\sum_{i \in I} b_i
\] 
where $I$ is a strict subset of $\{1, 2, \dots, n+1\}$ and $I$ is not empty.
\end{theorem}

\begin{corollary}
Short vectors in $\Lambda$ are of the form $\sum_{i \in I} b_i$.
\end{corollary}
}

\frame{
\frametitle{Lattices of Voronoi's first kind}
A na\"{i}ve way to compute a short vector is to compute
\[
\| \sum_{i \in I} b_i \|^2
\]
for all of the $2^{n+1} - 2$ possible subsets $I$.  
\begin{itemize}
\item Requires a number of operations that grows exponentially with the dimension $n$.
\item We can improve this using a \term{minimum cut algorithm}.
\end{itemize}
}

\section{Graphs, cuts, and minimum cuts}
\frame{
\frametitle{Graphs, cuts, and minimum cuts}
Let $G$ be a weighted graph with:
\begin{itemize}
\item $n+1$ vertices $v_1, \dots, v_{n+1}$,  
\item edges $e_{ij}$ connecting vertex $v_i$ to vertex $v_j$,
\item edge weights $w_{ij} \in \reals$.
\end{itemize} 
}

\frame{
\begin{figure}[tp]
	\centering 
        \includegraphics{figs/graphs-3.mps}
		\caption{A graph with 4 vertices and 4 weighted edges.}
		\label{fig:graphexample}
\end{figure}
}

\frame{
\frametitle{Graphs, cuts, and minimum cuts}
A \term{cut} in $G$ is a partition of the vertices into two nonempty sets $C$ and its complement $\bar{C}$.
\begin{itemize}
\item The \term{weight} of a cut is the sum of the weights on the edges crossing from the vertices in $C$ to the vertices in $\bar{C}$.%, that is,
%\[
%W(C,\bar{C}) = \sum_{i \in I} \sum_{j \in J} w_{ij}, 
%\]
%where $I = \{ i \mid v_i \in C\}$ and $J = \{j \mid v_j \in \bar{C}\}$.
%\item \term{Loops} and zero weight edges don't affect the weight of any cut, so we ignore them.
\item A \term{minimum cut} is a pair ($C$,$\bar{C}$) with smallest weight. %that minimise $W(C,\bar{C})$.  
\end{itemize} 
}

\frame{
\begin{figure}[tp]
	\centering 
        \includegraphics{figs/graphs-4.mps}
		\caption{A graph with 4 vertices and 4 weighted edges.}
		\label{fig:graphexample2}
\end{figure}
}

\frame{
\begin{figure}[tp]
	\centering 
        \includegraphics{figs/graphs-5.mps} 
		\caption{The cut $C = \{v_2\}$ and $\bar{C} = \{v_1,v_3,v_4 \}$}
		\label{fig:graphcut1}
\end{figure}
}

\frame{
\begin{figure}[tp]
	\centering 
        \includegraphics{figs/graphs-6.mps}
		\caption{The cut $C = \{v_2\}$ and $\bar{C} = \{v_1,v_3,v_4 \}$ has weight 5.}
		\label{fig:graphcut2}
\end{figure}
} 

\frame{
\begin{figure}[tp]
	\centering 
        \includegraphics{figs/graphs-7.mps}
		\caption{The minimum cut $C = \{v_3,v_4\}$ and $\bar{C} = \{v_1,v_2 \}$}
		\label{fig:graphcut3}
\end{figure}
}

\frame{
\begin{figure}[tp]
	\centering 
        \includegraphics{figs/graphs-8.mps}
		\caption{The minimum cut $C = \{v_3,v_4\}$ and $\bar{C} = \{v_1,v_2 \}$ has weight 2.}
		\label{fig:graphcut4}
\end{figure}
} 

\frame{
\frametitle{Graphs, cuts, and minimum cuts}
If the edge weights $w_{ij}$ are all nonnegative, a minimum cut can be computed:
\begin{itemize}
\item deterministically in $O(n^3)$ operations using the algorithm of Stoer and Wagner,
\item with high probability in $O(n^2 \log(n)^3)$ operations using the randomised algorithm of Karger and Stien.
\end{itemize}	
}
 
     
\frame{
\begin{theorem}
Let $\Lambda$ be a $n$-dimensional lattice of Voronoi's first kind with obtuse superbase 
\[
b_1, \dots, b_{n+1}.
\]  
Let $G$ be a graph with $n+1$ vertices $v_{1}, \dots, v_{n+1}$ and edge weights 
\[
w_{ij} = -q_{ij} = -b_i \cdot b_j \geq 0 \qquad i \neq j.
\]  
Let $(C, \bar{C})$ be a minimum cut in $G$.  A short vector in $\Lambda$ is 
\[
\sum_{i \in I} b_i \;\;\; \text{where} \;\;\; I = \{ i \mid v_i \in C\}.
\]  
The squared length of the short vector is given by the weight of the minimum cut. %$W(C,\bar{C})$.
\end{theorem}
}

\section{Examples}

% \frame{
% \frametitle{Example: The root lattice $A_n$}
% \begin{itemize} 
% \item Recall the root lattice $A_n$ of Voronoi's first kind.
% \item The Selling parameters are
% \[ 
% q_{ij} = \dotprod{b_i}{b_j} = \begin{cases} 2, & i = j \\
% -1, & i - j \equiv 1 \bmod{n+1} \\
% 0, & \text{otherwise}.
% \end{cases}
% \]  
% \item The corresponding graph has positive edge weights
% \[
% w_{ij} = 1 \qquad i - j \equiv 1 \bmod{n+1}.
% \]
% \item This is the cycle graph with $n+1$ vertices and weight 1 on every edge.
% \end{itemize}
% }

% \frame{
% \begin{figure}[tp]
% 	\centering 
%         \includegraphics{cyclegraph-1.mps}
% 		\caption{The cycle graph corresponding to $A_n$.}
% 		\label{fig:graphcycleAn}  
% \end{figure}
% }

% \frame{
% \begin{figure}[tp]
% 	\centering 
%         \includegraphics{cyclegraph-2.mps}
% 		\caption{Cut corresponding with short vector $b_1 = [1,-1,0,\dots,0]$}
% 		\label{fig:graphcycleAn}  
% \end{figure}
% } 

% \frame{
% \begin{figure}[tp]
% 	\centering 
%         \includegraphics{cyclegraph-3.mps} 
% 		\caption{Cut corresponding with $b_1+b_{2}+b_3 = [1,0,0,-1\dots,0]$}
% 		\label{fig:graphcycleAn}  
% \end{figure}
% } 

\frame{
\frametitle{An example}
Consider again the 3-dimensional lattice with obtuse superbase
\[
\begin{split}
b_1 &= \left[ \begin{array}{rrr} 2 & -1 & 0 \end{array}\right] \\
b_2 &= \left[ \begin{array}{rrr}  -1 & 2 & 0  \end{array}\right] \\
b_3 &= \left[ \begin{array}{rrr}  0 & 0 & 2   \end{array}\right] \\
b_4 &= \left[ \begin{array}{rrr} -1 &  -1  & -2 \end{array}\right].
\end{split} 
\]
The Selling parameters are given in matrix form as
\[
\left[ \begin{array}{cccc} 
q_{11} & q_{12} & q_{13} & q_{14}\\
q_{21} & q_{22} & q_{23} & q_{24} \\
q_{31} & q_{32} & q_{33} & q_{34} \\
q_{41} & q_{42} & q_{43} & q_{44}
 \end{array}\right]
=
\left[ \begin{array}{rrrr} 
5 & -4 & 0 & -1 \\
-4 & 5 & 0 & -1 \\
0 & 0 & 4 & -4 \\
-1 & -1 & -4 & 6
 \end{array}\right].
\]
}

\frame{
\begin{figure}[tp]
	\centering 
        \includegraphics{figs/graphs-4.mps}
		\caption{We have seen this graph before!}
		%\label{fig:graphcut4}
\end{figure}
} 


\frame{
\begin{figure}[tp]
	\centering 
        \includegraphics{figs/graphs-8.mps}
		\caption{The minimum cut $C = \{v_3,v_4\}$ and $\bar{C} = \{v_1,v_2 \}$ has weight 2.}
		%\label{fig:graphcut4}
\end{figure}
} 

\frame{
\frametitle{An example}
The minimum cut corresponds with the short vectors
\[
b_1 + b_2 = [1,1,0]
\]
and 
\[
b_3 + b_4 = -b_1 - b_2 = [-1,-1,0]
\]
of squared length 2.
} 

\frame{
\frametitle{Some questions we asked in 2012}

\begin{itemize}
\item Can we efficiently decide whether a lattice is of Voronoi's first kind?
\item Can we efficiently find an obtuse superbase if it exists?
\item Can a similar approach be taken to solve the \term{nearest lattice point problem}?
\end{itemize}

}

\frame{
\frametitle{Some questions we asked in 2012}

\begin{itemize}
\item Can we \sout{efficiently} decide whether a lattice is of Voronoi's first kind? {\color{red}\textbf{Yes}}
\item Can we \sout{efficiently} find an obtuse superbase if it exists? {\color{red}\textbf{Yes}}
\item Can a similar approach be taken to solve the \term{closest lattice point problem}? 
\end{itemize}

}

\frame{
\frametitle{Some questions we asked in 2012}

\begin{itemize}
\item Can we \sout{efficiently} decide whether a lattice is of Voronoi's first kind? {\color{red}\textbf{Yes}}
\item Can we \sout{efficiently} find an obtuse superbase if it exists? {\color{red}\textbf{Yes}}
\item Can a similar approach be taken to solve the \term{closest lattice point problem}? {\color{red}\textbf{$O(n^4)$}}
\end{itemize}

}

\frame{
START TALKING ABOUT CLOSEST POINT ALGORITHM
}


\section{What now?}
\frame{
\frametitle{What now?}

\begin{itemize}
\item Can good codes or quantisers be constructed from lattices of Voronoi's first kind?
\item Signal processing applications, phase unwrapping, circular statistics, etc.
\end{itemize}

}



\end{document}