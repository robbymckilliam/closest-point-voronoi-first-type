\documentclass{beamer}

\input{presentationdefs}

\input{../mathbf-abbrevs.sty}

\usepackage{ulem}

\usetheme{default}

\setbeamertemplate{footline}[frame number]
\setbeamertemplate{navigation symbols}{}

%\usepackage{beamerthemesplit}

%\usetheme[width=2.8cm, hideothersubsections]{Goettingen}
%\usetheme[width=2.8cm]{Goettingen}

\setbeamercovered{transparent=20}


\title{Finding shortest and closest vectors in a lattice of Voronoi's first kind}
\author{Robby McKilliam \\
%\vspace{0.2cm}
Alex Grant \\ Vaughan Clarkson \\
\vspace{0.7cm}
{\small
Institute for Telecommunications Research \\ University of South Australia 
\\ \vspace{0.2cm}
School of Information Technology and Electrical Engineering \\ The University of Queensland
}
}

\date{\today}


\begin{document}

\frame{\titlepage}


\frame{\tableofcontents}


\section{Lattices}

\frame{
\frametitle{Lattices}
An $n$-dimensional \term{lattice} $\Lambda$ is a discrete set of vectors from $\reals^m$, $m \geq n$, given by
\[
  \Lambda = \{ b_1 u_1 + b_2u_2 + \dots + b_n u_n \mid u_1, \dots u_n \in \ints \},
\] 
where $b_1, \dots, b_n \in \reals^m$ are \term{basis vectors} of $\Lambda$.
}

\frame{
\begin{figure}[tp]
	\centering 
		\includegraphics{figs/lattice-1.mps}
		\caption{A 2-dimensional lattice.}
		%\label{fig:latfig1}
\end{figure}
}

\frame{
\frametitle{Short vectors}
Those lattice points with smallest non-zero length are called \term{short vectors}.  That is, the short vectors have squared length
\[
\min_{x\in \Lambda \backslash \{ 0 \} } \| x \|^2.
\]
}

\frame{
\begin{figure}[tp]
	\centering 
		\includegraphics{figs/lattice-1.mps}
		\caption{A 2-dimensional lattice.}
		%\label{fig:latfig2}
\end{figure}
}

\frame{
\begin{figure}[tp]
	\centering 
		\includegraphics{figs/latticewithspherepacking-1.mps}
		\caption{A 2-dimensional lattice.  There are 4 short vectors.}
		%\label{fig:latfigandpacking1}
\end{figure}
}

\frame{
\frametitle{The shortest vector problem}
\begin{itemize}
\item Computing a short vector is called the \term{shortest vector problem}.
\item Applications in cryptography and number theory.
\item NP-hard for arbitrary lattices.
\item Easier for specific lattices.
\item For example, short vectors are easy to find in the \term{root lattices} $\ints^n$, $A_n$, and $D_n$.
\item We will show that the problem is relatively easy to solve for lattices of \term{Voronoi's first kind}.
\end{itemize}
}
 
\frame{ 
\frametitle{The Voronoi cell}

The \term{Voronoi cell} of a lattice $\Lambda \subset \reals^m$ is the subset of $\reals^m$ at least as close to the origin than to any lattice point,
\[
\vor(\Lambda) = \{ x \in \reals^m \mid \|x\| \leq \|x - y\|, y \in \Lambda \}.
\]

}  

\frame{ 
\begin{figure}[tp]
	\centering 
		\includegraphics{figs/voronoicell-1.mps}
		\caption{A 2-dimensional lattice and its Voronoi cell.}
		%\label{fig:latfigandpacking1} 
\end{figure}
}

\frame{ 
\frametitle{Relevant vectors}

The \term{relevant vectors} of a lattice $\Lambda$ are those which contribute a face to the Voronoi cell.

}

\frame{
\begin{figure}[tp] 
	\centering 
		\includegraphics{figs/relevantvectors-1.mps}
		\caption{A 2-dimensional lattice with 6 relevant vectors.}
		%\label{fig:latfigandpacking1}
\end{figure}
}

\frame{ 
\frametitle{Relevant vectors}
\begin{itemize}
\item The relevant vectors are those $v \in \Lambda \backslash \{0\}$ such that  
\[
\dotprod{v}{x} \leq \dotprod{x}{x} \qquad \text{for all $x \in \Lambda$.}
\]
\item The Voronoi cell can be defined using the relevant vectors, 
\[
\vor(\Lambda) = \{ x \in\reals^m \mid \|x\| \leq \|x - v\|, v \in\relevant(\Lambda) \}.
\]
\item Short vectors are relevant vectors.
\end{itemize}
}

\frame{
\begin{figure}[tp] 
	\centering 
		\includegraphics{figs/relevantvectorsandspherepacking-1.mps}   
		\caption{A lattice with 6 relevant vectors and 4 short vectors.}
		%\label{fig:latfigandpacking1}
\end{figure}
}

\frame{
\frametitle{The closest lattice point problem}

Given a lattice $\Lambda \subset \reals^m$ and a vector $y \in \reals^m$ find $x \in \Lambda$ such that
\[
\| y - x \|^2
\] 
is minimised.  

\begin{itemize}
\item This is called the \term{closest lattice point problem} and a solution is called a \term{closest lattice point} to $y$.
\item The lattice point $x \in \Lambda$ is closest to $y \in \reals^m$ if and only if
\[
y \in \vor(\Lambda) + x.
\]
\end{itemize}

}


\frame{
\begin{figure}[tp]  
	\centering 
		\includegraphics{figs/closestpoint-1.mps}   
		\caption{The closest lattice point.}
		%\label{fig:latfigandpacking1} 
\end{figure}
}

% \frame{
% \frametitle{The closest lattice point problem}
% \begin{itemize}
% \item Equivalently $x$ is closest to $y$ if and only if
% \[
% \dotprod{(y - x)}{v} \leq \tfrac{1}{2} \dotprod{v}{v}
% \]
% for all $v \in \relevant(\Lambda)$.
% \end{itemize}
% }

\frame{
\frametitle{The closest lattice point problem}
Applications to:
\begin{itemize}
\item coding and quantisation,
\item multi-antenna communications (MIMO),
\item unwrapping of phase data for electronic distance measurement in GPS and surveying,
\item single frequency estimation,
\item polynomial phase estimation,
\item circular statistics.
\end{itemize}
}
 
\frame{
\frametitle{The closest lattice point problem}
\begin{itemize}
\item NP-hard for arbitrary lattices.
\item Easier for specific lattices.
\item For example, fast algorithms exist for the \term{root lattices} $\ints^n$, $A_n$, and $D_n$.
\item We will describe a fast algorithm to compute a closest point in lattices of \term{Voronoi's first kind}.
\end{itemize}
}


  
\section{Lattices of Voronoi's first kind}
\frame{
\frametitle{Lattices of Voronoi's first kind}
An $n$-dimensional lattice $\Lambda$ is of \term{Voronoi's first kind} if it has an \term{obtuse superbase}, that is, a set of $n+1$ vectors 
\[
b_1,\dots,b_{n+1}
\] 
such that
\begin{itemize}
\item $b_1,\dots,b_n$ are a basis for $\Lambda$,
\item $b_1 + b_2 \dots + b_{n+1} = 0$ \hspace{0.2cm} (the superbase condition),
\item $q_{ij} = b_i \cdot b_j \leq 0$ whenever $i \neq j$ \hspace{0.2cm} (the obtuse condition).
\end{itemize}
\vspace{0.2cm}
The $q_{ij}$ are called \term{Selling parameters}.
%\begin{itemize}
%\item Conway and Sloane have shown that every lattice of dimensional less than 3 is of Vornoi's first kind.
%\end{itemize}
}

\frame{
\frametitle{An example}
Consider the 3-dimensional lattice with basis
\[
\begin{split}
b_1 &= \left[ \begin{array}{rrr} 2 & -1 & 0 \end{array}\right] \\
b_2 &= \left[ \begin{array}{rrr}  -1 & 2 & 0  \end{array}\right] \\
b_3 &= \left[ \begin{array}{rrr}  0 & 0 & 2   \end{array}\right].
\end{split}
\]
Define a $4$th vector as
\[
b_4 = -b_1 - b_2 - b_3 =  \left[ \begin{array}{rrr} -1 &  -1  & -2 \end{array}\right],
\]
so that $b_1, b_2, b_3, b_4$ satisfy the superbase condition.
}

\frame{
\frametitle{An example}
The Selling parameters can be written in a matrix
\[
\left[ \begin{array}{cccc} 
q_{11} & q_{12} & q_{13} & q_{14}\\
q_{21} & q_{22} & q_{23} & q_{24} \\
q_{31} & q_{32} & q_{33} & q_{34} \\
q_{41} & q_{42} & q_{43} & q_{44}
 \end{array}\right]
=
\left[ \begin{array}{rrrr} 
5 & -4 & 0 & -1 \\
-4 & 5 & 0 & -1 \\
0 & 0 & 4 & -4 \\
-1 & -1 & -4 & 6
 \end{array}\right].
\]
The off diagonal elements are not positive so the obtuse condition is satisfied.
%\begin{itemize}
%\item All lattices of dimension less than 4 are of Vornoi's first kind.
%\end{itemize}
} 

% \frame{
% \frametitle{Example: The root lattice $A_n$}
% \begin{itemize}
% \item The root lattice $A_n$ is of Voronoi's first kind.
% \item An obtuse superbase is all cyclic shifts of the vector 
% \[
% \left[ \begin{array}{rrrrrrr} 1 & -1 & 0 & 0 & \cdots & 0 \end{array}\right]
% \] 
% from $\reals^{n+1}$, that is,
% \begin{align*}
% b_1 &= \left[ \begin{array}{rrrrrrr} 1 & -1 & 0 & 0 & \cdots & 0 \end{array}\right] \\
% b_2 &= \left[ \begin{array}{rrrrrrr} 0 & 1 & -1 & 0 & \cdots & 0 \end{array}\right] \\
% &\vdots \\
% b_{n+1} &= \left[ \begin{array}{rrrrrrr} -1 & 0 & 0 & \cdots & 0 & 1 \end{array}\right].
% \end{align*}
% \item The Selling parameters are 
% \[
% q_{ij} = \dotprod{b_i}{b_j} = \begin{cases} 2, & i = j \\
% -1, & i - j \equiv 1 \bmod{n+1} \\
% 0, & \text{otherwise}.
% \end{cases}
% \]
% \end{itemize}
% }



\frame{
\frametitle{Lattices of Voronoi's first kind}
\begin{theorem}
Let $\Lambda$ be a $n$-dimensional lattice of Voronoi's first kind with obtuse superbase $b_1, \dots, b_{n+1}$.  The relevant vectors in $\Lambda$ are of the form 
\[
\sum_{i \in I} b_i
\] 
where $I$ is a strict subset of $\{1, 2, \dots, n+1\}$ and $I$ is not empty.
\end{theorem}

\begin{corollary}
Short vectors in $\Lambda$ are of the form $\sum_{i \in I} b_i$.
\end{corollary}
}

\frame{
\frametitle{Lattices of Voronoi's first kind}
A na\"{i}ve way to compute a short vector is to compute
\[
\| \sum_{i \in I} b_i \|^2
\]
for all of the $2^{n+1} - 2$ possible subsets $I$.  
\begin{itemize}
\item Requires a number of operations that grows exponentially with the dimension $n$.
\item We can improve this using a \term{minimum cut algorithm}.
\end{itemize}
}

\section{Graphs, cuts, and minimum cuts}
\frame{
\frametitle{Graphs, cuts, and minimum cuts}
Let $G$ be a weighted graph with:
\begin{itemize}
\item $n+1$ vertices $v_1, \dots, v_{n+1}$,  
\item edges $e_{ij}$ connecting vertex $v_i$ to vertex $v_j$,
\item edge weights $w_{ij} \in \reals$.
\end{itemize} 
}

\frame{
\begin{figure}[tp]
	\centering 
        \includegraphics{figs/graphs-3.mps}
		\caption{A graph with 4 vertices and 4 weighted edges.}
		\label{fig:graphexample}
\end{figure}
}

\frame{
\frametitle{Graphs, cuts, and minimum cuts}
A \term{cut} in $G$ is a partition of the vertices into two nonempty sets $C$ and its complement $\bar{C}$.
\begin{itemize}
\item The \term{weight} of a cut is the sum of the weights on the edges crossing from the vertices in $C$ to the vertices in $\bar{C}$.%, that is,
%\[
%W(C,\bar{C}) = \sum_{i \in I} \sum_{j \in J} w_{ij}, 
%\]
%where $I = \{ i \mid v_i \in C\}$ and $J = \{j \mid v_j \in \bar{C}\}$.
%\item \term{Loops} and zero weight edges don't affect the weight of any cut, so we ignore them.
\item A \term{minimum cut} is a pair ($C$,$\bar{C}$) with smallest weight. %that minimise $W(C,\bar{C})$.  
\end{itemize} 
}

\frame{
\begin{figure}[tp]
	\centering 
        \includegraphics{figs/graphs-4.mps}
		\caption{A graph with 4 vertices and 4 weighted edges.}
		\label{fig:graphexample2}
\end{figure}
}

\frame{
\begin{figure}[tp]
	\centering 
        \includegraphics{figs/graphs-5.mps} 
		\caption{The cut $C = \{v_2\}$ and $\bar{C} = \{v_1,v_3,v_4 \}$}
		\label{fig:graphcut1}
\end{figure}
}

\frame{
\begin{figure}[tp]
	\centering 
        \includegraphics{figs/graphs-6.mps}
		\caption{The cut $C = \{v_2\}$ and $\bar{C} = \{v_1,v_3,v_4 \}$ has weight 5.}
		\label{fig:graphcut2}
\end{figure}
} 

\frame{
\begin{figure}[tp]
	\centering 
        \includegraphics{figs/graphs-7.mps}
		\caption{The minimum cut $C = \{v_3,v_4\}$ and $\bar{C} = \{v_1,v_2 \}$}
		\label{fig:graphcut3}
\end{figure}
}

\frame{
\begin{figure}[tp]
	\centering 
        \includegraphics{figs/graphs-8.mps}
		\caption{The minimum cut $C = \{v_3,v_4\}$ and $\bar{C} = \{v_1,v_2 \}$ has weight 2.}
		\label{fig:graphcut4}
\end{figure}
} 

\frame{
\frametitle{Graphs, cuts, and minimum cuts}
If the edge weights $w_{ij}$ are all nonnegative, a minimum cut can be computed:
\begin{itemize}
\item deterministically in $O(n^3)$ operations using the algorithm of Stoer and Wagner,
\item with high probability in $O(n^2 \log(n)^3)$ operations using the randomised algorithm of Karger and Stien.
\end{itemize}	
}
 
     
\frame{
\begin{theorem}
Let $\Lambda$ be a $n$-dimensional lattice of Voronoi's first kind with obtuse superbase 
\[
b_1, \dots, b_{n+1}.
\]  
Let $G$ be a graph with $n+1$ vertices $v_{1}, \dots, v_{n+1}$ and edge weights 
\[
w_{ij} = -q_{ij} = -b_i \cdot b_j \geq 0 \qquad i \neq j.
\]  
Let $(C, \bar{C})$ be a minimum cut in $G$.  A short vector in $\Lambda$ is 
\[
\sum_{i \in I} b_i \;\;\; \text{where} \;\;\; I = \{ i \mid v_i \in C\}.
\]  
The squared length of the short vector is given by the weight of the minimum cut. %$W(C,\bar{C})$.
\end{theorem}
}

%\section{Examples}

% \frame{
% \frametitle{Example: The root lattice $A_n$}
% \begin{itemize} 
% \item Recall the root lattice $A_n$ of Voronoi's first kind.
% \item The Selling parameters are
% \[ 
% q_{ij} = \dotprod{b_i}{b_j} = \begin{cases} 2, & i = j \\
% -1, & i - j \equiv 1 \bmod{n+1} \\
% 0, & \text{otherwise}.
% \end{cases}
% \]  
% \item The corresponding graph has positive edge weights
% \[
% w_{ij} = 1 \qquad i - j \equiv 1 \bmod{n+1}.
% \]
% \item This is the cycle graph with $n+1$ vertices and weight 1 on every edge.
% \end{itemize}
% }

% \frame{
% \begin{figure}[tp]
% 	\centering 
%         \includegraphics{cyclegraph-1.mps}
% 		\caption{The cycle graph corresponding to $A_n$.}
% 		\label{fig:graphcycleAn}  
% \end{figure}
% }

% \frame{
% \begin{figure}[tp]
% 	\centering 
%         \includegraphics{cyclegraph-2.mps}
% 		\caption{Cut corresponding with short vector $b_1 = [1,-1,0,\dots,0]$}
% 		\label{fig:graphcycleAn}  
% \end{figure}
% } 

% \frame{
% \begin{figure}[tp]
% 	\centering 
%         \includegraphics{cyclegraph-3.mps} 
% 		\caption{Cut corresponding with $b_1+b_{2}+b_3 = [1,0,0,-1\dots,0]$}
% 		\label{fig:graphcycleAn}  
% \end{figure}
% } 

\frame{
\frametitle{An example}
Consider again the 3-dimensional lattice with obtuse superbase
\[
\begin{split}
b_1 &= \left[ \begin{array}{rrr} 2 & -1 & 0 \end{array}\right] \\
b_2 &= \left[ \begin{array}{rrr}  -1 & 2 & 0  \end{array}\right] \\
b_3 &= \left[ \begin{array}{rrr}  0 & 0 & 2   \end{array}\right] \\
b_4 &= \left[ \begin{array}{rrr} -1 &  -1  & -2 \end{array}\right].
\end{split} 
\]
The Selling parameters are given in matrix form as
\[
\left[ \begin{array}{cccc} 
q_{11} & q_{12} & q_{13} & q_{14}\\
q_{21} & q_{22} & q_{23} & q_{24} \\
q_{31} & q_{32} & q_{33} & q_{34} \\
q_{41} & q_{42} & q_{43} & q_{44}
 \end{array}\right]
=
\left[ \begin{array}{rrrr} 
5 & -4 & 0 & -1 \\
-4 & 5 & 0 & -1 \\
0 & 0 & 4 & -4 \\
-1 & -1 & -4 & 6
 \end{array}\right].
\]
}

\frame{
\begin{figure}[tp]
	\centering 
        \includegraphics{figs/graphs-4.mps}
		\caption{We have seen this graph before!}
		%\label{fig:graphcut4}
\end{figure}
} 


\frame{
\begin{figure}[tp]
	\centering 
        \includegraphics{figs/graphs-8.mps}
		\caption{The minimum cut $C = \{v_3,v_4\}$ and $\bar{C} = \{v_1,v_2 \}$ has weight 2.}
		%\label{fig:graphcut4}
\end{figure}
} 

\frame{
\frametitle{An example}
The minimum cut corresponds with the short vectors
\[
b_1 + b_2 = [1,1,0]
\]
and 
\[
b_3 + b_4 = -b_1 - b_2 = [-1,-1,0]
\]
of squared length 2.
} 

\frame{
\frametitle{Some questions we asked in 2012}

\begin{itemize}
\item Can we efficiently decide whether a lattice is of Voronoi's first kind?
\item Can we efficiently find an obtuse superbase if it exists?
\item Can a similar approach be taken to solve the \term{closest lattice point problem}?
\end{itemize}

}

\frame{
\frametitle{Some questions we asked in 2012}

\begin{itemize}
\item Can we \sout{efficiently} decide whether a lattice is of Voronoi's first kind? {\color{red}\textbf{Yes}}
\item Can we \sout{efficiently} find an obtuse superbase if it exists? {\color{red}\textbf{Yes}}
\item Can a similar approach be taken to solve the \term{closest lattice point problem}? 
\end{itemize}

}

\frame{
\frametitle{Some questions we asked in 2012}

\begin{itemize}
\item Can we \sout{efficiently} decide whether a lattice is of Voronoi's first kind? {\color{red}\textbf{Yes}}
\item Can we \sout{efficiently} find an obtuse superbase if it exists? {\color{red}\textbf{Yes}}
\item Can a similar approach be taken to solve the \term{closest lattice point problem}? {\color{red}\textbf{$O(n^4)$}}
\end{itemize}

}

\section{A series of relevant vectors}

\frame{
\frametitle{A series of relevant vectors}

Let $x_0$ be some lattice point from $\Lambda$ and consider the following iteration,
\begin{align*}
x_{k+1} &= x_k + v_k \\
v_k &= \arg\min_{ v \in \relevant(\Lambda) \cup \{0\} } \|y - x_k - v \|
\end{align*} 
}

\frame{
\begin{figure}[tp]
	\centering 
        \includegraphics{figs/seriesofrelevantvectors-1.mps}
		\caption{Computing a closest point by a series of relevant vectors.}
		%\label{fig:graphcut4}
\end{figure}
}

\newtheorem{proposition}{Proposition}

\frame{
\frametitle{A series of relevant vectors}
 \begin{proposition}
 There is a finite number $K$ such that 
\[
x_K, x_{K+1}, x_{K+2}, \dots
\]
 are all closest points to $y$.
 \end{proposition}
}

\frame{
\frametitle{A series of relevant vectors}
\begin{itemize}
\item The number $K$ of iterations might be large.
\item The initial point $x_0$ affects $K$.
\item Minimising over the set of relevant vectors, that is computing
\[
\arg\min_{ v \in \relevant(\Lambda) \cup \{0\} } \|y - x_k - v \|
\]
might be expensive.
\item There are as many as $2^{n+1}-2$ relevant vectors.
\end{itemize}
}


\frame{
\frametitle{A series of relevant vectors}
For a lattice of Voronoi's first kind:
\begin{itemize}
\item $x_0$ can be chosen to ensure that a closest lattice point in found after at most $K \leq n$ iterations.
\item Minimisation over the set of relevant vectors can be performed by computing a minimum cut in a flow network.
\item Using known algorithms a minimum cut can be found in $O(n^3)$ operations.
\item In total $O(n^4)$ operations are required to compute a closest lattice point.
\end{itemize}
}

\frame{
%\frametitle{A series of relevant vectors}

\begin{theorem}
Let $\Lambda$ be a $n$-dimensional lattice of Voronoi's first kind with obtuse superbase $b_1, \dots, b_{n+1}$.  Let $z_1,\dots,z_{n+1} \in \reals$ minimise
\[
\| y - \sum_{n=1}^{n+1} b_i z_i \| 
\]
and put
\[
x_0 = \sum_{n=1}^{n+1} b_i \floor{z_i}.
\]
The iterative procedure, initialized at $x_0$, converges to a closest lattice point in at most $K \leq n$ iterations.
\end{theorem}
}



\section{What now?}
\frame{
\frametitle{What now?}

\begin{itemize}
\item Can good codes or quantisers be constructed from lattices of Voronoi's first kind?
\item Do applications such as global position, phase unwrapping, circular statistics, etc., involve lattices of Voronoi's first kind?
\item Are there subfamilies of Voronoi's first kind that admit even faster algorithms?
\item Are there other families of lattices for which similar techniques lead to polynomial time algorithms?
\end{itemize}

}



\end{document}