%some math functions and symbols
\newcommand{\round}[1]{\left\lceil #1 \right\rfloor}
\newcommand{\floor}[1]{\left\lfloor #1 \right\rfloor}
\newcommand{\ceil}[1]{\left\lceil #1 \right\rceil}
\newcommand{\reals}{{\mathbb R}}
\newcommand{\ints}{{\mathbb Z}}
\newcommand{\complex}{{\mathbb C}}
\newcommand{\integers}{{\mathbb Z}}
\newcommand{\sign}{\mathtt{sign}}
\newcommand{\NP}{\operatorname{NearestPt}}
\newcommand{\NS}{\operatorname{NearestSet}}
\newcommand{\bres}{\operatorname{Bres}}
\newcommand{\vol}{\operatorname{vol}}
\newcommand{\vor}{\operatorname{Vor}}
\newcommand{\relevant}{\operatorname{Rel}}
\newcommand{\coef}{\operatorname{coef}}
\newcommand{\eval}{\operatorname{eval}}
\newcommand{\Int}{\operatorname{Int}}
\newcommand{\pvec}{\operatorname{vec}}
\newcommand{\rem}{\operatorname{rem}}
\newcommand{\var}{\operatorname{var}}
\newcommand{\covar}{\operatorname{covar}}
\newcommand{\erf}{\operatorname{erf}}
\newcommand{\adj}{\operatorname{adj}}
\newcommand{\pad}{\operatorname{pad}}
\newcommand{\dealias}{\operatorname{dealias}}


%distribution fucntions
\newcommand{\projnorm}{\operatorname{ProjectedNormal}}
\newcommand{\vonmises}{\operatorname{VonMises}}
\newcommand{\wrapnorm}{\operatorname{WrappedNormal}}
\newcommand{\wrapunif}{\operatorname{WrappedUniform}}

%sorting and selecting
\newcommand{\selectindicies}{\operatorname{selectindices}}
\newcommand{\sortindicies}{\operatorname{sortindices}}
\newcommand{\largest}{\operatorname{largest}}
\newcommand{\quickpartition}{\operatorname{quickpartition}}
\newcommand{\quickpartitiontwo}{\operatorname{quickpartition2}}

%caligraphic letters, b means bold.
\newcommand{\bcalL}{\bm{\mathcal{L}}}
\newcommand{\bcalX}{\bm{\mathcal{X}}}
\newcommand{\bcalP}{\bm{\mathcal{P}}}
\newcommand{\calP}{\mathcal{P}}
\newcommand{\calR}{\mathcal{R}}
\newcommand{\calV}{\mathcal{V}}

%Brackets
\newcommand{\br}[1]{{\left( #1 \right)}}
\newcommand{\sqbr}[1]{{\left[ #1 \right]}}
\newcommand{\cubr}[1]{{\left\{ #1 \right\}}}
\newcommand{\abr}[1]{\left< #1 \right>}
\newcommand{\abs}[1]{{\left| #1 \right|}}
\newcommand{\ceiling}[1]{{\left\lceil #1 \right\rceil}}
\newcommand{\magn}[1]{\left\| #1 \right\|}
\newcommand{\fracpart}[1]{\left< #1 \right>}
\newcommand{\dotprod}[2]{ #1 \cdot #2}

\definecolor{darkgreen}{rgb}{0,0.5,0}
\newcommand{\term}[1]{{\color{darkgreen}\textbf{#1}}}
\definecolor{red}{rgb}{0.8,0.0,0}
\newcommand{\redsix}{{\color{red}\textbf{6}}}

%some commonly used underlined and
%hated symbols
\newcommand{\uY}{\ushort{\mbf{Y}}}
\newcommand{\ueY}{\ushort{Y}}
\newcommand{\uy}{\ushort{\mbf{y}}}
\newcommand{\uey}{\ushort{y}}
\newcommand{\ux}{\ushort{\mbf{x}}}
\newcommand{\uex}{\ushort{x}}
\newcommand{\uhx}{\ushort{\mbf{\hat{x}}}}
\newcommand{\uehx}{\ushort{\hat{x}}}

\newcommand {\figwidth} {100mm}
\newcommand {\Ref}[1] {Reference~\cite{#1}}
\newcommand {\Sec}[1] {Section~\ref{#1}}
\newcommand {\App}[1] {Appendix~\ref{#1}}
\newcommand {\Chap}[1] {Chapter~\ref{#1}}
\newcommand {\etal} {\emph{~et~al.}}
\newcommand {\bul} {$\bullet$ }   % bullet
\newcommand {\fig}[1] {Figure~\ref{#1}}   % references Figure x
\newcommand {\imp} {$\Rightarrow$}   % implication symbol (default)
\newcommand {\impt} {$\Rightarrow$}   % implication symbol (text mode)
\newcommand {\impm} {\Rightarrow}   % implication symbol (math mode)
\newcommand {\vect}[1] {\mathbf{#1}} 
\newcommand {\hvect}[1] {\hat{\mathbf{#1}}}
\newcommand {\del} {\partial}
\newcommand {\eqn}[1] {Equation~(\ref{#1})} 
\newcommand {\tab}[1] {Table~\ref{#1}} % references Table x
\newcommand {\half} {\frac{1}{2}} 
\newcommand {\ten}[1] {\times10^{#1}}
\newcommand {\bra}[2] {\mbox{}_{#2}\langle #1 |}
\newcommand {\ket}[2] {| #1 \rangle_{#2}}
\newcommand {\Bra}[2] {\mbox{}_{#2}\left.\left\langle #1 \right.\right|}
\newcommand {\Ket}[2] {\left.\left| #1 \right.\right\rangle_{#2}}
\newcommand {\im} {\mathrm{Im}}
\newcommand {\re} {\mathrm{Re}}
\newcommand {\braket}[4] {\mbox{}_{#3}\langle #1 | #2 \rangle_{#4}} 
%\newcommand {\dotprod}[4] {\mbox{}_{#3}\langle #1 | #2 \rangle_{#4}} 
\newcommand {\trace}[1] {\text{tr}\left(#1\right)}

% spell things correctly
\newenvironment{centre}{\begin{center}}{\end{center}}
\newenvironment{itemise}{\begin{itemize}}{\end{itemize}}


%%%%% set up the bibliography style
%\bibliographystyle{../../bib/IEEEbib}
%\bibliographystyle{uqthesis}  
						% uqthesis bibliography style file, made
			      % with makebst

%%%%% optional packages
\usepackage[square,comma,numbers,sort&compress]{natbib}
		% this is the natural sciences bibliography citation
		% style package.  The options here give citations in
		% the text as numbers in square brackets, separated by
		% commas, citations sorted and consecutive citations
		% compressed 
		% output example: [1,4,12-15]	
%\usepackage[notocbib]{apacite}
		
\usepackage{booktabs}
		%creates nice looking tables
		
%\usepackage[nottoc]{tocbibind}  
				% allows the table of contents, bibliography
				% and index to be added to the table of
				% contents if desired, the option used
				% here specifies that the table of
				% contents is not to be added.
				% tocbibind needs to be after natbib
				% otherwise bits of it get trampled.

\usepackage{amsmath,amsfonts,amssymb, amsthm, bm} % this is handy for mathematicians and physicists
			      % see http://www.ams.org/tex/amslatex.html

%\usepackage[intoc]{nomencl}
%\usepackage{showkeys} % this shows what labels you are using for cross
		      % references
		      
		 
\usepackage[vlined, linesnumbered]{algorithm2e}
	%algorithm writing package
	
\usepackage{mathrsfs}
%fancy math script

\usepackage{ushort, units}
%enable good underlining in math mode

%------------------------------------------------------------
% Theorem like environments
%
%\newtheorem{theorem}{Theorem}
%\theoremstyle{plain}
%\newtheorem{acknowledgement}{Acknowledgement}
%%\newtheorem{algorithm}{Algorithm}
%\newtheorem{axiom}{Axiom}
%\newtheorem{case}{Case}
%\newtheorem{claim}{Claim}
%\newtheorem{conclusion}{Conclusion}
%\newtheorem{condition}{Condition}
%\newtheorem{conjecture}{Conjecture}
%\newtheorem{corollary}{Corollary}
%\newtheorem{criterion}{Criterion}
%\newtheorem{definition}{Definition}
%\newtheorem{example}{Example}
%\newtheorem{exercise}{Exercise}
%\newtheorem{lemma}{Lemma}
%\newtheorem{notation}{Notation}
%\newtheorem{problem}{Problem}
%\newtheorem{proposition}{Proposition}
%\newtheorem{remark}{Remark}
%\newtheorem{solution}{Solution}
%\newtheorem{summary}{Summary}
%\numberwithin{equation}{section}
%--------------------------------------------------------
